\startmanpage
\mantitle{MPI}{tex}{1/27/2005}
\manname{MPI}
--- Introduction to the Message-Passing Interface 
\subhead{Description}
\par
MPI stands for Message Passing Interface.
MPI is a specification (like C or Fortran) and there are a number of
implementations.  The rest of this man page describes the use of the MPICH
implementation of MPI.
\par
\subhead{Getting started}
\par
Add MPI to your path
\begin{verbatim}
        % set path = ( $path /usr/local/mpi/bin )
\end{verbatim}

for the {\tt csh} and {\tt tcsh} shells, or
\begin{verbatim}
        % export path=$path:/usr/local/mpi/bin
\end{verbatim}

for {\tt sh}, {\tt ksh}, and {\tt bash} shells.
\par
Compute pi to a given resolution on 8 processes
\begin{verbatim}
        % mpiexec -n 8 /usr/local/mpi/examples/cpi
\end{verbatim}

\par
You can compile and link your own MPI programs with the commands {\tt mpicc},
{\tt mpif77}, {\tt mpicxx}, and {\tt mpif90}:
\begin{verbatim}
        % mpicc -o cpi cpi.c
        % mpif77 -o fpi fpi.f
        % mpicxx -o cxxpi cxxpi.cxx
        % mpif790 -o pi3f90 pi3f90.f90
\end{verbatim}

using the source code from {\em /usr/local/mpi/examples}.
\par
\subhead{Documentation}
\par
PDF documentation can be found in directory
{\tt /usr/local/mpi/doc/}.  These include an installation manual ({\tt install.pdf})
and a user's manual ({\tt usermanual.pdf}).
\par
Man pages exist for every MPI subroutine and function.  The man pages are
also available on the Web at {\tt http://www.mcs.anl.gov/mpi/www}.
Additional on-line information is available at {\tt http://www.mcs.anl.gov/mpi},
including a hypertext version of the standard, information on other libraries
that use MPI, and pointers to other MPI resources.
\par
\subhead{Version}
\par
MPICH2 version 1.0
\par
\subhead{License}
\par
Copyright 2002 University of Chicago.
See the file {\tt COPYRIGHT} for details.  The source code is freely available
by anonymous ftp from {\tt ftp.mcs.anl.gov} in {\tt pub/mpi/mpich2-beta.tar.gz} .
\par
\subhead{Files}
\par
\begin{verbatim}
/usr/local/mpi/                 MPI software directory
/usr/local/mpi/COPYRIGHT        Copyright notice
/usr/local/mpi/README           various notes and instructions
/usr/local/mpi/bin/             binaries, including mpiexec and mpicc
/usr/local/mpi/examples         elementary MPI programs
/usr/local/mpi/doc/             documentation
/usr/local/mpi/include/         include files
/usr/local/mpi/lib/             library files
\end{verbatim}

\par
\subhead{Contact}
\par
MPI-specific suggestions and bug reports should
be sent to {\tt mpich2-maint@mcs.anl.gov}.
\par
\location{manpage.txt}
\endmanpage
