\startmanpage
\mantitle{MPI{\tt \char`\_}Query{\tt \char`\_}thread}{tex}{12/9/2006}
\manname{MPI{\tt \char`\_}Query{\tt \char`\_}thread}
--- Return the level of thread support provided by the MPI  library 
\subhead{Synopsis}
\startvb\begin{verbatim}
int MPI_Query_thread( int *provided )

\end{verbatim}
\endvb

\subhead{Output Parameter}
\startarg{provided }{Level of thread support provided.  This is the same value
that was returned in the {\tt provided} argument in {\tt MPI{\tt \char`\_}Init{\tt \char`\_}thread}.
}
\par
\subhead{Notes}
The valid values for the level of thread support are:
\startarg{MPI{\tt \char`\_}THREAD{\tt \char`\_}SINGLE }{Only one thread will execute. 
}
\startarg{MPI{\tt \char`\_}THREAD{\tt \char`\_}FUNNELED }{The process may be multi-threaded, but only the main 
thread will make MPI calls (all MPI calls are funneled to the 
main thread). 
}
\startarg{MPI{\tt \char`\_}THREAD{\tt \char`\_}SERIALIZED }{The process may be multi-threaded, and multiple 
threads may make MPI calls, but only one at a time: MPI calls are not 
made concurrently from two distinct threads (all MPI calls are serialized). 
}
\startarg{MPI{\tt \char`\_}THREAD{\tt \char`\_}MULTIPLE }{Multiple threads may call MPI, with no restrictions. 
}
\par
If {\tt MPI{\tt \char`\_}Init} was called instead of {\tt MPI{\tt \char`\_}Init{\tt \char`\_}thread}, the level of
thread support is defined by the implementation.  This routine allows
you to find out the provided level.  It is also useful for library
routines that discover that MPI has already been initialized and
wish to determine what level of thread support is available.
\par
\subhead{Thread and Interrupt Safety}
\par
This routine is both thread- and interrupt-safe.
This means that this routine may safely be used by multiple threads and
from within a signal handler.
\par
\subhead{Notes for Fortran}
All MPI routines in Fortran (except for {\tt MPI{\tt \char`\_}WTIME} and {\tt MPI{\tt \char`\_}WTICK}) have
an additional argument {\tt ierr} at the end of the argument list.  {\tt ierr
}is an integer and has the same meaning as the return value of the routine
in C.  In Fortran, MPI routines are subroutines, and are invoked with the
{\tt call} statement.
\par
All MPI objects (e.g., {\tt MPI{\tt \char`\_}Datatype}, {\tt MPI{\tt \char`\_}Comm}) are of type {\tt INTEGER
}in Fortran.
\par
\subhead{Errors}
\par
All MPI routines (except {\tt MPI{\tt \char`\_}Wtime} and {\tt MPI{\tt \char`\_}Wtick}) return an error value;
C routines as the value of the function and Fortran routines in the last
argument.  Before the value is returned, the current MPI error handler is
called.  By default, this error handler aborts the MPI job.  The error handler
may be changed with {\tt MPI{\tt \char`\_}Comm{\tt \char`\_}set{\tt \char`\_}errhandler} (for communicators),
{\tt MPI{\tt \char`\_}File{\tt \char`\_}set{\tt \char`\_}errhandler} (for files), and {\tt MPI{\tt \char`\_}Win{\tt \char`\_}set{\tt \char`\_}errhandler} (for
RMA windows).  The MPI-1 routine {\tt MPI{\tt \char`\_}Errhandler{\tt \char`\_}set} may be used but
its use is deprecated.  The predefined error handler
{\tt MPI{\tt \char`\_}ERRORS{\tt \char`\_}RETURN} may be used to cause error values to be returned.
Note that MPI does {\em not} guarentee that an MPI program can continue past
an error; however, MPI implementations will attempt to continue whenever
possible.
\par
\startarg{MPI{\tt \char`\_}SUCCESS }{No error; MPI routine completed successfully.
}
\location{querythread.c}
\endmanpage
