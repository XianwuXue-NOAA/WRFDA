\startmanpage
\mantitle{MPI{\tt \char`\_}Win{\tt \char`\_}post}{tex}{5/8/2006}
\manname{MPI{\tt \char`\_}Win{\tt \char`\_}post}
--- Start an RMA exposure epoch 
\subhead{Synopsis}
\startvb\begin{verbatim}
int MPI_Win_post(MPI_Group group, int assert, MPI_Win win)

\end{verbatim}
\endvb

\subhead{Input parameters}
\startarg{group }{group of origin processes (handle) 
}
\startarg{assert }{Used to optimize this call; zero may be used as a default.
See notes. (integer) 
}
\startarg{win }{window object (handle) 
}
\par
\subhead{Notes}
The {\tt assert} argument is used to indicate special conditions for the
fence that an implementation may use to optimize the {\tt MPI{\tt \char`\_}Win{\tt \char`\_}post
}operation.  The value zero is always correct.  Other assertion values
may be or'ed together.  Assertions that are valid for {\tt MPI{\tt \char`\_}Win{\tt \char`\_}post} are:
\par
\startarg{MPI{\tt \char`\_}MODE{\tt \char`\_}NOCHECK }{the matching calls to {\tt MPI{\tt \char`\_}WIN{\tt \char`\_}START} have not yet 
occurred on any origin processes when the call to {\tt MPI{\tt \char`\_}WIN{\tt \char`\_}POST} is made. 
The nocheck option can be specified by a post call if and only if it is 
specified by each matching start call. 
}
\startarg{MPI{\tt \char`\_}MODE{\tt \char`\_}NOSTORE }{the local window was not updated by local stores (or 
local get or receive calls) since last synchronization. This may avoid 
the need for cache synchronization at the post call. 
}
\startarg{MPI{\tt \char`\_}MODE{\tt \char`\_}NOPUT }{the local window will not be updated by put or accumulate 
calls after the post call, until the ensuing (wait) synchronization. This 
may avoid the need for cache synchronization at the wait call. 
}
\par
\subhead{Notes for Fortran}
All MPI routines in Fortran (except for {\tt MPI{\tt \char`\_}WTIME} and {\tt MPI{\tt \char`\_}WTICK}) have
an additional argument {\tt ierr} at the end of the argument list.  {\tt ierr
}is an integer and has the same meaning as the return value of the routine
in C.  In Fortran, MPI routines are subroutines, and are invoked with the
{\tt call} statement.
\par
All MPI objects (e.g., {\tt MPI{\tt \char`\_}Datatype}, {\tt MPI{\tt \char`\_}Comm}) are of type {\tt INTEGER
}in Fortran.
\par
\subhead{Errors}
\par
All MPI routines (except {\tt MPI{\tt \char`\_}Wtime} and {\tt MPI{\tt \char`\_}Wtick}) return an error value;
C routines as the value of the function and Fortran routines in the last
argument.  Before the value is returned, the current MPI error handler is
called.  By default, this error handler aborts the MPI job.  The error handler
may be changed with {\tt MPI{\tt \char`\_}Comm{\tt \char`\_}set{\tt \char`\_}errhandler} (for communicators),
{\tt MPI{\tt \char`\_}File{\tt \char`\_}set{\tt \char`\_}errhandler} (for files), and {\tt MPI{\tt \char`\_}Win{\tt \char`\_}set{\tt \char`\_}errhandler} (for
RMA windows).  The MPI-1 routine {\tt MPI{\tt \char`\_}Errhandler{\tt \char`\_}set} may be used but
its use is deprecated.  The predefined error handler
{\tt MPI{\tt \char`\_}ERRORS{\tt \char`\_}RETURN} may be used to cause error values to be returned.
Note that MPI does {\em not} guarentee that an MPI program can continue past
an error; however, MPI implementations will attempt to continue whenever
possible.
\par
\startarg{MPI{\tt \char`\_}SUCCESS }{No error; MPI routine completed successfully.
}
\location{win{\tt \char`\_}post.c}
\endmanpage
