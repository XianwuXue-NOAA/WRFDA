\startmanpage
\mantitle{MPI{\tt \char`\_}Accumulate}{tex}{5/8/2006}
\manname{MPI{\tt \char`\_}Accumulate}
--- Accumulate data into the target process using remote  memory access  
\subhead{Synopsis}
\startvb\begin{verbatim}
int MPI_Accumulate(void *origin_addr, int origin_count, MPI_Datatype
                   origin_datatype, int target_rank, MPI_Aint
                   target_disp, int target_count, MPI_Datatype
                   target_datatype, MPI_Op op, MPI_Win win) 

\end{verbatim}
\endvb

\subhead{Input Parameters}
\startarg{origin{\tt \char`\_}addr }{initial address of buffer (choice) 
}
\startarg{origin{\tt \char`\_}count }{number of entries in buffer (nonnegative integer) 
}
\startarg{origin{\tt \char`\_}datatype }{datatype of each buffer entry (handle) 
}
\startarg{target{\tt \char`\_}rank }{rank of target (nonnegative integer) 
}
\startarg{target{\tt \char`\_}disp }{displacement from start of window to beginning of target 
buffer (nonnegative integer)  
}
\startarg{target{\tt \char`\_}count }{number of entries in target buffer (nonnegative integer) 
}
\startarg{target{\tt \char`\_}datatype }{datatype of each entry in target buffer (handle) 
}
\startarg{op }{predefined reduce operation (handle) 
}
\startarg{win }{window object (handle) 
}
\par
\subhead{Notes}
The basic components of both the origin and target datatype must be the same
predefined datatype (e.g., all {\tt MPI{\tt \char`\_}INT} or all {\tt MPI{\tt \char`\_}DOUBLE{\tt \char`\_}PRECISION}).
\par
\subhead{Notes for Fortran}
All MPI routines in Fortran (except for {\tt MPI{\tt \char`\_}WTIME} and {\tt MPI{\tt \char`\_}WTICK}) have
an additional argument {\tt ierr} at the end of the argument list.  {\tt ierr
}is an integer and has the same meaning as the return value of the routine
in C.  In Fortran, MPI routines are subroutines, and are invoked with the
{\tt call} statement.
\par
All MPI objects (e.g., {\tt MPI{\tt \char`\_}Datatype}, {\tt MPI{\tt \char`\_}Comm}) are of type {\tt INTEGER
}in Fortran.
\par
\subhead{Errors}
\par
All MPI routines (except {\tt MPI{\tt \char`\_}Wtime} and {\tt MPI{\tt \char`\_}Wtick}) return an error value;
C routines as the value of the function and Fortran routines in the last
argument.  Before the value is returned, the current MPI error handler is
called.  By default, this error handler aborts the MPI job.  The error handler
may be changed with {\tt MPI{\tt \char`\_}Comm{\tt \char`\_}set{\tt \char`\_}errhandler} (for communicators),
{\tt MPI{\tt \char`\_}File{\tt \char`\_}set{\tt \char`\_}errhandler} (for files), and {\tt MPI{\tt \char`\_}Win{\tt \char`\_}set{\tt \char`\_}errhandler} (for
RMA windows).  The MPI-1 routine {\tt MPI{\tt \char`\_}Errhandler{\tt \char`\_}set} may be used but
its use is deprecated.  The predefined error handler
{\tt MPI{\tt \char`\_}ERRORS{\tt \char`\_}RETURN} may be used to cause error values to be returned.
Note that MPI does {\em not} guarentee that an MPI program can continue past
an error; however, MPI implementations will attempt to continue whenever
possible.
\par
\startarg{MPI{\tt \char`\_}SUCCESS }{No error; MPI routine completed successfully.
}
\startarg{MPI{\tt \char`\_}ERR{\tt \char`\_}ARG }{Invalid argument.  Some argument is invalid and is not
identified by a specific error class (e.g., {\tt MPI{\tt \char`\_}ERR{\tt \char`\_}RANK}).
}
\startarg{MPI{\tt \char`\_}ERR{\tt \char`\_}COUNT }{Invalid count argument.  Count arguments must be 
non-negative; a count of zero is often valid.
}
\startarg{MPI{\tt \char`\_}ERR{\tt \char`\_}RANK }{Invalid source or destination rank.  Ranks must be between
zero and the size of the communicator minus one; ranks in a receive
({\tt MPI{\tt \char`\_}Recv}, {\tt MPI{\tt \char`\_}Irecv}, {\tt MPI{\tt \char`\_}Sendrecv}, etc.) may also be {\tt MPI{\tt \char`\_}ANY{\tt \char`\_}SOURCE}.
}
\startarg{MPI{\tt \char`\_}ERR{\tt \char`\_}TYPE }{Invalid datatype argument.  May be an uncommitted 
MPI{\tt \char`\_}Datatype (see {\tt MPI{\tt \char`\_}Type{\tt \char`\_}commit}).
}
\startarg{MPI{\tt \char`\_}ERR{\tt \char`\_}WIN }{Invalid MPI window object
}
\location{accumulate.c}
\endmanpage
