\startmanpage
\mantitle{MPI{\tt \char`\_}Comm{\tt \char`\_}create{\tt \char`\_}keyval}{tex}{7/11/2006}
\manname{MPI{\tt \char`\_}Comm{\tt \char`\_}create{\tt \char`\_}keyval}
--- Create a new attribute key  
\subhead{Synopsis}
\startvb\begin{verbatim}
int MPI_Comm_create_keyval(MPI_Comm_copy_attr_function *comm_copy_attr_fn, 
                         MPI_Comm_delete_attr_function *comm_delete_attr_fn, 
                         int *comm_keyval, void *extra_state)

\end{verbatim}
\endvb

\subhead{Input Parameters}
\startarg{copy{\tt \char`\_}fn }{Copy callback function for {\tt keyval} 
}
\startarg{delete{\tt \char`\_}fn }{Delete callback function for {\tt keyval} 
}
\startarg{extra{\tt \char`\_}state }{Extra state for callback functions 
}
\par
\subhead{Output Parameter}
\startarg{comm{\tt \char`\_}keyval }{key value for future access (integer) 
}
\par
\subhead{Notes}
Key values are global (available for any and all communicators).
\par
Default copy and delete functions are available.  These are
\startarg{MPI{\tt \char`\_}COMM{\tt \char`\_}NULL{\tt \char`\_}COPY{\tt \char`\_}FN   }{empty copy function
}
\startarg{MPI{\tt \char`\_}COMM{\tt \char`\_}NULL{\tt \char`\_}DELETE{\tt \char`\_}FN }{empty delete function
}
\startarg{MPI{\tt \char`\_}COMM{\tt \char`\_}DUP{\tt \char`\_}FN         }{simple dup function
}
\par
There are subtle differences between C and Fortran that require that the
copy{\tt \char`\_}fn be written in the same language from which {\tt MPI{\tt \char`\_}Comm{\tt \char`\_}create{\tt \char`\_}keyval
}is called.
This should not be a problem for most users; only programers using both
Fortran and C in the same program need to be sure that they follow this rule.
\par
\par
\subhead{Return value from attribute callbacks}
The MPI-2 versions of the attribute callbacks should return either
{\tt MPI{\tt \char`\_}SUCCESS} on success or a valid MPI error code or class on failure.
The MPI standard is ambiguous on this point, but as MPI-2 provides
the routines {\tt MPI{\tt \char`\_}Add{\tt \char`\_}error{\tt \char`\_}class} and {\tt MPI{\tt \char`\_}Add{\tt \char`\_}error{\tt \char`\_}code} that allow the
user to define and use MPI error codes and classes.
\par
\subhead{Thread and Interrupt Safety}
\par
This routine is thread-safe.  This means that this routine may be
safely used by multiple threads without the need for any user-provided
thread locks.  However, the routine is not interrupt safe.  Typically,
this is due to the use of memory allocation routines such as {\tt malloc
}or other non-MPICH runtime routines that are themselves not interrupt-safe.
\par
\subhead{Notes for Fortran}
All MPI routines in Fortran (except for {\tt MPI{\tt \char`\_}WTIME} and {\tt MPI{\tt \char`\_}WTICK}) have
an additional argument {\tt ierr} at the end of the argument list.  {\tt ierr
}is an integer and has the same meaning as the return value of the routine
in C.  In Fortran, MPI routines are subroutines, and are invoked with the
{\tt call} statement.
\par
All MPI objects (e.g., {\tt MPI{\tt \char`\_}Datatype}, {\tt MPI{\tt \char`\_}Comm}) are of type {\tt INTEGER
}in Fortran.
\par
\subhead{Errors}
\par
All MPI routines (except {\tt MPI{\tt \char`\_}Wtime} and {\tt MPI{\tt \char`\_}Wtick}) return an error value;
C routines as the value of the function and Fortran routines in the last
argument.  Before the value is returned, the current MPI error handler is
called.  By default, this error handler aborts the MPI job.  The error handler
may be changed with {\tt MPI{\tt \char`\_}Comm{\tt \char`\_}set{\tt \char`\_}errhandler} (for communicators),
{\tt MPI{\tt \char`\_}File{\tt \char`\_}set{\tt \char`\_}errhandler} (for files), and {\tt MPI{\tt \char`\_}Win{\tt \char`\_}set{\tt \char`\_}errhandler} (for
RMA windows).  The MPI-1 routine {\tt MPI{\tt \char`\_}Errhandler{\tt \char`\_}set} may be used but
its use is deprecated.  The predefined error handler
{\tt MPI{\tt \char`\_}ERRORS{\tt \char`\_}RETURN} may be used to cause error values to be returned.
Note that MPI does {\em not} guarentee that an MPI program can continue past
an error; however, MPI implementations will attempt to continue whenever
possible.
\par
\startarg{MPI{\tt \char`\_}SUCCESS }{No error; MPI routine completed successfully.
}
\par
\subhead{See Also}
MPI{\tt \char`\_}Comm{\tt \char`\_}free{\tt \char`\_}keyval
\nextline
\location{comm{\tt \char`\_}create{\tt \char`\_}keyval.c}
\endmanpage
