\startmanpage
\mantitle{MPI{\tt \char`\_}Comm{\tt \char`\_}get{\tt \char`\_}name}{tex}{12/9/2006}
\manname{MPI{\tt \char`\_}Comm{\tt \char`\_}get{\tt \char`\_}name}
--- Return the print name from the communicator 
\subhead{Synopsis}
\startvb\begin{verbatim}
int MPI_Comm_get_name(MPI_Comm comm, char *comm_name, int *resultlen)

\end{verbatim}
\endvb

\subhead{Input Parameter}
\startarg{comm }{Communicator to get name of (handle)
}
\par
\subhead{Output Parameters}
\startarg{comm{\tt \char`\_}name }{On output, contains the name of the communicator.  It must
be an array of size at least {\tt MPI{\tt \char`\_}MAX{\tt \char`\_}OBJECT{\tt \char`\_}NAME}.
}
\startarg{resultlen }{Number of characters in name
}
\par
\subhead{Notes}
\par
Because MPI specifies that null objects (e.g., {\tt MPI{\tt \char`\_}COMM{\tt \char`\_}NULL}) are invalid
as input to MPI routines unless otherwise specified, using {\tt MPI{\tt \char`\_}COMM{\tt \char`\_}NULL
}as input to this routine is an error.
\par
\par
\subhead{Thread and Interrupt Safety}
\par
This routine is thread and interrupt safe only if no MPI routine that
updates or frees the same MPI object may be called concurrently
with this routine.
\par
The MPI standard defined a thread-safe interface but this does not
mean that all routines may be called without any thread locks.  For
example, two threads must not attempt to change the contents of the
same {\tt MPI{\tt \char`\_}Info} object concurrently.  The user is responsible in this
case for using some mechanism, such as thread locks, to ensure that
only one thread at a time makes use of this routine.
\par
\par
\subhead{Notes for Fortran}
All MPI routines in Fortran (except for {\tt MPI{\tt \char`\_}WTIME} and {\tt MPI{\tt \char`\_}WTICK}) have
an additional argument {\tt ierr} at the end of the argument list.  {\tt ierr
}is an integer and has the same meaning as the return value of the routine
in C.  In Fortran, MPI routines are subroutines, and are invoked with the
{\tt call} statement.
\par
All MPI objects (e.g., {\tt MPI{\tt \char`\_}Datatype}, {\tt MPI{\tt \char`\_}Comm}) are of type {\tt INTEGER
}in Fortran.
\par
\subhead{Errors}
\par
All MPI routines (except {\tt MPI{\tt \char`\_}Wtime} and {\tt MPI{\tt \char`\_}Wtick}) return an error value;
C routines as the value of the function and Fortran routines in the last
argument.  Before the value is returned, the current MPI error handler is
called.  By default, this error handler aborts the MPI job.  The error handler
may be changed with {\tt MPI{\tt \char`\_}Comm{\tt \char`\_}set{\tt \char`\_}errhandler} (for communicators),
{\tt MPI{\tt \char`\_}File{\tt \char`\_}set{\tt \char`\_}errhandler} (for files), and {\tt MPI{\tt \char`\_}Win{\tt \char`\_}set{\tt \char`\_}errhandler} (for
RMA windows).  The MPI-1 routine {\tt MPI{\tt \char`\_}Errhandler{\tt \char`\_}set} may be used but
its use is deprecated.  The predefined error handler
{\tt MPI{\tt \char`\_}ERRORS{\tt \char`\_}RETURN} may be used to cause error values to be returned.
Note that MPI does {\em not} guarentee that an MPI program can continue past
an error; however, MPI implementations will attempt to continue whenever
possible.
\par
\startarg{MPI{\tt \char`\_}SUCCESS }{No error; MPI routine completed successfully.
}
\startarg{MPI{\tt \char`\_}ERR{\tt \char`\_}COMM }{Invalid communicator.  A common error is to use a null
communicator in a call (not even allowed in {\tt MPI{\tt \char`\_}Comm{\tt \char`\_}rank}).
}
\location{comm{\tt \char`\_}get{\tt \char`\_}name.c}
\endmanpage
