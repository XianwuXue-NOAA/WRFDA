\startmanpage
\mantitle{MPI{\tt \char`\_}Type{\tt \char`\_}get{\tt \char`\_}contents}{tex}{5/8/2006}
\manname{MPI{\tt \char`\_}Type{\tt \char`\_}get{\tt \char`\_}contents}
--- get type contents 
\subhead{Synopsis}
\startvb\begin{verbatim}
int MPI_Type_get_contents(MPI_Datatype datatype,
                        int max_integers,
                        int max_addresses,
                        int max_datatypes,
                        int array_of_integers[],
                        MPI_Aint array_of_addresses[],
                        MPI_Datatype array_of_datatypes[])

\end{verbatim}
\endvb

\subhead{Arguments}
\startarg{MPI{\tt \char`\_}Datatype datatype }{datatype
}
\startarg{int max{\tt \char`\_}integers }{max integers
}
\startarg{int max{\tt \char`\_}addresses }{max addresses
}
\startarg{int max{\tt \char`\_}datatypes }{max datatypes
}
\startarg{int array{\tt \char`\_}of{\tt \char`\_}integers[] }{integers
}
\startarg{MPI{\tt \char`\_}Aint array{\tt \char`\_}of{\tt \char`\_}addresses[] }{addresses
}
\startarg{MPI{\tt \char`\_}Datatype array{\tt \char`\_}of{\tt \char`\_}datatypes[] }{datatypes
}
\par
\subhead{Notes}
\par
\subhead{Notes for Fortran}
All MPI routines in Fortran (except for {\tt MPI{\tt \char`\_}WTIME} and {\tt MPI{\tt \char`\_}WTICK}) have
an additional argument {\tt ierr} at the end of the argument list.  {\tt ierr
}is an integer and has the same meaning as the return value of the routine
in C.  In Fortran, MPI routines are subroutines, and are invoked with the
{\tt call} statement.
\par
All MPI objects (e.g., {\tt MPI{\tt \char`\_}Datatype}, {\tt MPI{\tt \char`\_}Comm}) are of type {\tt INTEGER
}in Fortran.
\par
\subhead{Errors}
\par
All MPI routines (except {\tt MPI{\tt \char`\_}Wtime} and {\tt MPI{\tt \char`\_}Wtick}) return an error value;
C routines as the value of the function and Fortran routines in the last
argument.  Before the value is returned, the current MPI error handler is
called.  By default, this error handler aborts the MPI job.  The error handler
may be changed with {\tt MPI{\tt \char`\_}Comm{\tt \char`\_}set{\tt \char`\_}errhandler} (for communicators),
{\tt MPI{\tt \char`\_}File{\tt \char`\_}set{\tt \char`\_}errhandler} (for files), and {\tt MPI{\tt \char`\_}Win{\tt \char`\_}set{\tt \char`\_}errhandler} (for
RMA windows).  The MPI-1 routine {\tt MPI{\tt \char`\_}Errhandler{\tt \char`\_}set} may be used but
its use is deprecated.  The predefined error handler
{\tt MPI{\tt \char`\_}ERRORS{\tt \char`\_}RETURN} may be used to cause error values to be returned.
Note that MPI does {\em not} guarentee that an MPI program can continue past
an error; however, MPI implementations will attempt to continue whenever
possible.
\par
\startarg{MPI{\tt \char`\_}SUCCESS }{No error; MPI routine completed successfully.
}
\location{type{\tt \char`\_}get{\tt \char`\_}contents.c}
\endmanpage
