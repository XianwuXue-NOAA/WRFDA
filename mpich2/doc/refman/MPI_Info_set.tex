\startmanpage
\mantitle{MPI{\tt \char`\_}Info{\tt \char`\_}set}{tex}{5/8/2006}
\manname{MPI{\tt \char`\_}Info{\tt \char`\_}set}
--- Adds a (key,value) pair to info 
\subhead{Synopsis}
\startvb\begin{verbatim}
int MPI_Info_set( MPI_Info info, char *key, char *value )

\end{verbatim}
\endvb

\subhead{Input Parameters}
\startarg{info }{info object (handle)
}
\startarg{key }{key (string)
}
\startarg{value }{value (string)
}
\par
\subhead{Thread and Interrupt Safety}
\par
The user is responsible for ensuring that multiple threads do not try to
update the same MPI object from different threads.  This routine should
not be used from within a signal handler.
\par
The MPI standard defined a thread-safe interface but this does not
mean that all routines may be called without any thread locks.  For
example, two threads must not attempt to change the contents of the
same {\tt MPI{\tt \char`\_}Info} object concurrently.  The user is responsible in this
case for using some mechanism, such as thread locks, to ensure that
only one thread at a time makes use of this routine.
\par
\subhead{Notes for Fortran}
All MPI routines in Fortran (except for {\tt MPI{\tt \char`\_}WTIME} and {\tt MPI{\tt \char`\_}WTICK}) have
an additional argument {\tt ierr} at the end of the argument list.  {\tt ierr
}is an integer and has the same meaning as the return value of the routine
in C.  In Fortran, MPI routines are subroutines, and are invoked with the
{\tt call} statement.
\par
All MPI objects (e.g., {\tt MPI{\tt \char`\_}Datatype}, {\tt MPI{\tt \char`\_}Comm}) are of type {\tt INTEGER
}in Fortran.
\par
\subhead{Errors}
\par
All MPI routines (except {\tt MPI{\tt \char`\_}Wtime} and {\tt MPI{\tt \char`\_}Wtick}) return an error value;
C routines as the value of the function and Fortran routines in the last
argument.  Before the value is returned, the current MPI error handler is
called.  By default, this error handler aborts the MPI job.  The error handler
may be changed with {\tt MPI{\tt \char`\_}Comm{\tt \char`\_}set{\tt \char`\_}errhandler} (for communicators),
{\tt MPI{\tt \char`\_}File{\tt \char`\_}set{\tt \char`\_}errhandler} (for files), and {\tt MPI{\tt \char`\_}Win{\tt \char`\_}set{\tt \char`\_}errhandler} (for
RMA windows).  The MPI-1 routine {\tt MPI{\tt \char`\_}Errhandler{\tt \char`\_}set} may be used but
its use is deprecated.  The predefined error handler
{\tt MPI{\tt \char`\_}ERRORS{\tt \char`\_}RETURN} may be used to cause error values to be returned.
Note that MPI does {\em not} guarentee that an MPI program can continue past
an error; however, MPI implementations will attempt to continue whenever
possible.
\par
\startarg{MPI{\tt \char`\_}SUCCESS }{No error; MPI routine completed successfully.
}
\par
\startarg{MPI{\tt \char`\_}ERR{\tt \char`\_}INFO{\tt \char`\_}KEY }{Invalid or null key string for info.
}
\startarg{MPI{\tt \char`\_}ERR{\tt \char`\_}INFO{\tt \char`\_}VALUE }{Invalid or null value string for info
}
\startarg{MPI{\tt \char`\_}ERR{\tt \char`\_}INTERN }{This error is returned when some part of the MPICH 
implementation is unable to acquire memory.  
}
\location{info{\tt \char`\_}set.c}
\endmanpage
