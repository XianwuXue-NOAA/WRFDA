\startmanpage
\mantitle{MPI{\tt \char`\_}Grequest{\tt \char`\_}start}{tex}{7/11/2006}
\manname{MPI{\tt \char`\_}Grequest{\tt \char`\_}start}
--- Create and return a user-defined request 
\subhead{Synopsis}
\startvb\begin{verbatim}
int MPI_Grequest_start( MPI_Grequest_query_function *query_fn, 
                      MPI_Grequest_free_function *free_fn, 
                      MPI_Grequest_cancel_function *cancel_fn, 
                      void *extra_state, MPI_Request *request )

\end{verbatim}
\endvb

\subhead{Input Parameters}
\startarg{query{\tt \char`\_}fn }{callback function invoked when request status is queried (function)  
}
\startarg{free{\tt \char`\_}fn }{callback function invoked when request is freed (function) 
}
\startarg{cancel{\tt \char`\_}fn }{callback function invoked when request is cancelled (function) 
}
\startarg{extra{\tt \char`\_}state }{Extra state passed to the above functions.
}
\par
\subhead{Output Parameter}
\startarg{request }{Generalized request (handle)
}
\par
\subhead{Notes on the callback functions}
The return values from the callback functions must be a valid MPI error code
or class.  This value may either be the return value from any MPI routine
(with one exception noted below) or any of the MPI error classes.
For portable programs, {\tt MPI{\tt \char`\_}ERR{\tt \char`\_}OTHER} may be used; to provide more
specific information, create a new MPI error class or code with
{\tt MPI{\tt \char`\_}Add{\tt \char`\_}error{\tt \char`\_}class} or {\tt MPI{\tt \char`\_}Add{\tt \char`\_}error{\tt \char`\_}code} and return that value.
\par
The MPI standard is not clear on the return values from the callback routines.
However, there are notes in the standard that imply that these are MPI error
codes.  For example, pages 169 line 46 through page 170, line 1 require that
the {\tt free{\tt \char`\_}fn} return an MPI error code that may be used in the MPI completion
functions when they return {\tt MPI{\tt \char`\_}ERR{\tt \char`\_}IN{\tt \char`\_}STATUS}.
\par
The one special case is the error value returned by {\tt MPI{\tt \char`\_}Comm{\tt \char`\_}dup} when
the attribute callback routine returns a failure.  The MPI standard is not
clear on what values may be used to indicate an error return.  Further,
the Intel MPI test suite made use of non-zero values to indicate failure,
and expected these values to be returned by the {\tt MPI{\tt \char`\_}Comm{\tt \char`\_}dup} when the
attribute routines encountered an error.  Such error values may not be valid
MPI error codes or classes.  Because of this, it is the user{\tt s responsibility
to either use valid MPI error codes in return from the attribute callbacks,
if those error codes are to be returned by a generalized request callback,
or to detect and convert those error codes to valid MPI error codes (recall
that MPI error classes are valid error codes).
\par
\subhead{Thread and Interrupt Safety}
\par
This routine is thread-safe.  This means that this routine may be
safely used by multiple threads without the need for any user-provided
thread locks.  However, the routine is not interrupt safe.  Typically,
this is due to the use of memory allocation routines such as }malloc
{\tt or other non-MPICH runtime routines that are themselves not interrupt-safe.
\par
\subhead{Notes for Fortran}
All MPI routines in Fortran (except for }MPI{\tt \char`\_}WTIME{\tt  and }MPI{\tt \char`\_}WTICK{\tt ) have
an additional argument }ierr{\tt  at the end of the argument list.  }ierr
{\tt is an integer and has the same meaning as the return value of the routine
in C.  In Fortran, MPI routines are subroutines, and are invoked with the
}call{\tt  statement.
\par
All MPI objects (e.g., }MPI{\tt \char`\_}Datatype{\tt , }MPI{\tt \char`\_}Comm{\tt ) are of type }INTEGER
{\tt in Fortran.
\par
\subhead{Errors}
\par
All MPI routines (except }MPI{\tt \char`\_}Wtime{\tt  and }MPI{\tt \char`\_}Wtick{\tt ) return an error value;
C routines as the value of the function and Fortran routines in the last
argument.  Before the value is returned, the current MPI error handler is
called.  By default, this error handler aborts the MPI job.  The error handler
may be changed with }MPI{\tt \char`\_}Comm{\tt \char`\_}set{\tt \char`\_}errhandler{\tt  (for communicators),
}MPI{\tt \char`\_}File{\tt \char`\_}set{\tt \char`\_}errhandler{\tt  (for files), and }MPI{\tt \char`\_}Win{\tt \char`\_}set{\tt \char`\_}errhandler{\tt  (for
RMA windows).  The MPI-1 routine }MPI{\tt \char`\_}Errhandler{\tt \char`\_}set{\tt  may be used but
its use is deprecated.  The predefined error handler
}MPI{\tt \char`\_}ERRORS{\tt \char`\_}RETURN{\tt  may be used to cause error values to be returned.
Note that MPI does }{\em not} guarentee that an MPI program can continue past
an error; however, MPI implementations will attempt to continue whenever
possible.
\par
\startarg{MPI{\tt \char`\_}SUCCESS }{No error; MPI routine completed successfully.
}
\startarg{MPI{\tt \char`\_}ERR{\tt \char`\_}ARG }{Invalid argument.  Some argument is invalid and is not
identified by a specific error class (e.g., MPI{\tt \char`\_}ERR{\tt \char`\_}RANK{\tt ).
}
\location{greq{\tt \char`\_}start.c}
\endmanpage
