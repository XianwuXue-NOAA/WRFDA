\startmanpage
\mantitle{MPI{\tt \char`\_}Irecv}{tex}{5/8/2006}
\manname{MPI{\tt \char`\_}Irecv}
--- Begins a nonblocking receive 
\subhead{Synopsis}
\startvb\begin{verbatim}
int MPI_Irecv(void *buf, int count, MPI_Datatype datatype, int source,
              int tag, MPI_Comm comm, MPI_Request *request)

\end{verbatim}
\endvb

\subhead{Input Parameters}
\startarg{buf }{initial address of receive buffer (choice) 
}
\startarg{count }{number of elements in receive buffer (integer) 
}
\startarg{datatype }{datatype of each receive buffer element (handle) 
}
\startarg{source }{rank of source (integer) 
}
\startarg{tag }{message tag (integer) 
}
\startarg{comm }{communicator (handle) 
}
\par
\subhead{Output Parameter}
\startarg{request }{communication request (handle) 
}
\par
\subhead{Thread and Interrupt Safety}
\par
This routine is thread-safe.  This means that this routine may be
safely used by multiple threads without the need for any user-provided
thread locks.  However, the routine is not interrupt safe.  Typically,
this is due to the use of memory allocation routines such as {\tt malloc
}or other non-MPICH runtime routines that are themselves not interrupt-safe.
\par
\subhead{Notes for Fortran}
All MPI routines in Fortran (except for {\tt MPI{\tt \char`\_}WTIME} and {\tt MPI{\tt \char`\_}WTICK}) have
an additional argument {\tt ierr} at the end of the argument list.  {\tt ierr
}is an integer and has the same meaning as the return value of the routine
in C.  In Fortran, MPI routines are subroutines, and are invoked with the
{\tt call} statement.
\par
All MPI objects (e.g., {\tt MPI{\tt \char`\_}Datatype}, {\tt MPI{\tt \char`\_}Comm}) are of type {\tt INTEGER
}in Fortran.
\par
\subhead{Errors}
\par
All MPI routines (except {\tt MPI{\tt \char`\_}Wtime} and {\tt MPI{\tt \char`\_}Wtick}) return an error value;
C routines as the value of the function and Fortran routines in the last
argument.  Before the value is returned, the current MPI error handler is
called.  By default, this error handler aborts the MPI job.  The error handler
may be changed with {\tt MPI{\tt \char`\_}Comm{\tt \char`\_}set{\tt \char`\_}errhandler} (for communicators),
{\tt MPI{\tt \char`\_}File{\tt \char`\_}set{\tt \char`\_}errhandler} (for files), and {\tt MPI{\tt \char`\_}Win{\tt \char`\_}set{\tt \char`\_}errhandler} (for
RMA windows).  The MPI-1 routine {\tt MPI{\tt \char`\_}Errhandler{\tt \char`\_}set} may be used but
its use is deprecated.  The predefined error handler
{\tt MPI{\tt \char`\_}ERRORS{\tt \char`\_}RETURN} may be used to cause error values to be returned.
Note that MPI does {\em not} guarentee that an MPI program can continue past
an error; however, MPI implementations will attempt to continue whenever
possible.
\par
\startarg{MPI{\tt \char`\_}SUCCESS }{No error; MPI routine completed successfully.
}
\startarg{MPI{\tt \char`\_}ERR{\tt \char`\_}COMM }{Invalid communicator.  A common error is to use a null
communicator in a call (not even allowed in {\tt MPI{\tt \char`\_}Comm{\tt \char`\_}rank}).
}
\startarg{MPI{\tt \char`\_}ERR{\tt \char`\_}COUNT }{Invalid count argument.  Count arguments must be 
non-negative; a count of zero is often valid.
}
\startarg{MPI{\tt \char`\_}ERR{\tt \char`\_}TYPE }{Invalid datatype argument.  May be an uncommitted 
MPI{\tt \char`\_}Datatype (see {\tt MPI{\tt \char`\_}Type{\tt \char`\_}commit}).
}
\startarg{MPI{\tt \char`\_}ERR{\tt \char`\_}TAG }{Invalid tag argument.  Tags must be non-negative; tags
in a receive ({\tt MPI{\tt \char`\_}Recv}, {\tt MPI{\tt \char`\_}Irecv}, {\tt MPI{\tt \char`\_}Sendrecv}, etc.) may
also be {\tt MPI{\tt \char`\_}ANY{\tt \char`\_}TAG}.  The largest tag value is available through the 
the attribute {\tt MPI{\tt \char`\_}TAG{\tt \char`\_}UB}.
}
\startarg{MPI{\tt \char`\_}ERR{\tt \char`\_}RANK }{Invalid source or destination rank.  Ranks must be between
zero and the size of the communicator minus one; ranks in a receive
({\tt MPI{\tt \char`\_}Recv}, {\tt MPI{\tt \char`\_}Irecv}, {\tt MPI{\tt \char`\_}Sendrecv}, etc.) may also be {\tt MPI{\tt \char`\_}ANY{\tt \char`\_}SOURCE}.
}
\startarg{MPI{\tt \char`\_}ERR{\tt \char`\_}INTERN }{This error is returned when some part of the MPICH 
implementation is unable to acquire memory.  
}
\location{irecv.c}
\endmanpage
