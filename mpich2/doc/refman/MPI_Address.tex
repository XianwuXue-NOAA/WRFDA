\startmanpage
\mantitle{MPI{\tt \char`\_}Address}{tex}{12/9/2006}
\manname{MPI{\tt \char`\_}Address}
--- Gets the address of a location in memory   
\subhead{Synopsis}
\startvb\begin{verbatim}
int MPI_Address( void *location, MPI_Aint *address )

\end{verbatim}
\endvb

\subhead{Input Parameter}
\startarg{location }{location in caller memory (choice) 
}
\par
\subhead{Output Parameter}
\startarg{address }{address of location (integer) 
}
\par
\subhead{Note}
This routine is provided for both the Fortran and C programmers.
On many systems, the address returned by this routine will be the same
as produced by the C {\tt \&} operator, but this is not required in C and
may not be true of systems with word- rather than byte-oriented
instructions or systems with segmented address spaces.
\par
\subhead{Thread and Interrupt Safety}
\par
This routine is both thread- and interrupt-safe.
This means that this routine may safely be used by multiple threads and
from within a signal handler.
\par
\subhead{Deprecated Function}
The MPI-2 standard deprecated a number of routines because MPI-2 provides
better versions.  This routine is one of those that was deprecated.  The
routine may continue to be used, but new code should use the replacement
routine.
The replacement for this routine is {\tt MPI{\tt \char`\_}Get{\tt \char`\_}address}.
\par
\subhead{Notes for Fortran}
All MPI routines in Fortran (except for {\tt MPI{\tt \char`\_}WTIME} and {\tt MPI{\tt \char`\_}WTICK}) have
an additional argument {\tt ierr} at the end of the argument list.  {\tt ierr
}is an integer and has the same meaning as the return value of the routine
in C.  In Fortran, MPI routines are subroutines, and are invoked with the
{\tt call} statement.
\par
All MPI objects (e.g., {\tt MPI{\tt \char`\_}Datatype}, {\tt MPI{\tt \char`\_}Comm}) are of type {\tt INTEGER
}in Fortran.
\par
\subhead{Errors}
\par
All MPI routines (except {\tt MPI{\tt \char`\_}Wtime} and {\tt MPI{\tt \char`\_}Wtick}) return an error value;
C routines as the value of the function and Fortran routines in the last
argument.  Before the value is returned, the current MPI error handler is
called.  By default, this error handler aborts the MPI job.  The error handler
may be changed with {\tt MPI{\tt \char`\_}Comm{\tt \char`\_}set{\tt \char`\_}errhandler} (for communicators),
{\tt MPI{\tt \char`\_}File{\tt \char`\_}set{\tt \char`\_}errhandler} (for files), and {\tt MPI{\tt \char`\_}Win{\tt \char`\_}set{\tt \char`\_}errhandler} (for
RMA windows).  The MPI-1 routine {\tt MPI{\tt \char`\_}Errhandler{\tt \char`\_}set} may be used but
its use is deprecated.  The predefined error handler
{\tt MPI{\tt \char`\_}ERRORS{\tt \char`\_}RETURN} may be used to cause error values to be returned.
Note that MPI does {\em not} guarentee that an MPI program can continue past
an error; however, MPI implementations will attempt to continue whenever
possible.
\par
\startarg{MPI{\tt \char`\_}SUCCESS }{No error; MPI routine completed successfully.
}
\startarg{MPI{\tt \char`\_}ERR{\tt \char`\_}OTHER }{Other error; use {\tt MPI{\tt \char`\_}Error{\tt \char`\_}string} to get more information
about this error code. 
}
\location{address.c}
\endmanpage
