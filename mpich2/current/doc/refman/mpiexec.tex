\startmanpage
\mantitle{mpiexec}{tex}{9/21/2006}
\manname{mpiexec}
--- Run an MPI program 
\subhead{Synopsis}
\startvb\begin{verbatim}

\end{verbatim}
\endvb
\begin{verbatim}
    mpiexec args executable pgmargs [ : args executable pgmargs ... ]
\end{verbatim}

where {\tt args} are command line arguments for {\tt mpiexec} (see below),
{\tt executable} is the name of an executable MPI program, and {\tt pgmargs
}are command line arguments for the executable.  Multiple executables
can be specified by using the colon notation (for MPMD - Multiple Program
Multiple Data applications).
\par
\subhead{Standard Parameters}
\par
\par
\startarg{-n num }{number of processes
}
\startarg{-host hostname }{name of a host (machine, system) on which to run
}
\startarg{-arch architecture{\tt \char`\_}name }{name of the computer architecture required
for this executable
}
\startarg{-wdir working{\tt \char`\_}directory }{The working directory to use when running the 
executable (set before the executable starts)
}
\startarg{-path pathlist }{use this to find the executable
}
\startarg{-soft list }{comma separated triplets specifying valid numbers of processes
}
\startarg{-file name }{implementation-defined specification file
}
\startarg{-configfile name }{file containing specifications of host/program, 
one per line, with {\tt \char`\#} as a comment indicator, e.g., the usual
mpiexec input, but with ":" replaced with a newline.  That is,
the configfile contains lines with {\tt -soft}, {\tt -n} etc.
}
\par
\par
\subhead{Environment Variables}
The following environment variables affect the behavior of {\tt mpiexec}:
\startarg{MPIEXEC{\tt \char`\_}UNIVERSE{\tt \char`\_}SIZE }{Sets the maximum number of processes to allow.
}
\startarg{MPIEXEC{\tt \char`\_}TIMEOUT }{Maximum running time in seconds.  {\tt mpiexec} will
terminate MPI programs that take longer than the value specified by
{\tt MPIEXEC{\tt \char`\_}TIMEOUT}.  This version of {\tt mpiexec} (forker process manager)
sets a default timelimit of 3 minutes (180 seconds) since the forker
process manager is intended for debugging applications.
}
\startarg{MPIEXEC{\tt \char`\_}PREFIX{\tt \char`\_}DEFAULT }{If this environment variable is set, output
to standard output is prefixed by the rank in {\tt MPI{\tt \char`\_}COMM{\tt \char`\_}WORLD} of the 
process and output to standard error is prefixed by the rank and the 
text {\tt (err)}; both are followed by an angle bracket ({\tt $>$}).  If
this variable is not set, there is no prefix.
}
\startarg{MPIEXEC{\tt \char`\_}PREFIX{\tt \char`\_}STDOUT }{Set the prefix for output from standard output.
This is a string containing text and (optionally) the expression {\tt \%d};
if {\tt \%d} is seen, it is replaced by the rank in {\tt MPI{\tt \char`\_}COMM{\tt \char`\_}WORLD} of the 
process.  If there is more than one {\tt MPI{\tt \char`\_}COMM{\tt \char`\_}WORLD}; for example, 
if {\tt MPI{\tt \char`\_}Comm{\tt \char`\_}spawn} is used, the expression {\tt \%w} is replaced with a
numeric {\em world number}, assigned by {\tt mpiexec}.  The expression {\tt \%\%} 
produces a single percent character.
}
\startarg{MPIEXEC{\tt \char`\_}PREFIX{\tt \char`\_}STDERR }{Like {\tt MPIEXEC{\tt \char`\_}PREFIX{\tt \char`\_}STDOUT}, but for output from
standard error.
}
\startarg{MPIEXEC{\tt \char`\_}STDIN{\tt \char`\_}DEST }{Sets the destination process for input from standard in,
by rank in the corresponding {\tt MPI{\tt \char`\_}COMM{\tt \char`\_}WORLD}.
}
\startarg{MPIEXEC{\tt \char`\_}STDOUTBUF }{Sets the buffering mode for standard output.  Valid
values are {\tt NONE} (no buffering), {\tt LINE} (buffering by lines), and
{\tt BLOCK} (buffering by blocks of characters; the size of the block is
implementation defined).  The default is {\tt NONE}.
}
\startarg{MPIEXEC{\tt \char`\_}STDERRBUF }{Like {\tt MPIEXEC{\tt \char`\_}STDOUTBUF}, but for standard error.
}
\par
Note that the {\tt MPIEXEC{\tt \char`\_}PREFIX{\tt \char`\_}xxx} and {\tt MPIEXEC{\tt \char`\_}STDIN{\tt \char`\_}DEST} variables
are not fully implemented.
\par
\subhead{Startup Environment}
All of the users environment variables are provided to each of the
created processes.  This is not required by the MPI standard, but
is often very convenient.
\par
\subhead{Return Status}
{\tt mpiexec} returns the maximum of the exit status values of all of the
processes created by {\tt mpiexec}.
\par
\subhead{Notes}
A few additional enviroment variables are defined for the use of
developers and in debugging.  These are
\par
\startarg{MPIEXEC{\tt \char`\_}DEBUG }{If set, {\tt mpiexec} will write additional information
about the steps that it is taking.
}
\par
\par
\location{mpiexec.txt}
\endmanpage
