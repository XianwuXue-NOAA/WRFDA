\startmanpage
\mantitle{MPI{\tt \char`\_}Allgatherv}{tex}{10/24/2006}
\manname{MPI{\tt \char`\_}Allgatherv}
--- Gathers data from all tasks and deliver the combined data to all tasks 
\subhead{Synopsis}
\startvb\begin{verbatim}
int MPI_Allgatherv(void *sendbuf, int sendcount, MPI_Datatype sendtype, 
                   void *recvbuf, int *recvcounts, int *displs, 
                   MPI_Datatype recvtype, MPI_Comm comm)

\end{verbatim}
\endvb

\subhead{Input Parameters}
\startarg{sendbuf }{starting address of send buffer (choice) 
}
\startarg{sendcount }{number of elements in send buffer (integer) 
}
\startarg{sendtype }{data type of send buffer elements (handle) 
}
\startarg{recvcounts }{integer array (of length group size) 
containing the number of elements that are received from each process 
}
\startarg{displs }{integer array (of length group size). Entry 
{\tt i}  specifies the displacement (relative to recvbuf ) at
which to place the incoming data from process  {\tt i}  
}
\startarg{recvtype }{data type of receive buffer elements (handle) 
}
\startarg{comm }{communicator (handle) 
}
\par
\subhead{Output Parameter}
\startarg{recvbuf }{address of receive buffer (choice) 
}
\par
\subhead{Notes}
The MPI standard (1.0 and 1.1) says that
\nextline
\par
\nextline
\par
The jth block of data sent from
each proess is received by every process and placed in the jth block of the
buffer {\tt recvbuf}.
\nextline
\par
\nextline
\par
This is misleading; a better description is
\nextline
\par
\nextline
\par
The block of data sent from the jth process is received by every
process and placed in the jth block of the buffer {\tt recvbuf}.
\nextline
\par
\nextline
\par
This text was suggested by Rajeev Thakur, and has been adopted as a
clarification to the MPI standard by the MPI-Forum.
\par
\subhead{Thread and Interrupt Safety}
\par
This routine is thread-safe.  This means that this routine may be
safely used by multiple threads without the need for any user-provided
thread locks.  However, the routine is not interrupt safe.  Typically,
this is due to the use of memory allocation routines such as {\tt malloc
}or other non-MPICH runtime routines that are themselves not interrupt-safe.
\par
\subhead{Notes for Fortran}
All MPI routines in Fortran (except for {\tt MPI{\tt \char`\_}WTIME} and {\tt MPI{\tt \char`\_}WTICK}) have
an additional argument {\tt ierr} at the end of the argument list.  {\tt ierr
}is an integer and has the same meaning as the return value of the routine
in C.  In Fortran, MPI routines are subroutines, and are invoked with the
{\tt call} statement.
\par
All MPI objects (e.g., {\tt MPI{\tt \char`\_}Datatype}, {\tt MPI{\tt \char`\_}Comm}) are of type {\tt INTEGER
}in Fortran.
\par
\subhead{Errors}
\par
All MPI routines (except {\tt MPI{\tt \char`\_}Wtime} and {\tt MPI{\tt \char`\_}Wtick}) return an error value;
C routines as the value of the function and Fortran routines in the last
argument.  Before the value is returned, the current MPI error handler is
called.  By default, this error handler aborts the MPI job.  The error handler
may be changed with {\tt MPI{\tt \char`\_}Comm{\tt \char`\_}set{\tt \char`\_}errhandler} (for communicators),
{\tt MPI{\tt \char`\_}File{\tt \char`\_}set{\tt \char`\_}errhandler} (for files), and {\tt MPI{\tt \char`\_}Win{\tt \char`\_}set{\tt \char`\_}errhandler} (for
RMA windows).  The MPI-1 routine {\tt MPI{\tt \char`\_}Errhandler{\tt \char`\_}set} may be used but
its use is deprecated.  The predefined error handler
{\tt MPI{\tt \char`\_}ERRORS{\tt \char`\_}RETURN} may be used to cause error values to be returned.
Note that MPI does {\em not} guarentee that an MPI program can continue past
an error; however, MPI implementations will attempt to continue whenever
possible.
\par
\startarg{MPI{\tt \char`\_}ERR{\tt \char`\_}BUFFER }{Invalid buffer pointer.  Usually a null buffer where
one is not valid.
}
\startarg{MPI{\tt \char`\_}ERR{\tt \char`\_}COUNT }{Invalid count argument.  Count arguments must be 
non-negative; a count of zero is often valid.
}
\startarg{MPI{\tt \char`\_}ERR{\tt \char`\_}TYPE }{Invalid datatype argument.  May be an uncommitted 
MPI{\tt \char`\_}Datatype (see {\tt MPI{\tt \char`\_}Type{\tt \char`\_}commit}).
}
\location{allgatherv.c}
\endmanpage
