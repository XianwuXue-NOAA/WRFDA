\startmanpage
\mantitle{MPI{\tt \char`\_}Init{\tt \char`\_}thread}{tex}{12/9/2006}
\manname{MPI{\tt \char`\_}Init{\tt \char`\_}thread}
--- Initialize the MPI execution environment 
\subhead{Synopsis}
\startvb\begin{verbatim}
int MPI_Init_thread( int *argc, char ***argv, int required, int *provided )

\end{verbatim}
\endvb

\subhead{Input Parameters}
\startarg{argc }{Pointer to the number of arguments 
}
\startarg{argv }{Pointer to the argument vector
}
\startarg{required }{Level of desired thread support
}
\par
\subhead{Output Parameter}
\startarg{provided }{Level of provided thread support
}
\par
\subhead{Command line arguments}
MPI specifies no command-line arguments but does allow an MPI
implementation to make use of them.  See {\tt MPI{\tt \char`\_}INIT} for a description of
the command line arguments supported by {\tt MPI{\tt \char`\_}INIT} and {\tt MPI{\tt \char`\_}INIT{\tt \char`\_}THREAD}.
\par
\subhead{Notes}
The valid values for the level of thread support are:
\startarg{MPI{\tt \char`\_}THREAD{\tt \char`\_}SINGLE }{Only one thread will execute. 
}
\startarg{MPI{\tt \char`\_}THREAD{\tt \char`\_}FUNNELED }{The process may be multi-threaded, but only the main 
thread will make MPI calls (all MPI calls are funneled to the 
main thread). 
}
\startarg{MPI{\tt \char`\_}THREAD{\tt \char`\_}SERIALIZED }{The process may be multi-threaded, and multiple 
threads may make MPI calls, but only one at a time: MPI calls are not 
made concurrently from two distinct threads (all MPI calls are serialized). 
}
\startarg{MPI{\tt \char`\_}THREAD{\tt \char`\_}MULTIPLE }{Multiple threads may call MPI, with no restrictions. 
}
\par
\subhead{Notes for Fortran}
Note that the Fortran binding for this routine does not have the {\tt argc} and
{\tt argv} arguments. ({\tt MPI{\tt \char`\_}INIT{\tt \char`\_}THREAD(required, provided, ierror)})
\par
\par
\subhead{Errors}
\par
All MPI routines (except {\tt MPI{\tt \char`\_}Wtime} and {\tt MPI{\tt \char`\_}Wtick}) return an error value;
C routines as the value of the function and Fortran routines in the last
argument.  Before the value is returned, the current MPI error handler is
called.  By default, this error handler aborts the MPI job.  The error handler
may be changed with {\tt MPI{\tt \char`\_}Comm{\tt \char`\_}set{\tt \char`\_}errhandler} (for communicators),
{\tt MPI{\tt \char`\_}File{\tt \char`\_}set{\tt \char`\_}errhandler} (for files), and {\tt MPI{\tt \char`\_}Win{\tt \char`\_}set{\tt \char`\_}errhandler} (for
RMA windows).  The MPI-1 routine {\tt MPI{\tt \char`\_}Errhandler{\tt \char`\_}set} may be used but
its use is deprecated.  The predefined error handler
{\tt MPI{\tt \char`\_}ERRORS{\tt \char`\_}RETURN} may be used to cause error values to be returned.
Note that MPI does {\em not} guarentee that an MPI program can continue past
an error; however, MPI implementations will attempt to continue whenever
possible.
\par
\startarg{MPI{\tt \char`\_}SUCCESS }{No error; MPI routine completed successfully.
}
\startarg{MPI{\tt \char`\_}ERR{\tt \char`\_}OTHER }{Other error; use {\tt MPI{\tt \char`\_}Error{\tt \char`\_}string} to get more information
about this error code. 
}
\par
\subhead{See Also}
 MPI{\tt \char`\_}Init, MPI{\tt \char`\_}Finalize
\nextline
\location{initthread.c}
\endmanpage
