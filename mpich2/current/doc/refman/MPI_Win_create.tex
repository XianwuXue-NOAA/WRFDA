\startmanpage
\mantitle{MPI{\tt \char`\_}Win{\tt \char`\_}create}{tex}{8/9/2006}
\manname{MPI{\tt \char`\_}Win{\tt \char`\_}create}
--- Create an MPI Window object for one-sided communication 
\subhead{Synopsis}
\startvb\begin{verbatim}
int MPI_Win_create(void *base, MPI_Aint size, int disp_unit, MPI_Info info, 
                  MPI_Comm comm, MPI_Win *win)

\end{verbatim}
\endvb

\subhead{Input Parameters}
\startarg{base }{initial address of window (choice) 
}
\startarg{size }{size of window in bytes (nonnegative integer) 
}
\startarg{disp{\tt \char`\_}unit }{local unit size for displacements, in bytes (positive integer) 
}
\startarg{info }{info argument (handle) 
}
\startarg{comm }{communicator (handle) 
}
\par
\subhead{Output Parameter}
\startarg{win }{window object returned by the call (handle) 
}
\par
\subhead{Thread and Interrupt Safety}
\par
This routine is thread-safe.  This means that this routine may be
safely used by multiple threads without the need for any user-provided
thread locks.  However, the routine is not interrupt safe.  Typically,
this is due to the use of memory allocation routines such as {\tt malloc
}or other non-MPICH runtime routines that are themselves not interrupt-safe.
\subhead{Notes for Fortran}
All MPI routines in Fortran (except for {\tt MPI{\tt \char`\_}WTIME} and {\tt MPI{\tt \char`\_}WTICK}) have
an additional argument {\tt ierr} at the end of the argument list.  {\tt ierr
}is an integer and has the same meaning as the return value of the routine
in C.  In Fortran, MPI routines are subroutines, and are invoked with the
{\tt call} statement.
\par
All MPI objects (e.g., {\tt MPI{\tt \char`\_}Datatype}, {\tt MPI{\tt \char`\_}Comm}) are of type {\tt INTEGER
}in Fortran.
\par
\subhead{Errors}
\par
All MPI routines (except {\tt MPI{\tt \char`\_}Wtime} and {\tt MPI{\tt \char`\_}Wtick}) return an error value;
C routines as the value of the function and Fortran routines in the last
argument.  Before the value is returned, the current MPI error handler is
called.  By default, this error handler aborts the MPI job.  The error handler
may be changed with {\tt MPI{\tt \char`\_}Comm{\tt \char`\_}set{\tt \char`\_}errhandler} (for communicators),
{\tt MPI{\tt \char`\_}File{\tt \char`\_}set{\tt \char`\_}errhandler} (for files), and {\tt MPI{\tt \char`\_}Win{\tt \char`\_}set{\tt \char`\_}errhandler} (for
RMA windows).  The MPI-1 routine {\tt MPI{\tt \char`\_}Errhandler{\tt \char`\_}set} may be used but
its use is deprecated.  The predefined error handler
{\tt MPI{\tt \char`\_}ERRORS{\tt \char`\_}RETURN} may be used to cause error values to be returned.
Note that MPI does {\em not} guarentee that an MPI program can continue past
an error; however, MPI implementations will attempt to continue whenever
possible.
\par
\startarg{MPI{\tt \char`\_}SUCCESS }{No error; MPI routine completed successfully.
}
\startarg{MPI{\tt \char`\_}ERR{\tt \char`\_}COMM }{Invalid communicator.  A common error is to use a null
communicator in a call (not even allowed in {\tt MPI{\tt \char`\_}Comm{\tt \char`\_}rank}).
}
\startarg{MPI{\tt \char`\_}ERR{\tt \char`\_}INFO }{Invalid Info 
}
\startarg{MPI{\tt \char`\_}ERR{\tt \char`\_}OTHER }{Other error; use {\tt MPI{\tt \char`\_}Error{\tt \char`\_}string} to get more information
about this error code. 
}
\location{win{\tt \char`\_}create.c}
\endmanpage
