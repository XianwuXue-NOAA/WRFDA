\documentclass[dvipdfm,11pt]{article}
\usepackage[dvipdfm]{hyperref} % Upgraded url package
\parskip=.1in

% Formatting conventions for contributors
% 
% A quoting mechanism is needed to set off things like file names, command
% names, code fragments, and other strings that would confuse the flow of
% text if left undistinguished from preceding and following text.  In this
% document we use the LaTeX macro '\texttt' to indicate such text in the
% source, which normally produces, when used as in '\texttt{special text}',
% the typewriter font.

% It is particularly easy to use this convention if one is using emacs as
% the editor and LaTeX mode within emacs for editing LaTeX documents.  In
% such a case the key sequence ^C^F^T (hold down the control key and type
% 'cft') produces '\texttt{}' with the cursor positioned between the
% braces, ready for the special text to be typed.  The closing brace can
% be skipped over by typing ^e (go to the end of the line) if entering
% text or ^C-} to just move the cursor past the brace.
%
% Please add index entries for important terms and keywords, including
% environment variables that may control the behavior of MPI or one of the
% tools and concepts such as line labeling from mpiexec.

% LaTeX mode is usually loaded automatically.  At Argonne, one way to 
% get several useful emacs tools working for you automatically is to put
% the following in your .emacs file.

% (require 'tex-site)
% (setq LaTeX-mode-hook '(lambda ()
%                        (auto-fill-mode 1)
%                        (flyspell-mode 1)
%                        (reftex-mode 1)
%                        (setq TeX-command "latex")))
   

\begin{document}
\markright{MPICH2 User's Guide}
\title{\textbf{MPICH2 User's Guide}\thanks{This work was supported by the Mathematical,
    Information, and Computational Sciences Division subprogram of the
    Office of Advanced Scientific Computing Research, SciDAC Program,
    Office of Science, U.S. Department of Energy, under Contract
    DE-AC02-06CH11357.}\\
Version 1.0.5p4\\
Mathematics and Computer Science Division\\
Argonne National Laboratory}

\author{William Gropp\\
Ewing Lusk\\
David Ashton\\
Pavan Balaji\\
Darius Buntinas\\
Ralph Butler\\
Anthony Chan\\
Jayesh Krishna\\
Guillaume Mercier\\
Rob Ross\\
Rajeev Thakur\\
Brian Toonen}

\maketitle

\cleardoublepage

\pagenumbering{roman}
\tableofcontents
\clearpage

\pagenumbering{arabic}
\pagestyle{headings}


%% Here is a basic outline for the document
%% 1. Setting paths
%% 2. Compiling and linking
%%   2a. Chosing compilers (e.g., you need not use the same compiler that
%%       MPICH was built with)
%%   2b. Shared libraries
%%   2c. Special issues for Fortran 77 and Fortran 90
%%       (mostly the choice module, but also the various name mangling issues)
%% 3. Running with mpiexec
%%   3a. mpiexec standard options (from MPI-2)
%%   3b. device and pm specific options
%%   3c. environment variables
%%       for example, \texttt{MPIEXEC\_TIMEOUT}
%%   3d. managing stdin/out/err
%%   3e. managing files (staging?), including executables
%% 4. Examples 
%%    4a. Simple programs
%%    4b. Benchmarking (similar or identical to text in the installation guide)
%%    4c. Pointers to other resources (books, tutorials, sample programs)
%% 5. Debugging
%%    5a. Working with single-process debuggers
%%    5b. Working with parallel debuggers such as Totalview
%% 6. Troubleshooting


\section{Introduction}
\label{sec:introduction}

This manual assumes that MPICH2 has already been installed.  For
instructions on how to install MPICH2, see the \emph{MPICH2 Installer's Guide},
or the \texttt{README} in the top-level MPICH2 directory.  This manual
explains how to compile, link, and run MPI applications, and use certain
tools that come with MPICH2.  This is a preliminary version and some
sections are not complete yet.  However, there should be enough here to
get you started with MPICH2.


\section{Migrating to MPICH2 from MPICH1}
\label{sec:migrating}

If you have been using MPICH 1.2.x (1.2.7p1 is the latest version), you
will find a number of things about MPICH2 that are different (and
hopefully better in every case.)  Your MPI application programs need not
change, of course, but a number of things about how you run them will be
different.

MPICH2 is an all-new implementation of the MPI Standard, designed to
implement all of the MPI-2 additions to MPI (dynamic process management,
one-sided operations, parallel I/O, and other extensions) and to apply
the lessons learned in implementing MPICH1 to make MPICH2 more robust,
efficient, and convenient to use.  The \emph{MPICH2 Installer's Guide}
provides some information on changes between MPICH1 and MPICH2 to the process
of configuring and installing MPICH.  Changes to compiling, linking, and
running MPI programs between MPICH1 and MPICH2 are described below.


\subsection{Default Runtime Environment}
\label{sec:default-environment}

In MPICH1, the default configuration used the now-old \texttt{p4}
portable programming environment.  Processes were started via remote
shell commands (\texttt{rsh} or \texttt{ssh}) and the information
necessary for processes to find and connect with one another over
sockets was collected and then distributed at startup time in a
non-scalable fashion.  Furthermore, the entanglement of process
managment functionality with the communication mechanism led to
confusing behavior of the system when things went wrong.

MPICH2 provides a separation of process management and communication.
The default runtime environment consists of a set of daemons, called
mpd's, that establish communication among the machines to be used before
application process startup, thus providing a clearer picture of what is
wrong when communication cannot be established and providing a fast and
scalable startup mechanism when parallel jobs are
started. Section~\ref{sec:managing-mpd} describes the MPD process
management system in more detail.  Other process managers are also available.

\subsection{Starting Parallel Jobs}
\label{sec:startup}

MPICH1 provided the \texttt{mpirun} command to start MPICH1 jobs.  The
MPI-2 Forum recommended a standard, portable command, called
\texttt{mpiexec}, for this purpose.  MPICH2 implements \texttt{mpiexec}
and all of its standard arguments, together with some extensions.  See
Section~\ref{sec:mpiexec-standard} for standard arguments to
\texttt{mpiexec} and various subsections of Section~\ref{sec:mpiexec}
for extensions particular to various process management systems.

MPICH2 also provides an \texttt{mpirun} command for simple backward
compatibility, but MPICH2's \texttt{mpirun} does not provide all the
options of \texttt{mpiexec} or all of the options of MPICH1's
\texttt{mpirun}. 


\subsection{Command-Line Arguments in Fortran}
\label{sec:fortran-command-line}

MPICH1 (more precisely MPICH1's \texttt{mpirun}) required access to
command line arguments in all application programs, including Fortran
ones, and MPICH1's \texttt{configure} devoted some effort to finding the
libraries that contained the right versions of \texttt{iargc} and
\texttt{getarg} and including those libraries with which the
\texttt{mpif77} script linked MPI programs.
Since MPICH2 does not require access to command line
arguments to applications, these functions are optional, and
\texttt{configure} does nothing special with them.  If you need them in
your applications, you will have to ensure that they are available in
the Fortran environment you are using.

\section{Quick Start}
\label{sec:quickstart}

To use MPICH2, you will have to know the directory where MPICH2 has been
installed. 
(Either you installed it there yourself, or your systems administrator has
installed it.  One place to look in this case might be \texttt{/usr/local}.
If MPICH2 has not yet been installed, see the \emph{MPICH2 Installer's Guide}.)
We suggest that you put the \texttt{bin} subdirectory of that directory into
your path.  This will give you access to assorted MPICH2 commands to
compile, link, and run your programs conveniently.  Other commands in
this directory manage parts of the run-time environment and execute
tools.  

One of the first commands you might run is \texttt{mpich2version} to
find out the exact version and configuration of MPICH2 you are working
with.  Some of the material in this manual depends on just what version
of MPICH2 you are using and how it was configured at installation time.

You should now be able to run an MPI program.  Let us assume that the
directory where MPICH2 has been installed is
\texttt{/home/you/mpich2-installed}, and that you have added that directory to
your path, using 
\begin{verbatim}
    setenv PATH /home/you/mpich2-installed/bin:$PATH
\end{verbatim}
for \texttt{tcsh} and \texttt{csh}, or 
\begin{verbatim}
    export PATH=/home/you/mpich2-installed/bin:$PATH
\end{verbatim}
for \texttt{bash} or \texttt{sh}.
Then to run an MPI program, albeit only on one machine, you can do:
\begin{verbatim}
    mpd &
    cd  /home/you/mpich2-installed/examples
    mpiexec -n 3 cpi
    mpdallexit
\end{verbatim}
Details for these commands are provided below, but if you can
successfully execute them here, then you have a correctly installed
MPICH2 and have run an MPI program. 

\section{Compiling and Linking}
\label{sec:compiling}

A convenient way to compile and link your program is by using scripts
that use the same compiler that MPICH2 was built with.  These are
\texttt{mpicc}, \texttt{mpicxx}, \texttt{mpif77}, and \texttt{mpif90},
for C, C++, Fortran 77, and Fortran 90 programs, respectively.  If any
of these commands are missing, it means that MPICH2 was configured
without support for that particular language.

\subsection{Specifying Compilers}
\label{sec:specifying-compilers}

You need not use the same compiler that MPICH2 was built with, but not
all compilers are compatible.  You can also specify the compiler for
building MPICH2 itself, as reported by \texttt{mpich2version}, just by
using the compiling and linking commands from the previous section.
%(See the \emph{Installer's Guide}).
The environment variables \texttt{MPICH\_CC}, \texttt{MPICH\_CXX},
\texttt{MPICH\_F77}, and \texttt{MPICH\_F90} may be used to specify
alternate C, C++, Fortran 77, and Fortran 90 compilers, respectively.

\subsection{Shared Libraries}
\label{sec:shared-libraries}

Currently shared libraries are only tested on Linux and Mac OS X, and there are
restrictions.  See the \emph{Installer's Guide} for how to build MPICH2
as a shared library.  If shared libraries have been built, you will
get them automatically when you link your program with any of the
MPICH2 compilation scripts.

\subsection{Special Issues for C++}
\label{sec:cxx}

Some users may get error messages such as
\begin{small}
\begin{verbatim}
    SEEK_SET is #defined but must not be for the C++ binding of MPI
\end{verbatim}
\end{small}
The problem is that both \texttt{stdio.h} and the MPI C++ interface use
\texttt{SEEK\_SET}, \texttt{SEEK\_CUR}, and \texttt{SEEK\_END}.  This is really a bug
in the MPI-2 standard.  You can try adding 
\begin{verbatim}
    #undef SEEK_SET
    #undef SEEK_END
    #undef SEEK_CUR
\end{verbatim}
before \texttt{mpi.h} is included, or add the definition
\begin{verbatim}
    -DMPICH_IGNORE_CXX_SEEK
\end{verbatim}
to the command line (this will cause the MPI versions of \texttt{SEEK\_SET}
etc. to be skipped).

\subsection{Special Issues for Fortran}
\label{sec:fortran}

MPICH2 provides two kinds of support for Fortran programs.  For
Fortran 77 programmers, the file \texttt{mpif.h} provides the
definitions of the MPI constants such as \texttt{MPI\_COMM\_WORLD}.
Fortran 90 programmers should use the \texttt{MPI} module instead;
this provides all of the definitions as well as interface definitions
for many of the MPI functions.  However, this MPI module does not
provide full Fortran 90 support; in particular, interfaces for the
routines, such as \texttt{MPI\_Send}, that take ``choice'' arguments
are not provided.

%% \begin{itemize}
%% \item Review of basic and extended support; the Fortran 90 module.
%% \item Flushing output.
%% G95 requires 
%% \begin{verbatim}
%% setenv G95_UNBUFFERED_ALL TRUE
%% \end{verbatim}
%% or at least
%% \begin{verbatim}
%% setenv G95_UNBUFFERED_6 TRUE
%% \end{verbatim}
%% to have output to \texttt{stdout} appear when it is printed.  For
%% example, in the example program \texttt{examples/f90/pi3f90}, without
%% this setting, the prompt to enter the number of intervals will not
%% appear until the program exits.

%% \item Various name-mangling issues
%% \end{itemize}


\section{Running Programs with \texttt{mpiexec}}
\label{sec:mpiexec}

If you have been using the original MPICH, or any of a number of other MPI
implementations, then you have probably been using \texttt{mpirun} as a
way to start your MPI programs.
The MPI-2 Standard describes \texttt{mpiexec} as a suggested way to run
MPI programs.  MPICH2 implements the \texttt{mpiexec} standard, and also
provides some extensions.  MPICH2 provides \texttt{mpirun} for backward
compatibility with existing scripts, but it does not support the same or
as many options as \texttt{mpiexec} or all of the options of MPICH1's
\texttt{mpirun}.   

\subsection{Standard \texttt{mpiexec}}
\label{sec:mpiexec-standard}

Here we describe the standard \texttt{mpiexec} arguments from the MPI-2
Standard~\cite{mpi-forum:mpi2-journal}.  The simplest form of a command
to start an MPI job is 
\begin{verbatim}
   mpiexec -n 32 a.out
\end{verbatim}
to start the executable \texttt{a.out} with 32 processes (providing an
\texttt{MPI\_COMM\_WORLD} of size 32 inside the MPI application).  Other
options are supported, for specifying hosts to run on,  search paths for
executables, working directories, and even a more general way of
specifying a number of processes.  Multiple sets of processes can be run
with different exectuables and different values for their arguments,
with ``\texttt{:}'' separating the sets of processes, as in:
\begin{verbatim}
   mpiexec -n 1 -host loginnode master : -n 32 -host smp slave
\end{verbatim}
The \texttt{-configfile} argument allows one to specify a file containing the
specifications for process sets on separate lines in the file.  This
makes it unnecessary to have long command lines for \texttt{mpiexec}.  
(See pg.~353 of~\cite{Snir:1998:MPI2Book}.)

It is also possible to start a one process MPI job (with a
\texttt{MPI\_COMM\_WORLD} whose size is equal to 1), without using
\texttt{mpiexec}. 
This process will become an MPI process when it calls \texttt{MPI\_Init}, and
it may then call other MPI functions.  
Currently, MPICH2 does not fully support calling the dynamic process routines
from MPI-2 (e.g., \texttt{MPI\_Comm\_spawn} or \texttt{MPI\_Comm\_accept})
from processes that are not started with \texttt{mpiexec}.  
% Note that MPD doesn't support comm_accept/connect; and the root version of
% MPD doesn't support spawn (since MPD uses a single KVS name for all 
% singleton inits).  Fixing this requires changes to PMI as well as MPD;
% other process managers will also need to handle this case.

\subsection{Extensions for All Process Management Environments}
\label{sec:extensions-uniform}

Some \texttt{mpiexec} arguments are specific to particular communication
subsystems (``devices'') or process management environments (``process
managers'').  Our intention is to make all arguments as uniform as
possible across devices and process managers.  For the time being we
will document these separately.

\subsection{Extensions for the MPD Process Management Environment}
\label{sec:extensions-various}

MPICH2 provides a number of process management systems.  The default is
called MPD.  MPD provides a number of extensions to the standard form of
\texttt{mpiexec}.

\subsubsection{Basic \texttt{mpiexec} arguments for MPD}
\label{sec:extensions-mpd}

The default configuration of MPICH2 chooses the MPD process manager and
the ``simple'' implementation of the Process Management Interface.
MPD provides a version of
\texttt{mpiexec} that supports both the standard arguments described in
Section~\ref{sec:mpiexec-standard} and other arguments described in this
section.  MPD also provides a number of commands for querying the MPD
process management environment and interacting with jobs it has started. 

Before running \texttt{mpiexec}, the runtime environment must be
established.  In the case of MPD, the daemons must be running.  See
Section~\ref{sec:managing-mpd} for how to run and manage the MPD daemons.

We assume that the MPD ring is up and the installation's \texttt{bin}
directory is in your path; that is, you can do:
\begin{verbatim}
    mpdtrace
\end{verbatim}
and it will output a list of nodes on which you can run MPI programs.
Now you are ready to run a program with \texttt{mpiexec}.  Let us assume
that you have compiled and linked the program \texttt{cpi} (in the
\texttt{installdir/examples} directory and that this directory is in your
\texttt{PATH}.  Or that is your current working directory and
`\texttt{.}' (``dot'') is in your PATH.   The simplest thing to do is
\begin{verbatim}
    mpiexec -n 5 cpi
\end{verbatim}
to run \texttt{cpi} on five nodes.  The process management system (such
as MPD) will choose machines to run them on, and \texttt{cpi} will tell
you where each is running.  

You can use \texttt{mpiexec} to run non-MPI programs as well.  This is
sometimes useful in making sure all the machines are up and ready for
use.  Useful examples include
\begin{verbatim}
    mpiexec -n 10 hostname
\end{verbatim}
and
\begin{verbatim}
    mpiexec -n 10 printenv
\end{verbatim}


\subsubsection{Other Command-Line Arguments to \texttt{mpiexec} for MPD}
\label{sec:environment}

The MPI-2 standard specifies the syntax and semantics of the arguments
\texttt{-n}, \texttt{-path},\texttt{-wdir}, \texttt{-host},
\texttt{-file}, \texttt{-configfile}, and \texttt{-soft}.  All of these
are currently implemented for MPD's \texttt{mpiexec}.
Each of these is what we call a ``local'' option, since
its scope is the processes in the set of processes described between
colons, or on separate lines of the file specified by
\texttt{-configfile}.  We add some extensions that are local in this way
and some that are ``global'' in the sense that they apply to all the
processes being started by the invocation of \texttt{mpiexec}.

The MPI-2 Standard provides a way to pass different arguments to different
application processes, but does not provide a way to pass environment
variables.  MPICH2 provides an extension that supports environment
variables.
The local parameter \texttt{-env} does this for one set of
processes.  That is,
\begin{verbatim}
   mpiexec -n 1 -env FOO BAR a.out : -n 2 -env BAZZ FAZZ b.out
\end{verbatim}
makes \texttt{BAR} the value of environment variable \texttt{FOO} on the
first process, running the executable \texttt{a.out}, and gives the
environment variable \texttt{BAZZ} the value \texttt{FAZZ} on the second
two processes, running the executable \texttt{b.out}.  To set an
environment variable without giving it a value, use \texttt{''} as the
value in the above command line.

The global parameter \texttt{-genv} can be used to pass the same
environment variables to all processes.  That is,
\begin{verbatim}
    mpiexec -genv FOO BAR -n 2 a.out : -n 4 b.out
\end{verbatim}
makes \texttt{BAR} the value of the environment variable \texttt{FOO} on
all six processes.  If \texttt{-genv} appears, it must appear in the
first group.  If both \texttt{-genv} and \texttt{-env} are used, the
\texttt{-env}'s add to the environment specified or added to by the
\texttt{-genv} variables.  If there is only one set of processes (no
``\texttt{:}''), the \texttt{-genv} and \texttt{-env} are equivalent.

The local parameter \texttt{-envall} is an abbreviation for passing the
entire environment in which \texttt{mpiexec} is executed.  The global
version of it is \texttt{-genvall}.  This global version is implicitly
present.  To pass no environment variables, use \texttt{-envnone} and
\texttt{-genvnone}.  So, for example, to set \emph{only} the environment
variable \texttt{FOO} and no others, regardless of the current
environment, you would use 
\begin{verbatim}
    mpiexec -genvnone -env FOO BAR -n 50 a.out
\end{verbatim}
In the case of MPD, we currently make an exception for the \texttt{PATH}
environment variable, which is always passed through.  This exception
was added to make it unnecessary to explicitly pass this variable in the
default case.

A list of environment variable names whose values are
to be copied from the current environment can be given with the
\texttt{-envlist} (respectively, \texttt{-genvlist}) parameter; for example,
\begin{verbatim}
    mpiexec -genvnone -envlist HOME,LD_LIBRARY_PATH -n 50 a.out
\end{verbatim}
sets the \texttt{HOME} and \texttt{LD\_LIBRARY\_PATH} in the environment
of the \texttt{a.out} processes to their values in the environment where
\texttt{mpiexec} is being run.  In this situation you can't have commas
in the environment variable names, although of course they are permitted
in values.

Some extension parameters have only global versions.  They are
\begin{description}
\item[\texttt{-l}] provides rank labels for lines of \texttt{stdout} and
  \texttt{stderr}.  These are a bit obscure for processes that have
  been explicitly spawned, but are still useful.
\item[\texttt{-usize}] sets the ``universe size'' that is retrieved by the MPI
  attribute \\
\texttt{MPI\_UNIVERSE\_SIZE} on \texttt{MPI\_COMM\_WORLD}. 
%% \item[\texttt{-gdb}] invokes \texttt{gdb} on the processes as they are
%%   started.
\item[\texttt{-bnr}] is used when one wants to run executables that have
  been compiled and linked using the \texttt{ch\_p4mpd} or
  \texttt{myrinet} device in MPICH1.  The MPD process manager provides
  backward compatibility in this case.
\item[\texttt{-machinefile}] can be used to specify information
about each of a set of machines.  This information may include the
number of processes to run on each host when executing user programs.
For example, assume that a machinefile named \texttt{mf} contains:
\begin{verbatim}
    # comment line
    hosta
    hostb:2
    hostc    ifhn=hostc-gige
    hostd:4  ifhn=hostd-gige
\end{verbatim}
In addition to specifying hosts and number of processes to run on each,
this machinefile indicates that processes running on hostc and hostd
should use the \texttt{gige} interface on \texttt{hostc} and
\texttt{hostd}
respectively for MPI communications.  (\texttt{ifhn} stands for
``interface host name'' and should be set to an alternate host name for
the machine that is used to designate an alternate communication interface.)
This interface information causes the MPI implementation to choose the
alternate host name when making connections.  When the alternate
hostname specifies a particular interface, MPICH communication will then
travel over that interface.

You might use this machinefile in the following way:
\begin{verbatim}
    mpiexec -machinefile mf -n 7 p0
\end{verbatim}
Process rank 0 is to run on \texttt{hosta}, ranks 1 and 2 on
\texttt{hostb}, rank 3 on \texttt{hostc}, and ranks 4-6 on
\texttt{hostd}.  Note that the file specifies information for up to 8
ranks and we only used 7.  That is OK.  But, if we had used ``\texttt{-n
  9}'', an error would be raised.  The file is not used as a pool of
machines that are cycled through; the processes are mapped to the hosts
in the order specified in the file.

A more complex command-line example might be:
\begin{verbatim}
    mpiexec -l -machinefile mf -n 3 p1 : -n 2 p2 : -n 2 p3
\end{verbatim}
Here, ranks 0-2 all run program \texttt{p1} and are executed placing
rank 0 on \texttt{hosta} and ranks 1-2 on \texttt{hostb}.  Similarly,
ranks 3-4 run \texttt{p2} and are executed on \texttt{hostc} and
\texttt{hostd}, respectively.  Ranks 5-6 run on \texttt{hostd} and
execute \texttt{p3}.
\item[\texttt{-s}] can be used to direct the \texttt{stdin} of
  \texttt{mpiexec} to specific processes in a parallel job.  For
  example:
\begin{verbatim}
    mpiexec -s all -n 5 a.out
\end{verbatim}
directs the \texttt{stdin} of \texttt{mpiexec} to all five processes.
\begin{verbatim}
    mpiexec -s 4 -n 5 a.out 
\end{verbatim}
directs it to just the process with rank 4, and 
\begin{verbatim}
    mpiexec -s 1,3 -n 5 a.out 
\end{verbatim}
sends it to processes 1 and 3, while
\begin{verbatim}
    mpiexec -s 0-3 -n 5 a.out 
\end{verbatim}
sends \texttt{stdin} to processes 0, 1, 2, and 3.

The default, if \texttt{-s} is not specified, is to send
\texttt{mpiexec}'s \texttt{stdin} to process 0 only.

The redirection of \texttt{-stdin} through \texttt{mpiexec} to various MPI
processes is intended primarily for interactive use.  Because of the
complexity of buffering large amounts of data at various processes that
may not have read it yet, the redirection of large amounts of data to
\texttt{mpiexec}'s \texttt{stdin} is discouraged, and may cause
unexpected results.  That is, 
\begin{verbatim}
    mpiexec -s all -n 5 a.out < bigfile
\end{verbatim}
should not be used if \texttt{bigfile} is more than a few lines long.
Have one of the processes open the file and read it instead.  The
functions in MPI-IO may be useful for this purpose.

% the following is obsolete
% \item[\texttt{-kx}] is used only for debugging.  The \texttt{mpd}
%   process manager encapsulates the command-line arguments, the contents
%   of the \texttt{-machinefile} argment, \texttt{-configfile}, and in
%   some cases the environment, into an XML file for delivery to the
%   internals of the process manager.  Under ordinary circumstances, this
%   file, created in \texttt{/tmp}, is not seen by the user, and is
%   deleted after use.  In some cases it may be desirable to examine the
%   contents of this file after it is used, in order to debug difficulties
%   with the installation and functioning of \texttt{mpd}.  In this case,
%   the \texttt{-kx} argument causes \texttt{mpiexec} to keep the file,
%   which you will be able to find in \texttt{/tmp} with a name like
%   \texttt{smith\_tempxml\_1234}.

\end{description}

A ``\texttt{:}'' can optionally be used between global args
and normal argument sets, e.g.:
\begin{verbatim}
    mpiexec -l -n 1 -host host1 pgm1 : -n 4 -host host2 pgm2
\end{verbatim}
is equivalent to:
\begin{verbatim}
    mpiexec -l : -n 1 -host host1 pgm1 : -n 4 -host host2 pgm2
\end{verbatim}
This option implies that the global arguments can occur on a separate
line in the file specified by \texttt{-configfile} when it is used to 
replace a long command line.

\subsubsection{Environment Variables Affecting \texttt{mpiexec} for MPD}
\label{sec:mpd-mpiexec-env}

A small number of environment variables affect the behavior of
\texttt{mpiexec}. 

\begin{description}
\item[\texttt{MPIEXEC\_TIMEOUT}] The value of this environment variable is the
  maximum number of seconds this job will be permitted to run.  When
  time is up, the job is aborted.
\item[\texttt{MPIEXEC\_PORT\_RANGE}] If this environment variable is
  defined then the MPD system will restrict its usage of ports for
  connecting its various processes to ports in this range.  If this
  variable is not assigned, but \texttt{MPICH\_PORT\_RANGE} \emph{is}
  assigned, then it will use the range specified by
  \texttt{MPICH\_PORT\_RANGE} for its ports.  Otherwise, it will use
  whatever paorts are assigned to it by the system.  Port ranges are
  given as a pair of integers separated by a colon.
\item[\texttt{MPIEXEC\_BNR}] If this environment variable is defined
  (its value, if any, is currently insignificant), then MPD will act in
  backward-compatibility mode, supporting the BNR interface from the
  original MPICH (e.g. versions 1.2.0 -- 1.2.7p1)
  instead of its native PMI interface, as a way for application
  processes to interact with the process management system.
\item[\texttt{MPD\_CON\_EXT}] Adds a string to the default Unix socket
  name used by \texttt{mpiexec} to find the local \texttt{mpd}.  This
  allows one to run multiple mpd rings at the same time.
\end{description}


\subsection{Extensions for SMPD Process Management Environment}
\label{sec:extensions-smpd}

SMPD is an alternate process manager that runs on both Unix and Windows.
It can launch jobs across both platforms if the binary formats match 
(big/little endianness and size of C types-- \texttt{int},
\texttt{long}, \texttt{void*}, etc).


\subsubsection{\texttt{mpiexec} arguments for SMPD}
\label{sec:mpiexec-smpd}

\texttt{mpiexec} for smpd accepts the standard MPI-2 \texttt{mpiexec}
options.  Execute
\begin{verbatim}
    mpiexec
\end{verbatim}
or
\begin{verbatim}
    mpiexec -help2
\end{verbatim}
to print the usage options.  Typical usage:
\begin{verbatim}
    mpiexec -n 10 myapp.exe
\end{verbatim}
All options to \texttt{mpiexec}:
\begin{description}
\item[\texttt{-n x}]
\item[\texttt{-np x}]\mbox{}\\
  launch \texttt{x} processes
\item[\texttt{-localonly x}]
\item[\texttt{-np x -localonly}]\mbox{}\\
  launch \texttt{x} processes on the local machine
\item[\texttt{-machinefile filename}]\mbox{}\\
  use a file to list the names of machines to launch on
\item[\texttt{-host hostname}]\mbox{}\\
  launch on the specified host.
\item[\texttt{-hosts n host1 host2 ... hostn}]
\item[\texttt{-hosts n host1 m1 host2 m2 ... hostn mn}]\mbox{}\\
  launch on the specified hosts.
  In the second version the number of processes = m1 + m2 + ... + mn
\item[\texttt{-dir drive:$\backslash$my$\backslash$working$\backslash$directory}]
\item[\texttt{-wdir /my/working/directory}]\mbox{}\\
  launch processes with the specified working directory. (\texttt{-dir}
  and \texttt{-wdir} are equivalent)
\item[\texttt{-env var val}]\mbox{}\\
  set environment variable before launching the processes
%\item[\texttt{-nocolor}]\mbox{}\\
%  don't use process specific output coloring
%\item[\texttt{-nompi}]\mbox{}\\
%  launch processes without the mpi startup mechanism
\item[\texttt{-exitcodes}]\mbox{}\\
  print the process exit codes when each process exits.
\item[\texttt{-noprompt}]\mbox{}\\
  prevent \texttt{mpiexec} from prompting for user credentials.  Instead errors will
be printed and \texttt{mpiexec} will exit.
\item[\texttt{-localroot}]\mbox{}\\
  launch the root process directly from mpiexec if the host is local.
  (This allows the root process to create windows and be debugged.)
\item[\texttt{-port port}]
\item[\texttt{-p port}]\mbox{}\\
  specify the port that \texttt{smpd} is listening on.
\item[\texttt{-phrase passphrase}]\mbox{}\\
  specify the passphrase to authenticate connections to \texttt{smpd} with.
\item[\texttt{-smpdfile filename}]\mbox{}\\
  specify the file where the \texttt{smpd} options are stored including the 
passphrase. (unix only option)
%\item[\texttt{-soft Fortran90\_triple}]\mbox{}\\
%  acceptable number of processes to launch up to maxprocs
\item[\texttt{-path search\_path}]\mbox{}\\
  search path for executable, ; separated
% This will probably not be implemented.
%\item[\texttt{-arch architecture}]\mbox{}\\
%  sun, linux, rs6000, ...
\item[\texttt{-timeout seconds}]\mbox{}\\
  timeout for the job. 
\end{description}
Windows specific options:
\begin{description}
\item[\texttt{-map drive:$\backslash\backslash$host$\backslash$share}]\mbox{}\\
  map a drive on all the nodes
  this mapping will be removed when the processes exit
\item[\texttt{-logon}]\mbox{}\\
  prompt for user account and password
\item[\texttt{-pwdfile filename}]\mbox{}\\
  read the account and password from the file specified.

  put the account on the first line and the password on the second
%\item[\texttt{-nomapping}]\mbox{}\\
%  don't try to map the current directory on the remote nodes
\item[\texttt{-nopopup\_debug}]\mbox{}\\
  disable the system popup dialog if the process crashes
%\item[\texttt{-dbg}]\mbox{}\\
%  catch unhandled exceptions
\item[\texttt{-priority class[:level]}]\mbox{}\\
  set the process startup priority class and optionally level.\mbox{}\\
  class = 0,1,2,3,4   = idle, below, normal, above, high\mbox{}\\
  level = 0,1,2,3,4,5 = idle, lowest, below, normal, above, highest\mbox{}\\
  the default is -priority 2:3
\item[\texttt{-register}]\mbox{}\\
  encrypt a user name and password to the Windows registry.
\item[\texttt{-remove}]\mbox{}\\
  delete the encrypted credentials from the Windows registry.
\item[\texttt{-validate [-host hostname]}]\mbox{}\\
  validate the encrypted credentials for the current or specified host.
\item[\texttt{-delegate}]\mbox{}\\
  use passwordless delegation to launch processes.
\item[\texttt{-impersonate}]\mbox{}\\
  use passwordless authentication to launch processes.
\item[\texttt{-plaintext}]\mbox{}\\
  don't encrypt the data on the wire.
\end{description}


\subsection{Extensions for the gforker Process Management Environment}
\label{sec:extensions-forker}
\texttt{gforker} is a process management system for starting
processes on a single machine, so called because the MPI processes are
simply \texttt{fork}ed from the \texttt{mpiexec} process.  This process
manager supports programs that use \texttt{MPI\_Comm\_spawn} and the other
dynamic process routines, but does not support the use of the dynamic process
routines from programs that are not started with \texttt{mpiexec}.  The
\texttt{gforker} process manager is primiarily intended as a debugging aid as
it simplifies development and testing of MPI programs on a single node or
processor.  

\subsubsection{\texttt{mpiexec} arguments for gforker}
\label{sec:mpiexec-forker}

In addition to the standard \texttt{mpiexec} command-line arguments, the
\texttt{gforker} \texttt{mpiexec} supports the following options:
\begin{description}
\item[\texttt{-np <num>}]A synonym for the standard \texttt{-n} argument
\item[\texttt{-env <name> <value>}]Set the environment variable
\texttt{<name>} to \texttt{<value>} for the processes being run by
\texttt{mpiexec}.
\item[\texttt{-envnone}]Pass no environment variables (other than ones
specified with  other \texttt{-env} or \texttt{-genv} arguments) to the
processes being run by \texttt{mpiexec}. 
By default, all environment
variables are provided to each MPI process (rationale: principle of
least surprise for the user)
\item[\texttt{-envlist <list>}]Pass the listed environment variables (names
separated  by commas), with their current values, to the processes being run by
 \texttt{mpiexec}.
\item[\texttt{-genv <name> <value>}]The \item{-genv} options have the same
meaning as their corresponding \texttt{-env} version, except they apply to all
executables, not just the current executable (in the case that the colon
syntax is used to specify multiple execuables).
\item[\texttt{-genvnone}]Like \texttt{-envnone}, but for all executables
\item[\texttt{-genvlist <list>}]Like \texttt{-envlist}, but for all executables
\item[\texttt{-usize <n>}]Specify the value returned for the value of the
attribute \texttt{MPI\_UNIVERSE\_SIZE}.
\item[\texttt{-l}]Label standard out and standard error (\texttt{stdout} and \texttt{stderr}) with 
  the rank of the process
\item[\texttt{-maxtime <n>}]Set a timelimit of \texttt{<n>} seconds.
\item[\texttt{-exitinfo}]Provide more information on the reason each process
exited if there is an abnormal exit
\end{description}

In addition to the commandline argments, the \texttt{gforker} \texttt{mpiexec}
provides a number of environment variables that can be used to control the
behavior of \texttt{mpiexec}:

\begin{description}
\item[\texttt{MPIEXEC\_TIMEOUT}]Maximum running time in seconds.
\texttt{mpiexec} will terminate MPI programs that take longer than the value
specified by \texttt{MPIEXEC\_TIMEOUT}.  
\item[\texttt{MPIEXEC\_UNIVERSE\_SIZE}]Set the universe size
\item[\texttt{MPIEXEC\_PORT\_RANGE}]Set the range of ports that
\texttt{mpiexec} will use  
  in communicating with the processes that it starts.  The format of 
  this is \texttt{<low>:<high>}.  For example, to specify any port between
  10000 and 10100, use \texttt{10000:10100}.  
\item[\texttt{MPICH\_PORT\_RANGE}]Has the same meaning as
\texttt{MPIEXEC\_PORT\_RANGE} and is used if \texttt{MPIEXEC\_PORT\_RANGE} is
not set. 
\item[\texttt{MPIEXEC\_PREFIX\_DEFAULT}]If this environment variable is set,
output to standard output is prefixed by the rank in \texttt{MPI\_COMM\_WORLD}
of the process and output to standard error is prefixed by the rank and the
text \texttt{(err)}; both are followed by an angle bracket (\texttt{>}).  If 
  this variable is not set, there is no prefix.
\item[\texttt{MPIEXEC\_PREFIX\_STDOUT}]Set the prefix used for lines sent to
standard output.  A \texttt{\%d} is replaced with the rank in
\texttt{MPI\_COMM\_WORLD}; a \texttt{\%w} is replaced with an indication of
which \texttt{MPI\_COMM\_WORLD} in MPI jobs that involve multiple
\texttt{MPI\_COMM\_WORLD}s (e.g., ones that use \texttt{MPI\_Comm\_spawn} or
\texttt{MPI\_Comm\_connect}). 
\item[\texttt{MPIEXEC\_PREFIX\_STDERR}]Like \texttt{MPIEXEC\_PREFIX\_STDOUT},
but for standard error. 
\item[\texttt{MPIEXEC\_STDOUTBUF}]Sets the buffering mode for standard
  output.  Valid  values are \texttt{NONE} (no buffering),
  \texttt{LINE} (buffering by lines), and \texttt{BLOCK} (buffering by
  blocks of characters; the size of the block is implementation
  defined).  The default is \texttt{NONE}. 
\item[\texttt{MPIEXEC\_STDERRBUF}]Like \texttt{MPIEXEC\_STDOUTBUF},
  but for standard error. 
\end{description}

\subsection{Restrictions of the remshell Process Management Environment}
\label{sec:restrictions-remshell}

The \texttt{remshell} ``process manager'' provides a very simple version of
\texttt{mpiexec} that makes use of the secure shell command (\texttt{ssh}) to
start processes on a collection of machines.  As this is intended primarily as
an illustration of how to build a version of \texttt{mpiexec} that works with
other process managers, it does not implement all of the features of the other
\texttt{mpiexec} programs described in this document.  In particular, it
ignores the command line options that control the environment variables given
to the MPI programs.  It does support the same output labeling features
provided by the \texttt{gforker} version of \texttt{mpiexec}. 
However, this version of \texttt{mpiexec} can be used
much like the \texttt{mpirun} for the \texttt{ch\_p4} device in MPICH-1 to run
programs on a collection of machines that allow remote shells.  A file by the
name of \texttt{machines} should contain the names of machines on which
processes can be run, one machine name per line.  There must be enough
machines listed to satisfy the requested number of processes; you can list the
same machine name multiple times if necessary.  

For more complex needs or for faster startup, we recommend the use of the
\texttt{mpd} process manager.

\section{Managing the Process Management Environment}
\label{sec:managing-pme}

Some of the process managers supply user commands that can be used to
interact with the process manager and to control jobs.  In this section
we describe user commands that may be useful.

\subsection{MPD}
\label{sec:managing-mpd}

\begin{description}
\item[\texttt{mpd}]  starts an mpd daemon.
\item[\texttt{mpdboot}] starts a set of mpd's on a list of machines.
\item[\texttt{mpdtrace}] lists all the MPD daemons that are running.  The
  \texttt{-l} option lists full hostnames and the port where the mpd is
  listening.
\item[\texttt{mpdlistjobs}] lists the jobs that the mpd's are running.
  Jobs are identified by the name of the mpd where they were submitted
  and a number.
\item[\texttt{mpdkilljob}] kills a job specified by the name returned by
  \texttt{mpdlistjobs }
\item[\texttt{mpdsigjob}] delivers a signal to the named job.  Signals
  are specified by name or number.
\end{description}
You can use keystrokes to provide signals in the usual way, where
\texttt{mpiexec} stands in for the entire parallel application.  That
is, if \texttt{mpiexec} is being run in a Unix shell in the foreground,
you can use \verb+^C+ (control-C) to send a \texttt{SIGINT} to the
processes, or \verb+^Z+ (control-Z) to suspend all of them.  A suspended
job can be continued in the usual way.

Precise argument formats can be obtained by passing any MPD command the
\texttt{--help} or \texttt{-h} argument.  More details can be found in
the \texttt{README} in the mpich2 top-level directory or the
\texttt{README} file in the MPD directory \texttt{mpich2/src/pm/mpd}.


\section{Debugging}
\label{sec:debugging}

Debugging parallel programs is notoriously difficult.  Here we describe
a number of approaches, some of which depend on the exact version of
MPICH2 you are using. 


\subsection{\texttt{gdb} via \texttt{mpiexec}}
\label{sec:gdb via mpiexec}

If you are using the MPD process manager, you can use the \texttt{-gdb}
argument to \texttt{mpiexec} to execute a program with each process
running under the control of the \texttt{gdb} sequential debugger.
The \texttt{-gdb} option helps control the multiple instances of
\texttt{gdb} by sending \texttt{stdin} either to all processes or to
a selected process and by labeling and merging output.  The current
implementation has some minor limitations.  For example, we do not
support setting your own prompt.  This is because we capture the \texttt{gdb}
output and examine it before processing it, e.g. merging identical lines.
Also, we set a breakpoint at the beginning of \texttt{main} to get all
processes synchronized at the beginning.  Thus, the user will have a
duplicate, unusable breakpoint if he sets one at the very first executable
line of \texttt{main}.  Otherwise, to the extent possible, we try to simply
pass user input through to \texttt{gdb} and lets things progress normally.

The following script of a \texttt{-gdb} session gives an idea of how
this works.  Input keystrokes are sent to all processes unless specifially
directed by the ``\texttt{z}'' command.

%
% We need to make this tiny; small is still too big for the line lengths
% that are in this example.
% But tiny makes it too hard to read, so I applied minor edits instead- RL
\begin{small}
\begin{verbatim}
ksl2% mpiexec -gdb -n 10 cpi
0-9:  (gdb) l
0-9:  5 double f(double);
0-9:  6 
0-9:  7 double f(double a)
0-9:  8 {
0-9:  9     return (4.0 / (1.0 + a*a));
0-9:  10        }
0-9:  11        
0-9:  12        int main(int argc,char *argv[])
0-9:  13        {
0-9:  14            int done = 0, n, myid, numprocs, i;
0-9:  (gdb) 
0-9:  15            double PI25DT = 3.141592653589793238462643;
0-9:  16            double mypi, pi, h, sum, x;
0-9:  17            double startwtime = 0.0, endwtime;
0-9:  18            int  namelen;
0-9:  19            char processor_name[MPI_MAX_PROCESSOR_NAME];
0-9:  20        
0-9:  21            MPI_Init(&argc,&argv);
0-9:  22            MPI_Comm_size(MPI_COMM_WORLD,&numprocs);
0-9:  23            MPI_Comm_rank(MPI_COMM_WORLD,&myid);
0-9:  24            MPI_Get_processor_name(processor_name,&namelen);
0-9:  (gdb) 
0-9:  25        
0-9:  26            fprintf(stdout,"Process %d of %d is on %s\n",
0-9:  27                    myid, numprocs, processor_name);
0-9:  28            fflush(stdout);
0-9:  29        
0-9:  30            n = 10000;       /* default # of rectangles */
0-9:  31            if (myid == 0)
0-9:  32                startwtime = MPI_Wtime();
0-9:  33        
0-9:  34            MPI_Bcast(&n, 1, MPI_INT, 0, MPI_COMM_WORLD);
0-9:  (gdb) b 30
0-9:  Breakpoint 2 at 0x4000000000002541: 
                      file /home/lusk/mpich2/examples/cpi.c, line 30.
0-9:  (gdb) r
0-9:  Continuing.
0:  Process 0 of 10 is on ksl2
1:  Process 1 of 10 is on ksl2
2:  Process 2 of 10 is on ksl2
3:  Process 3 of 10 is on ksl2
4:  Process 4 of 10 is on ksl2
5:  Process 5 of 10 is on ksl2
6:  Process 6 of 10 is on ksl2
7:  Process 7 of 10 is on ksl2
8:  Process 8 of 10 is on ksl2
9:  Process 9 of 10 is on ksl2
0-9:  
0-9:  Breakpoint 2, main (argc=1, argv=0x60000fffffffb4b8)
0-9:      at /home/lusk/mpich2/examples/cpi.c:30
0-9:  30            n = 10000;        * default # of rectangles */
0-9:  (gdb) n
0-9:  31            if (myid == 0)
0-9:  (gdb) n
0:  32          startwtime = MPI_Wtime();
1-9:  34            MPI_Bcast(&n, 1, MPI_INT, 0, MPI_COMM_WORLD);
0-9:  (gdb) z 0
0:  (gdb) n
0:  34      MPI_Bcast(&n, 1, MPI_INT, 0, MPI_COMM_WORLD);
0:  (gdb) z
0-9:  (gdb) where
0-9:  #0  main (argc=1, argv=0x60000fffffffb4b8)
0-9:      at /home/lusk/mpich2/examples/cpi.c:34
0-9:  (gdb) n
0-9:  36            h   = 1.0 / (double) n;
0-9:  (gdb) 
0-9:  37            sum = 0.0;
0-9:  (gdb) 
0-9:  39            for (i = myid + 1; i <= n; i += numprocs)
0-9:  (gdb) 
0-9:  41                x = h * ((double)i - 0.5);
0-9:  (gdb) 
0-9:  42                sum += f(x);
0-9:  (gdb) 
0-9:  39            for (i = myid + 1; i <= n; i += numprocs)
0-9:  (gdb) 
0-9:  41                x = h * ((double)i - 0.5);
0-9:  (gdb) 
0-9:  42                sum += f(x);
0-9:  (gdb) 
0-9:  39            for (i = myid + 1; i <= n; i += numprocs)
0-9:  (gdb) 
0-9:  41                x = h * ((double)i - 0.5);
0-9:  (gdb) 
0-9:  42                sum += f(x);
0-9:  (gdb) 
0-9:  39            for (i = myid + 1; i <= n; i += numprocs)
0-9:  (gdb) 
0-9:  41                x = h * ((double)i - 0.5);
0-9:  (gdb) 
0-9:  42                sum += f(x);
0-9:  (gdb) 
0-9:  39            for (i = myid + 1; i <= n; i += numprocs)
0-9:  (gdb) 
0-9:  41                x = h * ((double)i - 0.5);
0-9:  (gdb) 
0-9:  42                sum += f(x);
0-9:  (gdb) 
0-9:  39            for (i = myid + 1; i <= n; i += numprocs)
0-9:  (gdb) 
0-9:  41                x = h * ((double)i - 0.5);
0-9:  (gdb) 
0-9:  42                sum += f(x);
0-9:  (gdb) p sum
0:  $1 = 19.999875951497799
1:  $1 = 19.999867551672725
2:  $1 = 19.999858751863549
3:  $1 = 19.999849552071328
4:  $1 = 19.999839952297158
5:  $1 = 19.999829952542203
6:  $1 = 19.999819552807658
7:  $1 = 19.999808753094769
8:  $1 = 19.999797553404832
9:  $1 = 19.999785953739192
0-9:  (gdb) c
0-9:  Continuing.
0:  pi is approximately 3.1415926544231256, Error is 0.0000000008333325
1-9:  
1-9:  Program exited normally.
1-9:  (gdb) 0:  wall clock time = 44.909412
0:  
0:  Program exited normally.
0:  (gdb) q
0-9:  MPIGDB ENDING
ksl2% 
\end{verbatim}
\end{small}
You can attach to a running job with
\begin{verbatim}
    mpiexec -gdba <jobid>
\end{verbatim}
where \texttt{<jobid>} comes from \texttt{mpdlistjobs}.


\subsection{TotalView}
\label{sec:totalview}

MPICH2 supports use of the TotalView debugger from Etnus, through the
MPD process manager only.  If 
MPICH2 has been configured to enable debugging with TotalView
(See the section on configuration of the MPD process manager in the
\emph{Installer's Guide}) then one can debug an MPI program started
with MPD by adding \texttt{-tv} to the global \texttt{mpiexec}
arguments, as in
\begin{verbatim}
    mpiexec -tv -n 3 cpi
\end{verbatim}
You will get a popup window from TotalView asking whether you want to
start the job in a stopped state.  If so,
when the TotalView window appears, you may see assembly code in the
source window.  Click on \texttt{main} in the stack window (upper left)
to see the source of the \texttt{main} function.  TotalView will show
that the program (all processes) are stopped in the call to
\texttt{MPI\_Init}. 

When debugging with TotalView using the above startup sequence, MPICH2
jobs cannot be restarted without exiting TotalView. In MPICH2 version
1.0.5p4 or later, TotalView can be invoked on an MPICH2 job as follows: 
\begin{verbatim}
    totalview python -a `which mpiexec` -tvsu \
                 <mpiexec args> <program> <program args>
\end{verbatim}
and the MPICH2 job will be fully restartable within TotalView.

If you have MPICH2 version 1.0.5p4 or later and TotalView 8.1.0 or later,
you can use a TotalView feature called indirect launch with
MPICH2. Invoke TotalView as: 
\begin{verbatim}
    totalview <program> -a <program args>
\end{verbatim}
Then select the Process/Startup Parameters command. Choose the
Parallel tab in the resulting dialog box and choose MPICH2 as the
parallel system. Then set the number of tasks using the Tasks field 
and enter other needed mpiexec arguments into the Additional
Starter Arguments field.  

If you want to be able to attach to a running MPICH2 job using
TotalView, you must use the \texttt{-tvsu} option to mpiexec when
starting the job. Using this option will add a barrier inside
\texttt{MPI\_Init} and hence may affect startup performance slightly. It
will have no effect on the running of the job once all tasks have
returned from \texttt{MPI\_Init}. In order to debug a running MPICH2
job, you must attach TotalView to the instance of Python that is
running the mpiexec script. If you have just one task running on the
node where you invoked mpiexec, and no other Python scripts running,
there will be three instances of Python running on the node. One of
these is the parent of the MPICH2 task on that node, and one is the
parent of that Python process. Neither of those is the instance of Python you
want to attach to---they are both running the MPD script. The third
instance of Python has no children and is not the child of a
Python process. That is the one that is running mpiexec and is the one you
want to attach to.

\section{Other Tools Provided with MPICH2}
\label{sec:other-tools}
MPICH2 comes with the same MPE (Multi-Processing Environment) tools that are
included with MPICH1.  These include several trace libraries for recording the
execution of MPI programs and the Jumpshot and SLOG tools for performance
visualization.  The MPE tools are built and installed by default and should be
available without requiring any additional steps.  MPE is documented in 
a separate manual.

MPICH2 also includes a test suite for MPI-1 and MPI-2 functionality; this
suite may be found in the \texttt{mpich2/test/mpi} source directory and can be
run with the command \texttt{make testing}.  This test suite should work with
any MPI implementation, not just MPICH2.

\section{MPICH2 under Windows}
\label{sec:windows}

\subsection{Directories}
\label{sec:windir}

The default installation of MPICH2 is in
\texttt{C:$\backslash$Program Files$\backslash$MPICH2}. Under the installation
directory are three sub-directories: \texttt{include}, \texttt{bin}, and
\texttt{lib}.  The \texttt{include} and \texttt{lib} directories contain
the header files and libraries necessary to compile MPI applications.  
The \texttt{bin} directory contains the process manager, \texttt{smpd.exe},
and the MPI job launcher, \texttt{mpiexec.exe}.  The dlls that implement
MPICH2 are copied to the Windows system32 directory.

\subsection{Compiling}
\label{sec:wincompile}

The libraries in the \texttt{lib} directory were compiled with MS Visual C++ .NET 2003
and Intel Fortran 8.1.  These 
compilers and any others that can link with the MS \texttt{.lib} files can be used to
create user applications.  \texttt{gcc} and \texttt{g77} for \texttt{cygwin} can be used with the 
\texttt{libmpich*.a} libraries.

For MS Developer Studio users: Create a project and add
\begin{verbatim}
    C:\Program Files\MPICH2\include
\end{verbatim}
to the include path and
\begin{verbatim}
    C:\Program Files\MPICH2\lib
\end{verbatim}
to
the library path.  Add \texttt{mpi.lib} and \texttt{cxx.lib} to the
link command.  Add \texttt{cxxd.lib} to the Debug target link instead of
\texttt{cxx.lib}.

Intel Fortran 8 users should add \texttt{fmpich2.lib} to the link command. 

Cygwin users should use \texttt{libmpich2.a} \texttt{libfmpich2g.a}.

\subsection{Running}
\label{sec:winrun}

MPI jobs are run from a command prompt using \texttt{mpiexec.exe}.  See
Section~\ref{sec:extensions-smpd} on \texttt{mpiexec} for \texttt{smpd}
for a description of the options to \texttt{mpiexec}.

\clearpage
\appendix

\section{Frequently Asked Questions}
This is the content of the online FAQ, as of June 23, 2006.

% The FAQ builder is able to generate clean LaTeX that can be used immediately.
\input{faqsum.tex}

\bibliographystyle{plain}
\bibliography{user}

\end{document}


