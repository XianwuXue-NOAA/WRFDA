\startmanpage
\mantitle{MPI{\tt \char`\_}Op{\tt \char`\_}free}{tex}{12/9/2006}
\manname{MPI{\tt \char`\_}Op{\tt \char`\_}free}
--- Frees a user-defined combination function handle 
\subhead{Synopsis}
\startvb\begin{verbatim}
int MPI_Op_free(MPI_Op *op)

\end{verbatim}
\endvb

\subhead{Input Parameter}
\startarg{op }{operation (handle) 
}
\par
\subhead{Notes}
{\tt op} is set to {\tt MPI{\tt \char`\_}OP{\tt \char`\_}NULL} on exit.
\par
\subhead{Null Handles}
The MPI 1.1 specification, in the section on opaque objects, explicitly
disallows freeing a null communicator.  The text from the standard is:
\begin{verbatim}
 A null handle argument is an erroneous IN argument in MPI calls, unless an
 exception is explicitly stated in the text that defines the function. Such
 exception is allowed for handles to request objects in Wait and Test calls
 (sections Communication Completion and Multiple Completions ). Otherwise, a
 null handle can only be passed to a function that allocates a new object and
 returns a reference to it in the handle.
\end{verbatim}

\par
\subhead{Thread and Interrupt Safety}
\par
This routine is thread-safe.  This means that this routine may be
safely used by multiple threads without the need for any user-provided
thread locks.  However, the routine is not interrupt safe.  Typically,
this is due to the use of memory allocation routines such as {\tt malloc
}or other non-MPICH runtime routines that are themselves not interrupt-safe.
\par
\subhead{Notes for Fortran}
All MPI routines in Fortran (except for {\tt MPI{\tt \char`\_}WTIME} and {\tt MPI{\tt \char`\_}WTICK}) have
an additional argument {\tt ierr} at the end of the argument list.  {\tt ierr
}is an integer and has the same meaning as the return value of the routine
in C.  In Fortran, MPI routines are subroutines, and are invoked with the
{\tt call} statement.
\par
All MPI objects (e.g., {\tt MPI{\tt \char`\_}Datatype}, {\tt MPI{\tt \char`\_}Comm}) are of type {\tt INTEGER
}in Fortran.
\par
\subhead{Errors}
\par
All MPI routines (except {\tt MPI{\tt \char`\_}Wtime} and {\tt MPI{\tt \char`\_}Wtick}) return an error value;
C routines as the value of the function and Fortran routines in the last
argument.  Before the value is returned, the current MPI error handler is
called.  By default, this error handler aborts the MPI job.  The error handler
may be changed with {\tt MPI{\tt \char`\_}Comm{\tt \char`\_}set{\tt \char`\_}errhandler} (for communicators),
{\tt MPI{\tt \char`\_}File{\tt \char`\_}set{\tt \char`\_}errhandler} (for files), and {\tt MPI{\tt \char`\_}Win{\tt \char`\_}set{\tt \char`\_}errhandler} (for
RMA windows).  The MPI-1 routine {\tt MPI{\tt \char`\_}Errhandler{\tt \char`\_}set} may be used but
its use is deprecated.  The predefined error handler
{\tt MPI{\tt \char`\_}ERRORS{\tt \char`\_}RETURN} may be used to cause error values to be returned.
Note that MPI does {\em not} guarentee that an MPI program can continue past
an error; however, MPI implementations will attempt to continue whenever
possible.
\par
\startarg{MPI{\tt \char`\_}SUCCESS }{No error; MPI routine completed successfully.
}
\startarg{MPI{\tt \char`\_}ERR{\tt \char`\_}ARG }{Invalid argument.  Some argument is invalid and is not
identified by a specific error class (e.g., {\tt MPI{\tt \char`\_}ERR{\tt \char`\_}RANK}).
}
\startarg{MPI{\tt \char`\_}ERR{\tt \char`\_}ARG }{Invalid argument; the error code associated with this
error indicates an attempt to free an MPI permanent operation (e.g., 
{\tt MPI{\tt \char`\_}SUM}).
}
\par
\subhead{See Also}
 MPI{\tt \char`\_}Op{\tt \char`\_}create
\nextline
\location{op{\tt \char`\_}free.c}
\endmanpage
