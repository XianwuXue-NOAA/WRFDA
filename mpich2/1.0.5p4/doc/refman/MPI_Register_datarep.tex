\startmanpage
\mantitle{MPI{\tt \char`\_}Register{\tt \char`\_}datarep}{tex}{3/31/2006}
\manname{MPI{\tt \char`\_}Register{\tt \char`\_}datarep}
--- Register functions for user-defined data  representations 
\subhead{Synopsis}
\startvb\begin{verbatim}
int MPI_Register_datarep(char *name,
                       MPI_Datarep_conversion_function *read_conv_fn,
                       MPI_Datarep_conversion_function *write_conv_fn,
                       MPI_Datarep_extent_function *extent_fn,
                       void *state)

\end{verbatim}
\endvb

\subhead{Input Parameters}
\startarg{name }{data representation name (string)
}
\startarg{read{\tt \char`\_}conv{\tt \char`\_}fn }{function invoked to convert from file representation to
native representation (function)
}
\startarg{write{\tt \char`\_}conv{\tt \char`\_}fn }{function invoked to convert from native representation to
file representation (function)
}
\startarg{extent{\tt \char`\_}fn }{function invoked to get the exted of a datatype as represented
in the file (function)
}
\startarg{extra{\tt \char`\_}state }{pointer to extra state that is passed to each of the
three functions
}
\par
\subhead{Notes}
This function allows the user to provide routines to convert data from
an external representation, used within a file, and the native representation,
used within the CPU.  There is one predefined data representation,
{\tt external32}.  Please consult the MPI-2 standard for details on this
function.
\par
\par
\location{register{\tt \char`\_}datarep.c}
\endmanpage
