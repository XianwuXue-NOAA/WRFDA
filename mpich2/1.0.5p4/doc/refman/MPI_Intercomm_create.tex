\startmanpage
\mantitle{MPI{\tt \char`\_}Intercomm{\tt \char`\_}create}{tex}{12/9/2006}
\manname{MPI{\tt \char`\_}Intercomm{\tt \char`\_}create}
--- Creates an intercommuncator from two intracommunicators 
\subhead{Synopsis}
\startvb\begin{verbatim}
int MPI_Intercomm_create(MPI_Comm local_comm, int local_leader, 
                       MPI_Comm peer_comm, int remote_leader, int tag, 
                       MPI_Comm *newintercomm)

\end{verbatim}
\endvb

\subhead{Input Parameters}
\startarg{local{\tt \char`\_}comm }{Local (intra)communicator
}
\startarg{local{\tt \char`\_}leader }{Rank in local{\tt \char`\_}comm of leader (often 0)
}
\startarg{peer{\tt \char`\_}comm }{Communicator used to communicate between a 
designated process in the other communicator.  
Significant only at the process in {\tt local{\tt \char`\_}comm} with
rank {\tt local{\tt \char`\_}leader}.
}
\startarg{remote{\tt \char`\_}leader }{Rank in peer{\tt \char`\_}comm of remote leader (often 0)
}
\startarg{tag }{Message tag to use in constructing intercommunicator; if multiple
{\tt MPI{\tt \char`\_}Intercomm{\tt \char`\_}creates} are being made, they should use different tags (more
precisely, ensure that the local and remote leaders are using different
tags for each {\tt MPI{\tt \char`\_}intercomm{\tt \char`\_}create}).
}
\par
\subhead{Output Parameter}
\startarg{comm{\tt \char`\_}out }{Created intercommunicator
}
\par
\subhead{Notes}
{\tt peer{\tt \char`\_}comm} is significant only for the process designated the
{\tt local{\tt \char`\_}leader} in the {\tt local{\tt \char`\_}comm}.
\par
The MPI 1.1 Standard contains two mutually exclusive comments on the
input intracommunicators.  One says that their repective groups must be
disjoint; the other that the leaders can be the same process.  After
some discussion by the MPI Forum, it has been decided that the groups must
be disjoint.  Note that the {\em reason} given for this in the standard is
{\em not} the reason for this choice; rather, the {\em other} operations on
intercommunicators (like {\tt MPI{\tt \char`\_}Intercomm{\tt \char`\_}merge}) do not make sense if the
groups are not disjoint.
\par
\subhead{Thread and Interrupt Safety}
\par
This routine is thread-safe.  This means that this routine may be
safely used by multiple threads without the need for any user-provided
thread locks.  However, the routine is not interrupt safe.  Typically,
this is due to the use of memory allocation routines such as {\tt malloc
}or other non-MPICH runtime routines that are themselves not interrupt-safe.
\par
\subhead{Notes for Fortran}
All MPI routines in Fortran (except for {\tt MPI{\tt \char`\_}WTIME} and {\tt MPI{\tt \char`\_}WTICK}) have
an additional argument {\tt ierr} at the end of the argument list.  {\tt ierr
}is an integer and has the same meaning as the return value of the routine
in C.  In Fortran, MPI routines are subroutines, and are invoked with the
{\tt call} statement.
\par
All MPI objects (e.g., {\tt MPI{\tt \char`\_}Datatype}, {\tt MPI{\tt \char`\_}Comm}) are of type {\tt INTEGER
}in Fortran.
\par
\subhead{Errors}
\par
All MPI routines (except {\tt MPI{\tt \char`\_}Wtime} and {\tt MPI{\tt \char`\_}Wtick}) return an error value;
C routines as the value of the function and Fortran routines in the last
argument.  Before the value is returned, the current MPI error handler is
called.  By default, this error handler aborts the MPI job.  The error handler
may be changed with {\tt MPI{\tt \char`\_}Comm{\tt \char`\_}set{\tt \char`\_}errhandler} (for communicators),
{\tt MPI{\tt \char`\_}File{\tt \char`\_}set{\tt \char`\_}errhandler} (for files), and {\tt MPI{\tt \char`\_}Win{\tt \char`\_}set{\tt \char`\_}errhandler} (for
RMA windows).  The MPI-1 routine {\tt MPI{\tt \char`\_}Errhandler{\tt \char`\_}set} may be used but
its use is deprecated.  The predefined error handler
{\tt MPI{\tt \char`\_}ERRORS{\tt \char`\_}RETURN} may be used to cause error values to be returned.
Note that MPI does {\em not} guarentee that an MPI program can continue past
an error; however, MPI implementations will attempt to continue whenever
possible.
\par
\startarg{MPI{\tt \char`\_}SUCCESS }{No error; MPI routine completed successfully.
}
\startarg{MPI{\tt \char`\_}ERR{\tt \char`\_}COMM }{Invalid communicator.  A common error is to use a null
communicator in a call (not even allowed in {\tt MPI{\tt \char`\_}Comm{\tt \char`\_}rank}).
}
\startarg{MPI{\tt \char`\_}ERR{\tt \char`\_}TAG }{Invalid tag argument.  Tags must be non-negative; tags
in a receive ({\tt MPI{\tt \char`\_}Recv}, {\tt MPI{\tt \char`\_}Irecv}, {\tt MPI{\tt \char`\_}Sendrecv}, etc.) may
also be {\tt MPI{\tt \char`\_}ANY{\tt \char`\_}TAG}.  The largest tag value is available through the 
the attribute {\tt MPI{\tt \char`\_}TAG{\tt \char`\_}UB}.
}
\startarg{MPI{\tt \char`\_}ERR{\tt \char`\_}INTERN }{This error is returned when some part of the MPICH 
implementation is unable to acquire memory.  
}
\startarg{MPI{\tt \char`\_}ERR{\tt \char`\_}RANK }{Invalid source or destination rank.  Ranks must be between
zero and the size of the communicator minus one; ranks in a receive
({\tt MPI{\tt \char`\_}Recv}, {\tt MPI{\tt \char`\_}Irecv}, {\tt MPI{\tt \char`\_}Sendrecv}, etc.) may also be {\tt MPI{\tt \char`\_}ANY{\tt \char`\_}SOURCE}.
}
\par
\subhead{See Also}
 MPI{\tt \char`\_}Intercomm{\tt \char`\_}merge, MPI{\tt \char`\_}Comm{\tt \char`\_}free, MPI{\tt \char`\_}Comm{\tt \char`\_}remote{\tt \char`\_}group, 
\nextline
MPI{\tt \char`\_}Comm{\tt \char`\_}remote{\tt \char`\_}size
\par
\location{intercomm{\tt \char`\_}create.c}
\endmanpage
