\startmanpage
\mantitle{MPI{\tt \char`\_}Type{\tt \char`\_}create{\tt \char`\_}indexed{\tt \char`\_}block}{tex}{9/1/2006}
\manname{MPI{\tt \char`\_}Type{\tt \char`\_}create{\tt \char`\_}indexed{\tt \char`\_}block}
--- Create an indexed datatype with constant-sized blocks 
\subhead{Synopsis}
\startvb\begin{verbatim}
int MPI_Type_create_indexed_block(int count,
                               int blocklength,
                               int array_of_displacements[],
                               MPI_Datatype oldtype,
                               MPI_Datatype *newtype)

\end{verbatim}
\endvb

\subhead{Input Parameters}
\startarg{count }{length of array of displacements (integer) 
}
\startarg{blocklength }{size of block (integer) 
}
\startarg{array{\tt \char`\_}of{\tt \char`\_}displacements }{array of displacements (array of integer) 
}
\startarg{oldtype }{old datatype (handle) 
}
\par
\subhead{Output Parameter}
\startarg{newtype }{new datatype (handle) 
}
\par
\subhead{Notes}
The indices are displacements, and are based on a zero origin.  A common error
is to do something like the following
\begin{verbatim}
    integer a(100)
    integer blens(10), indices(10)
    do i=1,10
10       indices(i) = 1 + (i-1)*10
    call MPI_TYPE_CREATE_INDEXED_BLOCK(10,1,indices,MPI_INTEGER,newtype,ierr)
    call MPI_TYPE_COMMIT(newtype,ierr)
    call MPI_SEND(a,1,newtype,...)
\end{verbatim}

expecting this to send {\tt a(1),a(11),...} because the indices have values
{\tt 1,11,...}.   Because these are {\em displacements} from the beginning of {\tt a},
it actually sends {\tt a(1+1),a(1+11),...}.
\par
If you wish to consider the displacements as indices into a Fortran array,
consider declaring the Fortran array with a zero origin
\begin{verbatim}
    integer a(0:99)
\end{verbatim}

\par
\subhead{Thread and Interrupt Safety}
\par
This routine is thread-safe.  This means that this routine may be
safely used by multiple threads without the need for any user-provided
thread locks.  However, the routine is not interrupt safe.  Typically,
this is due to the use of memory allocation routines such as {\tt malloc
}or other non-MPICH runtime routines that are themselves not interrupt-safe.
\par
\subhead{Notes for Fortran}
All MPI routines in Fortran (except for {\tt MPI{\tt \char`\_}WTIME} and {\tt MPI{\tt \char`\_}WTICK}) have
an additional argument {\tt ierr} at the end of the argument list.  {\tt ierr
}is an integer and has the same meaning as the return value of the routine
in C.  In Fortran, MPI routines are subroutines, and are invoked with the
{\tt call} statement.
\par
All MPI objects (e.g., {\tt MPI{\tt \char`\_}Datatype}, {\tt MPI{\tt \char`\_}Comm}) are of type {\tt INTEGER
}in Fortran.
\par
\subhead{Errors}
\par
All MPI routines (except {\tt MPI{\tt \char`\_}Wtime} and {\tt MPI{\tt \char`\_}Wtick}) return an error value;
C routines as the value of the function and Fortran routines in the last
argument.  Before the value is returned, the current MPI error handler is
called.  By default, this error handler aborts the MPI job.  The error handler
may be changed with {\tt MPI{\tt \char`\_}Comm{\tt \char`\_}set{\tt \char`\_}errhandler} (for communicators),
{\tt MPI{\tt \char`\_}File{\tt \char`\_}set{\tt \char`\_}errhandler} (for files), and {\tt MPI{\tt \char`\_}Win{\tt \char`\_}set{\tt \char`\_}errhandler} (for
RMA windows).  The MPI-1 routine {\tt MPI{\tt \char`\_}Errhandler{\tt \char`\_}set} may be used but
its use is deprecated.  The predefined error handler
{\tt MPI{\tt \char`\_}ERRORS{\tt \char`\_}RETURN} may be used to cause error values to be returned.
Note that MPI does {\em not} guarentee that an MPI program can continue past
an error; however, MPI implementations will attempt to continue whenever
possible.
\par
\startarg{MPI{\tt \char`\_}SUCCESS }{No error; MPI routine completed successfully.
}
\startarg{MPI{\tt \char`\_}ERR{\tt \char`\_}TYPE }{Invalid datatype argument.  May be an uncommitted 
MPI{\tt \char`\_}Datatype (see {\tt MPI{\tt \char`\_}Type{\tt \char`\_}commit}).
}
\startarg{MPI{\tt \char`\_}ERR{\tt \char`\_}ARG }{Invalid argument.  Some argument is invalid and is not
identified by a specific error class (e.g., {\tt MPI{\tt \char`\_}ERR{\tt \char`\_}RANK}).
}
\location{type{\tt \char`\_}create{\tt \char`\_}indexed{\tt \char`\_}block.c}
\endmanpage
