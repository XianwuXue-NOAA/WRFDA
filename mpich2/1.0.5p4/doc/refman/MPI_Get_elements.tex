\startmanpage
\mantitle{MPI{\tt \char`\_}Get{\tt \char`\_}elements}{tex}{12/9/2006}
\manname{MPI{\tt \char`\_}Get{\tt \char`\_}elements}
--- get{\tt \char`\_}elements 
\subhead{Synopsis}
\startvb\begin{verbatim}
int MPI_Get_elements(MPI_Status *status, MPI_Datatype datatype, int *elements)

\end{verbatim}
\endvb

\subhead{Arguments}
\startarg{MPI{\tt \char`\_}Status *status }{status
}
\startarg{MPI{\tt \char`\_}Datatype datatype }{datatype
}
\startarg{int *elements }{elements
}
\par
\subhead{Notes}
\par
If the size of the datatype is zero and the amount of data returned as
determined by {\tt status} is also zero, this routine will return a count of
zero.  This is consistent with a clarification made by the MPI Forum.
\par
\subhead{Notes for Fortran}
All MPI routines in Fortran (except for {\tt MPI{\tt \char`\_}WTIME} and {\tt MPI{\tt \char`\_}WTICK}) have
an additional argument {\tt ierr} at the end of the argument list.  {\tt ierr
}is an integer and has the same meaning as the return value of the routine
in C.  In Fortran, MPI routines are subroutines, and are invoked with the
{\tt call} statement.
\par
All MPI objects (e.g., {\tt MPI{\tt \char`\_}Datatype}, {\tt MPI{\tt \char`\_}Comm}) are of type {\tt INTEGER
}in Fortran.
\par
\subhead{Errors}
\par
All MPI routines (except {\tt MPI{\tt \char`\_}Wtime} and {\tt MPI{\tt \char`\_}Wtick}) return an error value;
C routines as the value of the function and Fortran routines in the last
argument.  Before the value is returned, the current MPI error handler is
called.  By default, this error handler aborts the MPI job.  The error handler
may be changed with {\tt MPI{\tt \char`\_}Comm{\tt \char`\_}set{\tt \char`\_}errhandler} (for communicators),
{\tt MPI{\tt \char`\_}File{\tt \char`\_}set{\tt \char`\_}errhandler} (for files), and {\tt MPI{\tt \char`\_}Win{\tt \char`\_}set{\tt \char`\_}errhandler} (for
RMA windows).  The MPI-1 routine {\tt MPI{\tt \char`\_}Errhandler{\tt \char`\_}set} may be used but
its use is deprecated.  The predefined error handler
{\tt MPI{\tt \char`\_}ERRORS{\tt \char`\_}RETURN} may be used to cause error values to be returned.
Note that MPI does {\em not} guarentee that an MPI program can continue past
an error; however, MPI implementations will attempt to continue whenever
possible.
\par
\startarg{MPI{\tt \char`\_}SUCCESS }{No error; MPI routine completed successfully.
}
\location{get{\tt \char`\_}elements.c}
\endmanpage
