\startmanpage
\mantitle{MPI{\tt \char`\_}Intercomm{\tt \char`\_}merge}{tex}{12/9/2006}
\manname{MPI{\tt \char`\_}Intercomm{\tt \char`\_}merge}
--- Creates an intracommuncator from an intercommunicator 
\subhead{Synopsis}
\startvb\begin{verbatim}
int MPI_Intercomm_merge(MPI_Comm intercomm, int high, MPI_Comm *newintracomm)

\end{verbatim}
\endvb

\subhead{Input Parameters}
\startarg{comm }{Intercommunicator (handle)
}
\startarg{high }{Used to order the groups within comm (logical)
when creating the new communicator.  This is a boolean value; the group
that sets high true has its processes ordered {\em after} the group that sets 
this value to false.  If all processes in the intercommunicator provide
the same value, the choice of which group is ordered first is arbitrary.
}
\par
\subhead{Output Parameter}
\startarg{comm{\tt \char`\_}out }{Created intracommunicator (handle)
}
\par
\subhead{Notes}
While all processes may provide the same value for the {\tt high} parameter,
this requires the MPI implementation to determine which group of
processes should be ranked first.
\par
\subhead{Thread and Interrupt Safety}
\par
This routine is thread-safe.  This means that this routine may be
safely used by multiple threads without the need for any user-provided
thread locks.  However, the routine is not interrupt safe.  Typically,
this is due to the use of memory allocation routines such as {\tt malloc
}or other non-MPICH runtime routines that are themselves not interrupt-safe.
\par
\subhead{Notes for Fortran}
All MPI routines in Fortran (except for {\tt MPI{\tt \char`\_}WTIME} and {\tt MPI{\tt \char`\_}WTICK}) have
an additional argument {\tt ierr} at the end of the argument list.  {\tt ierr
}is an integer and has the same meaning as the return value of the routine
in C.  In Fortran, MPI routines are subroutines, and are invoked with the
{\tt call} statement.
\par
All MPI objects (e.g., {\tt MPI{\tt \char`\_}Datatype}, {\tt MPI{\tt \char`\_}Comm}) are of type {\tt INTEGER
}in Fortran.
\par
\subhead{Algorithm}
s
Allocate contexts
Local and remote group leaders swap high values
Determine the high value.
Merge the two groups and make the intra-communicator
e
\par
\subhead{Errors}
\par
All MPI routines (except {\tt MPI{\tt \char`\_}Wtime} and {\tt MPI{\tt \char`\_}Wtick}) return an error value;
C routines as the value of the function and Fortran routines in the last
argument.  Before the value is returned, the current MPI error handler is
called.  By default, this error handler aborts the MPI job.  The error handler
may be changed with {\tt MPI{\tt \char`\_}Comm{\tt \char`\_}set{\tt \char`\_}errhandler} (for communicators),
{\tt MPI{\tt \char`\_}File{\tt \char`\_}set{\tt \char`\_}errhandler} (for files), and {\tt MPI{\tt \char`\_}Win{\tt \char`\_}set{\tt \char`\_}errhandler} (for
RMA windows).  The MPI-1 routine {\tt MPI{\tt \char`\_}Errhandler{\tt \char`\_}set} may be used but
its use is deprecated.  The predefined error handler
{\tt MPI{\tt \char`\_}ERRORS{\tt \char`\_}RETURN} may be used to cause error values to be returned.
Note that MPI does {\em not} guarentee that an MPI program can continue past
an error; however, MPI implementations will attempt to continue whenever
possible.
\par
\startarg{MPI{\tt \char`\_}SUCCESS }{No error; MPI routine completed successfully.
}
\startarg{MPI{\tt \char`\_}ERR{\tt \char`\_}COMM }{Invalid communicator.  A common error is to use a null
communicator in a call (not even allowed in {\tt MPI{\tt \char`\_}Comm{\tt \char`\_}rank}).
}
\startarg{MPI{\tt \char`\_}ERR{\tt \char`\_}INTERN }{This error is returned when some part of the MPICH 
implementation is unable to acquire memory.  
}
\par
\subhead{See Also}
 MPI{\tt \char`\_}Intercomm{\tt \char`\_}create, MPI{\tt \char`\_}Comm{\tt \char`\_}free
\nextline
\location{intercomm{\tt \char`\_}merge.c}
\endmanpage
