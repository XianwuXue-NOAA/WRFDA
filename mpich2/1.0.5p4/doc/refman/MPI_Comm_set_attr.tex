\startmanpage
\mantitle{MPI{\tt \char`\_}Comm{\tt \char`\_}set{\tt \char`\_}attr}{tex}{10/14/2006}
\manname{MPI{\tt \char`\_}Comm{\tt \char`\_}set{\tt \char`\_}attr}
--- Stores attribute value associated with a key 
\subhead{Synopsis}
\startvb\begin{verbatim}
int MPI_Comm_set_attr(MPI_Comm comm, int comm_keyval, void *attribute_val)

\end{verbatim}
\endvb

\subhead{Input Parameters}
\startarg{comm }{communicator to which attribute will be attached (handle) 
}
\startarg{keyval }{key value, as returned by  {\tt MPI{\tt \char`\_}Comm{\tt \char`\_}create{\tt \char`\_}keyval} (integer) 
}
\startarg{attribute{\tt \char`\_}val }{attribute value 
}
\par
\subhead{Notes}
Values of the permanent attributes {\tt MPI{\tt \char`\_}TAG{\tt \char`\_}UB}, {\tt MPI{\tt \char`\_}HOST}, {\tt MPI{\tt \char`\_}IO},
{\tt MPI{\tt \char`\_}WTIME{\tt \char`\_}IS{\tt \char`\_}GLOBAL}, {\tt MPI{\tt \char`\_}UNIVERSE{\tt \char`\_}SIZE}, {\tt MPI{\tt \char`\_}LASTUSEDCODE}, and
{\tt MPI{\tt \char`\_}APPNUM} may not be changed.
\par
The type of the attribute value depends on whether C, C++, or Fortran
is being used.
In C and C++, an attribute value is a pointer ({\tt void *}); in Fortran, it is an
address-sized integer.
\par
If an attribute is already present, the delete function (specified when the
corresponding keyval was created) will be called.
\par
\subhead{Thread and Interrupt Safety}
\par
This routine is thread-safe.  This means that this routine may be
safely used by multiple threads without the need for any user-provided
thread locks.  However, the routine is not interrupt safe.  Typically,
this is due to the use of memory allocation routines such as {\tt malloc
}or other non-MPICH runtime routines that are themselves not interrupt-safe.
\par
\subhead{Notes for Fortran}
All MPI routines in Fortran (except for {\tt MPI{\tt \char`\_}WTIME} and {\tt MPI{\tt \char`\_}WTICK}) have
an additional argument {\tt ierr} at the end of the argument list.  {\tt ierr
}is an integer and has the same meaning as the return value of the routine
in C.  In Fortran, MPI routines are subroutines, and are invoked with the
{\tt call} statement.
\par
All MPI objects (e.g., {\tt MPI{\tt \char`\_}Datatype}, {\tt MPI{\tt \char`\_}Comm}) are of type {\tt INTEGER
}in Fortran.
\par
\subhead{Errors}
\par
All MPI routines (except {\tt MPI{\tt \char`\_}Wtime} and {\tt MPI{\tt \char`\_}Wtick}) return an error value;
C routines as the value of the function and Fortran routines in the last
argument.  Before the value is returned, the current MPI error handler is
called.  By default, this error handler aborts the MPI job.  The error handler
may be changed with {\tt MPI{\tt \char`\_}Comm{\tt \char`\_}set{\tt \char`\_}errhandler} (for communicators),
{\tt MPI{\tt \char`\_}File{\tt \char`\_}set{\tt \char`\_}errhandler} (for files), and {\tt MPI{\tt \char`\_}Win{\tt \char`\_}set{\tt \char`\_}errhandler} (for
RMA windows).  The MPI-1 routine {\tt MPI{\tt \char`\_}Errhandler{\tt \char`\_}set} may be used but
its use is deprecated.  The predefined error handler
{\tt MPI{\tt \char`\_}ERRORS{\tt \char`\_}RETURN} may be used to cause error values to be returned.
Note that MPI does {\em not} guarentee that an MPI program can continue past
an error; however, MPI implementations will attempt to continue whenever
possible.
\par
\startarg{MPI{\tt \char`\_}SUCCESS }{No error; MPI routine completed successfully.
}
\startarg{MPI{\tt \char`\_}ERR{\tt \char`\_}COMM }{Invalid communicator.  A common error is to use a null
communicator in a call (not even allowed in {\tt MPI{\tt \char`\_}Comm{\tt \char`\_}rank}).
}
\startarg{MPI{\tt \char`\_}ERR{\tt \char`\_}KEYVAL }{Invalid keyval
}
\startarg{MPI{\tt \char`\_}ERR{\tt \char`\_}ARG }{This error class is associated with an error code that 
indicates that an attempt was made to free one of the permanent keys.
}
\par
\subhead{See Also}
MPI{\tt \char`\_}Comm{\tt \char`\_}create{\tt \char`\_}keyval, MPI{\tt \char`\_}Comm{\tt \char`\_}delete{\tt \char`\_}attr
\nextline
\location{comm{\tt \char`\_}set{\tt \char`\_}attr.c}
\endmanpage
