\startmanpage
\mantitle{MPI{\tt \char`\_}Type{\tt \char`\_}set{\tt \char`\_}attr}{tex}{10/14/2006}
\manname{MPI{\tt \char`\_}Type{\tt \char`\_}set{\tt \char`\_}attr}
--- Stores attribute value associated with a key 
\subhead{Synopsis}
\startvb\begin{verbatim}
int MPI_Type_set_attr(MPI_Datatype type, int type_keyval, void *attribute_val)

\end{verbatim}
\endvb

\subhead{Input Parameters}
\startarg{type }{MPI Datatype to which attribute will be attached (handle) 
}
\startarg{keyval }{key value, as returned by  {\tt MPI{\tt \char`\_}Type{\tt \char`\_}create{\tt \char`\_}keyval} (integer) 
}
\startarg{attribute{\tt \char`\_}val }{attribute value 
}
\par
\subhead{Notes}
\par
The type of the attribute value depends on whether C or Fortran is being used.
In C, an attribute value is a pointer ({\tt void *}); in Fortran, it is an
address-sized integer.
\par
If an attribute is already present, the delete function (specified when the
corresponding keyval was created) will be called.
\subhead{Notes for Fortran}
All MPI routines in Fortran (except for {\tt MPI{\tt \char`\_}WTIME} and {\tt MPI{\tt \char`\_}WTICK}) have
an additional argument {\tt ierr} at the end of the argument list.  {\tt ierr
}is an integer and has the same meaning as the return value of the routine
in C.  In Fortran, MPI routines are subroutines, and are invoked with the
{\tt call} statement.
\par
All MPI objects (e.g., {\tt MPI{\tt \char`\_}Datatype}, {\tt MPI{\tt \char`\_}Comm}) are of type {\tt INTEGER
}in Fortran.
\par
\subhead{Errors}
\par
All MPI routines (except {\tt MPI{\tt \char`\_}Wtime} and {\tt MPI{\tt \char`\_}Wtick}) return an error value;
C routines as the value of the function and Fortran routines in the last
argument.  Before the value is returned, the current MPI error handler is
called.  By default, this error handler aborts the MPI job.  The error handler
may be changed with {\tt MPI{\tt \char`\_}Comm{\tt \char`\_}set{\tt \char`\_}errhandler} (for communicators),
{\tt MPI{\tt \char`\_}File{\tt \char`\_}set{\tt \char`\_}errhandler} (for files), and {\tt MPI{\tt \char`\_}Win{\tt \char`\_}set{\tt \char`\_}errhandler} (for
RMA windows).  The MPI-1 routine {\tt MPI{\tt \char`\_}Errhandler{\tt \char`\_}set} may be used but
its use is deprecated.  The predefined error handler
{\tt MPI{\tt \char`\_}ERRORS{\tt \char`\_}RETURN} may be used to cause error values to be returned.
Note that MPI does {\em not} guarentee that an MPI program can continue past
an error; however, MPI implementations will attempt to continue whenever
possible.
\par
\startarg{MPI{\tt \char`\_}SUCCESS }{No error; MPI routine completed successfully.
}
\startarg{MPI{\tt \char`\_}ERR{\tt \char`\_}TYPE }{Invalid datatype argument.  May be an uncommitted 
MPI{\tt \char`\_}Datatype (see {\tt MPI{\tt \char`\_}Type{\tt \char`\_}commit}).
}
\startarg{MPI{\tt \char`\_}ERR{\tt \char`\_}KEYVAL }{Invalid keyval
}
\location{type{\tt \char`\_}set{\tt \char`\_}attr.c}
\endmanpage
