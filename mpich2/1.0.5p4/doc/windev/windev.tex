\documentclass[dvipdfm,11pt]{article}
\usepackage[dvipdfm]{hyperref} % Upgraded url package
\parskip=.1in

% Formatting conventions for contributors
% 
% A quoting mechanism is needed to set off things like file names, command
% names, code fragments, and other strings that would confuse the flow of
% text if left undistinguished from preceding and following text.  In this
% document we use the LaTeX macro '\texttt' to indicate such text in the
% source, which normally produces, when used as in '\texttt{special text}',
% the typewriter font.

% It is particularly easy to use this convention if one is using emacs as
% the editor and LaTeX mode within emacs for editing LaTeX documents.  In
% such a case the key sequence ^C^F^T (hold down the control key and type
% 'cft') produces '\texttt{}' with the cursor positioned between the
% braces, ready for the special text to be typed.  The closing brace can
% be skipped over by typing ^e (go to the end of the line) if entering
% text or ^C-} to just move the cursor past the brace.

% LaTeX mode is usually loaded automatically.  At Argonne, one way to 
% get several useful emacs tools working for you automatically is to put
% the following in your .emacs file.

% (require 'tex-site)
% (setq LaTeX-mode-hook '(lambda ()
%          		 (auto-fill-mode 1)
%          		 (flyspell-mode 1)
%          		 (reftex-mode 1)
% 			 (setq TeX-command "latex")))


\begin{document}
\markright{MPICH2 Windows Development Guide}
\title{{\bf MPICH2 Windows Development Guide}\thanks{This work was supported by the
    Mathematical, Information, and Computational Sciences Division
    subprogram of the Office of Advanced Scientific Computing Research,
    SciDAC Program, Office of Science, U.S. Department of Energy, under
    Contract
    W-31-109-ENG-38.}\\
  Version 1.0.3\\
  Mathematics and Computer Science Division\\
  Argonne National Laboratory}

\author{David Ashton}

\maketitle
\cleardoublepage

\pagenumbering{roman}
\tableofcontents
\clearpage

\pagenumbering{arabic}
\pagestyle{headings}

\section{Introduction}
\label{sec:intro}
This manual describes how to set up a Windows machine to build and test MPICH2 on.

\section{Build machine}
\label{sec:machine}

Build a Windows XP or Windows Server 2003 machine.  Place this machine in the MPICH domain
and create a user in the domain with the same name as your MCS user name.  Use your MCS 
password for this new account.  This isn't a requirement; you can create any username you
want with any password you want and it doesn't have to be part of the MPICH domain.
But it makes life a lot easier if you do.

\section{Test machine}
\label{sec:test_machine}

Build a Windows XP or Windows Server 2003 machine on a 32bit CPU.
Also build a Windows Server 2003 X64 machine to test the Win64 distribution.

%Note: peta07 and peta08.mcs.anl.gov have Mellanox Infiniband cards in them for testing
%the Infiniband channel under Windows.

\section{Software}

This section describes the software necessary to build MPICH2.

\subsection{Packages}
\label{sec:packages}

To build MPICH2 you will need:
\begin{enumerate}
\item Microsoft Developer Studio .NET 2003
\item Microsoft Platform SDK
\item cygwin - full installation
\item Intel Fortran compiler IA32
\item Intel Fortran compiler EMT64
%\item Mellanox 32bit and 64bit Infiniband libraries for VAPI and IBAL
\item Java SDK
\end{enumerate}

Microsoft Developer Studio can be found on the CDs from an MSDN subscription.  The
2003 version is needed to get a compiler that is up to date on standards compliance.  This
is needed to compile the C++ interface and handle types like ``long long''.

The Platform SDK can also be found on the MSDN CDs or downloaded from Microsoft.com.  The
latest version as of the writing of this document was Platform SDK - Windows Server 2003 SP1.
This package needs to be installed to get the Win64 compilers.

The Intel Fortran compilers need to be installed after Developer Studio and the PSDK because
they integrate themselves into those two products.  The regular IA32 compiler needs to be 
installed and the EMT64 compiler needs to be installed.  They are two separate packages and 
they require a license file to use.  The license file is for a single user on a single 
machine.

%The Mellanox Infiniband libraries are simply a package that needs to be unzipped.  They don't
%need to be installed because the build machine won't have Infiniband hardware on it.  I put
%the 32bit vapi package in \texttt{C:$\backslash$Mellanox} and the 64bit package in
%\texttt{C:$\backslash$Mellanox64}.  The ibal package should go in 
%\texttt{c:$\backslash$ibal} with both the 32bit and 64bit libraries in 
%\texttt{c:$\backslash$ibal$\backslash$Install$\backslash$x86} and 
%\texttt{c:$\backslash$ibal$\backslash$Install$\backslash$AMD64}.  If different pathes are 
%chosen then the scripts and Visual Studio project files will need to be updated to reflect 
%this change.

cygwin needs to be installed to get cvs, perl and ssh.  MPICH2 can also be built completely
under cygwin just like a Unix build so I install all of cygwin.  The latest cygwin packages
are needed to build MPICH2 under cygwin and the MCS mirror site for cygwin does not contain
the latest versions.  So I would select a download site other than the mcs mirror site during
the cygwin install.  I also select to use the DOS file format during the install process for 
cygwin.

After installing cygwin, bring up a cygwin bash shell and generate some ssh keys.  I believe
the command is: \texttt{ssh-keygen -t dsa}.  Then copy the public key generated to your
mcs \texttt{/home/you/.ssh/authorized\_keys} file.  This will allow you to ssh from this machine
to harley.mcs.anl.gov without typing a password.  This is important in order for the 
automated scripts to run without user intervention.

Assuming you installed cygwin to the default \texttt{c:$\backslash$cygwin} directory, add 
\texttt{c:$\backslash$cygwin$\backslash$bin} to your path environment variable.  This is 
required so the automated scripts can run tools like ssh and perl without specifying the 
full path.

The Java SDK needs to be installed so the logging library can be compiled.  This can be a problem
because MPICH2 needs to know the include path for \texttt{jni.h}.  Since the JSDK can be installed anywhere
and the path includes the version there isn't a standard path that can be added to the MPICH2 
projects.  Currently \texttt{c:$\backslash$j2sdk1.4.2\_06$\backslash$include} is used in
\texttt{mpich2$\backslash$src$\backslash$util$\backslash$logging$\backslash$rlog$\backslash$TraceInput$\backslash$TraceInput.vcproj}.
This will have to be modified if you install a different version of the JSDK.

Run the following command from a command prompt to change the Windows script engine from
GUI mode to console mode:
\begin{verbatim}
cscript //H:cscript
\end{verbatim}

\section{Building MPICH2}
\label{sec:building}

This section describes how to make various packages once you have a working build machine.

\subsection{Developer Studio automated 32bit build}
\label{sec:devstudiobuild}

The easist way to build an MPICH2 distribution is to use the Developer Studio environment
and the makewindist.bat script from the top level of the mpich2 source tree.  You can check
out mpich2 from CVS or you can simply copy this batch file from the distribution.  The batch
file knows how to check out mpich2 so it the only file required to make a distribution.

The product GUIDs need to be changed when a new release is created.  To do this run 
``\texttt{perl update\_windows\_version $<$new\_version$>$}''.  Run this script with mpich2/maint 
as the current directory so the project files can be found.  Example:
\begin{verbatim}
perl update_windows_version 1.0.3
\end{verbatim}

Or you can modify the project files by hand.  Edit mpich2/maint/mpich2i.vdproj.  The ProductCode
and PackageCode entries need to be changed to use new GUIDs.  Under Unix or Windows, uuidgen can 
be used to generate a new GUID.  The ProductVersion entry needs to be changed to match the 
version of MPICH2.

Once the version and GUIDs have been updated, commit the changes to mpich2i.vdproj to CVS.

Then you can build a distribution.

Bring up a build command prompt by selecting Start$\to$Progams$\to$Microsoft Visual Studio 
.NET 2003$\to$Visual Studio .NET Tools$\to$Visual Studio .NET 2003 Command Prompt.

Change directories to wherever you want to create the distribution.  mpich2 will be checked 
out under the current directory.  Run the makewindist batch file:

\begin{verbatim}
makewindist.bat --with-checkout
\end{verbatim}

Since there are no bugs in MPICH2 everything will work automatically.  The batch file does the
following steps:
\begin{enumerate}
\item check out mpich2allWithMPE and change into the mpich2 directory
\item ssh to harley.mcs.anl.gov and checkout mpich2.  Then run \texttt{maint/updatefiles} on
harley, collect all the generated \texttt{.h.in} files and copy them back to the Windows
machine.
\item run ``\texttt{winconfigure.wsf --cleancode}'' to configure mpich2 for Windows and output
all the generated files like mpi.h and the fortran interface files, etc.
\item run the Developer Studio command line tool to build all the pieces of MPICH2.  This
includes each of the channels - sock, ssm, shm, sshm, and the multi-threaded sock
channel.  Two versions of each channel are built, the regular release build and the rlog
profiled version.  The mpi wrapper channel selector dll is built and three Fortran interfaces
are built, one for each set of common symbol types and calling conventions.  mpiexec and
smpd are built along with the Windows GUI tools and the cygwin libraries.  (These are the cygwin
link libraries to use the Windows native build of MPICH2, not a Unix-style build of MPICH2
under cygwin.)
\item package up everthing into \texttt{maint$\backslash$ReleaseMSI$\backslash$mpich2.msi}.
\end{enumerate}

When the batch file is finished you will be left with a mpich2.msi file that can be used to
install MPICH2 on any Win32 machine.  This file can be re-named to match our release naming 
conventions.

\subsubsection{Automated build from the source distribution}
If you don't have access to the CVS repository but you have the MPICH2 source
distribution you can still use the makewindist batch file.  Follow these steps:
\begin{enumerate}
\item unzip/untar the source distribution
\item bring up a Visual Studio Command Prompt
\item cd into the mpich2xxx directory
\item execute ``\texttt{winconfigure.wsf --cleancode}''
\item execute ``\texttt{makewindist.bat --with-curdir}''
\end{enumerate}

\subsubsection{Building without Fortran}
If you don't have a Fortran compiler you can use winconfigure.wsf to remove the
Fortran projects.  Execute \texttt{winconfigure.wsf --remove-fortran --cleancode}
Then you can build the projects without Fortran support.  If you want to use the
\texttt{makewindist.bat} script you will need to remove the Fortran lines from
it before executing it.

\subsection{Platform SDK builds}
\label{sec:psdk_build}

The makefile in the \texttt{mpich2$\backslash$winbuild} directory builds a distribution based 
on what compilers are specified in the environment so it can be used to compile any version of
MPICH2.  The following targets can all be built with this mechanism:
\begin{itemize}
\item Win64 X64
\item Win64 IA64
\item Win32 x86
\end{itemize}

But first you need to have mpich2 checked out and configured before building.
\begin{enumerate}
\item bring up a command prompt and check out mpich2: 

\texttt{cvs -d :ext:ashton@harley.mcs.anl.gov:/home/MPI/cvsMaster co mpich2allWithMPE}.  Replace ashton with your own username.
\item cd into mpich2
\item run \texttt{getdotin.bat}
\item run \texttt{winconfigure.wsf --cleancode}
\end{enumerate}

To build the Win64 X64 distribution do the following:
\begin{enumerate}
\item Bring up a build command prompt from the PSDK.  It can be found here: Start$\to$Programs
$\to$Microsoft Platform SDK for Windows Server 2003 SP1$\to$Open Build Environment Window$\to$
Windows Server 2003 64-bit Build Environment$\to$Set Win Svr 2003 x64 Build Env (Retail)
\item Run \texttt{$\backslash$Program Files$\backslash$Intel$\backslash$Fortran$\backslash$compiler80$\backslash$Ia32e$\backslash$Bin$\backslash$ifortvars.bat}
\item cd into \texttt{mpich2$\backslash$winbuild}
\item run \texttt{build.bat 2>\&1 | tee build.x64.out}
\end{enumerate}

To build the Win64 IA64 distribution do the following:
\begin{enumerate}
\item Bring up a build command prompt from the PSDK.  It can be found here: Start$\to$Programs
$\to$Microsoft Platform SDK for Windows Server 2003 SP1$\to$Open Build Environment Window$\to$
Windows Server 2003 64-bit Build Environment$\to$Set Win Svr 2003 IA64 Build Env (Retail)
\item Run \texttt{$\backslash$Program Files$\backslash$Intel$\backslash$Fortran$\backslash$compiler80$\backslash$Itanium$\backslash$Bin$\backslash$ifortvars.bat}
\item cd into \texttt{mpich2$\backslash$winbuild}
\item run \texttt{build.bat 2>\&1 | tee build.ia64.out}
\end{enumerate}

To build the Win32 x86 distribution do the following:
\begin{enumerate}
\item Bring up a build command prompt from the PSDK.  It can be found here: Start$\to$Programs
$\to$Microsoft Platform SDK for Windows Server 2003 SP1$\to$Open Build Environment Window$\to$
Windows 2000 Build Environment$\to$Set Windows 2000 Build Environment (Retail)
\item Run \texttt{$\backslash$Program Files$\backslash$Intel$\backslash$Fortran$\backslash$compiler80$\backslash$Ia32$\backslash$Bin$\backslash$ifortvars.bat}
\item cd into \texttt{mpich2$\backslash$winbuild}
\item run \texttt{build.bat 2>\&1 | tee build.x86.out}
\end{enumerate}

\section{Distributing MPICH2 builds}
\label{sec:distribute}

If you built an .msi file using the Developer Studio build process \ref{sec:devstudiobuild} then
all you have to do is rename the \texttt{mpich2.msi} file to something appropriate like
\texttt{mpich2-1.0.3-1-win32-ia32.msi}

If you built using the Platform SDK build process \ref{sec:psdk_build} then the output files
are left in their build locations and need to be collected and put in a zip file for
distributing.  This process should be automated with a script.

\section{Testing MPICH2}
\label{sec:testing}

Run the \texttt{testmpich2.wsf} script to checkout mpich2, build it, install it, checkout 
the test suites, build them, run the test suites, and collect the results in a web page.

\subsection{Testing from scratch}
Explain the use of testmpich2.wsf.

Run ``\texttt{testmpich2.wsf}'' without any parameters and it will create a \texttt{testmpich2}
subdirectory and check out into that directory mpich2 and the test suites - c++, mpich, intel
and mpich2.  It will then build mpich2 and all the tests from the test suites.  Then it will run
the tests and place a summary in \texttt{testmpich2$\backslash$summary$\backslash$index.html}.

\subsection{Testing a built mpich2 directory}
Explain how to run \texttt{testmpich2.wsf} if you have the mpich2 source tree on a machine and you 
have already built all of mpich2.

Here is a sample batch file to test mpich2 that has already been built in c:$\backslash$mpich2:
\begin{verbatim}
testmpich2.wsf /mpich2:c:\mpich2 /make- /configure- /buildbatch
pushd testmpich2\buildMPICH
call mpich_cmds.bat
popd
pushd testmpich2\buildCPP
call cpp_cmds.bat
popd
pushd testmpich2\buildINTEL
call intel_cmds.bat
popd
pushd testmpich2\buildMPICH2
call mpich2_cmds.bat
popd
testmpich2.wsf /mpich2:c:\mpich2 /make- /configure- /summarize
\end{verbatim}

\subsection{Testing an existing installation}
Explain the use of testmpich2.wsf to test an existing installation, one that was installed
with the .msi distribution.

\section{Development issues}
This section describes development issues that are particular to the Windows build.

Whenever a .h.in file is created on the Unix side, winconfigure.wsf needs to be updated to
create the .h file from the .h.in file.  Copy and paste an existing section in 
winconfigure.wsf that already does this and rename the file names.

When new definitions are added to the .h.in files these definitions, usually in the form HAVE\_FOO 
or USE\_FOO, need to be added to the AddDefinitions function in winconfigure.wsf.  Simply add 
new cases to the big case statement as needed.  winconfigure.wsf warns you of definitions that 
are not in the case statement.

Whenever a @FOO@ substitution is added on the Unix side, winconfigure.wsf needs to be updated
to handle the substitution.  Find the ReplaceAts function in winconfigure.wsf and add the 
substitution to the big case statement.  winconfigure.wsf warns you of new substitutions that
have not been added to the case statement.

\section{Runtime environment}

This section describes the MPICH2 environment that is particular to Windows.

\subsection{User credentials}
mpiexec must have the user name and password to launch MPI applications in the context of
that user.  This information can be stored in a secure encrypted manner for each user on a
machine.  Run \texttt{mpiexec -register} to save your username and password.  Then mpiexec
will not prompt you for this information.

This is also true for a nightly build script.  The user context under which the script is 
run must have saved credentials so mpiexec doesn't prompt for them.  So scripts won't hang,
mpiexec provides a flag, \texttt{-noprompt}, that will cause mpiexec to print out errors in 
cases when it normally would prompt for user input.  This can also be specified in the 
environment with the variable MPIEXEC\_NOPROMPT.

You can also save more than one set of user credentials.  Add the option \texttt{-user n}
to the \texttt{-register}, \texttt{-remove}, \texttt{-validate}, and \texttt{mpiexec}
commands to specify a saved user credential other than the default.  The parameter \texttt{n}
is a non-zero positive number.  For example this will save credentials in slot 1:
\begin{verbatim}
mpiexec -register -user 1
\end{verbatim}
And this command will use the user 3 to launch a job:
\begin{verbatim}
mpiexec -user 3 -n 4 cpi.exe
\end{verbatim}

User credentials can also be specified in a file using the \texttt{-pwdfile filename}
option to mpiexec.  Put the username on the first line of the file and the password
on the second line.  If you choose this option you should make sure the file is
only readable by the current user.

\subsection{MPICH2 channel selection}
MPICH2 for Windows comes with multiple complete implementations of MPI.  These are called
channels and each build represents a different transport mechanism used to move MPI messages.
The default channel uses sockets for communication.  There are two channels that use only
shared memory.  There is a channel that uses both sockets and shared memory.  There is a
channel that uses Infiniband.  And there is a thread-safe version of the sockets channel.

The short names for the channels are: sock, shm, sshm, ssm, mt.

These channels can be selected at runtime with an environment variable: MPICH2\_CHANNEL.
The following is an example that uses the Infiniband channel instead of the default sockets
channel:

\begin{verbatim}
mpiexec -env MPICH2_CHANNEL ib -n 4 myapp.exe
or
mpiexec -channel ib -n 4 myapp.exe
\end{verbatim}

If you specify \texttt{auto} for the channel then mpiexec will automatically choose a
channel for you.
\begin{verbatim}
mpiexec -channel auto -n 4 myapp.exe
\end{verbatim}
The rules are:
\begin{enumerate}
\item If numprocs is less than 8 on one machine, use the shm channel
\item If numprocs is greater than 7 on one machine, use the sshm channel
\item If running on multiple machines, use the ssm channel.  This channel can be changed 
using winconfigure.
\end{enumerate}

\subsection{MPI apps with GUI}
Many users on Windows machines want to build GUI apps that are also MPI applications.  This is
completely acceptable as long as the application follows the rules of MPI.  MPI\_Init must be
called before any other MPI function and it needs to be called soon after each process starts.
The processes must be started with mpiexec but they are not required to be console applications.

The one catch is that MPI applications are hidden from view so any Windows that a user 
application brings up will not be able to be seen.  mpiexec has an option to allow the MPI 
processes on the local machine to be able to bring up GUIs.  Add -localroot to the mpiexec
command to enable this capability.  But even with this option, all GUIs from processes on 
remote machines will be hidden.

So the only GUI application that MPICH2 cannot handle by default would be a video-wall type
application.  But this can be done by running smpd.exe by hand on each machine instead of
installing it as a service.  Log on to each machine and run ``\texttt{smpd.exe -stop}'' 
to stop the service and then run ``\texttt{smpd.exe -d 0}'' to start up the smpd again.  
As long as this process is running you will be able to run applications where every process 
is allowed to bring up GUIs.

\subsection{Security}
MPICH2 can use Microsoft's SSPI interface to launch processes without using any user 
passwords.  This is the most secure way to launch MPI jobs but it requires the machines to be 
configured in a certain way.
\begin{itemize}
\item All machines must be part of a Windows domain.
\item Each machine must have delegation enabled.
\item Each user that will run jobs must be allowed to use delegation.
\end{itemize}

If the machines are set up this way then an administrator can set up MPICH2 for passwordless
authentication.  On each node, a domain administrator needs to execute the following:
``\texttt{smpd -register\_spn}''.

Then a user can add the \texttt{-delegate} flag to their mpiexec commands and the job startup 
will be done without any passwords.  Example:
\begin{verbatim}
mpiexec -delegate -n 3 cpi.exe
\end{verbatim}

With SSPI enabled you can also control access to nodes with job objects.

First the nodes need to be set up so that only SSPI authentication is allowed.  An administrator
can run the following on each node:
\begin{enumerate}
\item \texttt{smpd.exe -set sspi\_protect yes}
\item \texttt{smpd.exe -set jobs\_only yes}
\item \texttt{smpd.exe -restart}
\end{enumerate}

These settings mean that authentication must be done through SSPI and mpiexec commands will only be
accepted for registered jobs.

To register jobs an administrator or a scheduler running with administrator privileges can execute
the following command:
\begin{verbatim}
mpiexec.exe -add_job <name> <domain\username> [-host <hostname>]
\end{verbatim}
This adds a job called ``name'' for the specified user on either the local or specified host.  Any
name can be used but it must not collide with another job with the same name on the same host.  The
command must be executed for each host that is to be allocated to the user.

Then when the job has finished or the allotted time has expired for the user to use the nodes the 
following command can be executed:
\begin{verbatim}
mpiexec.exe -remove_job <name> [-host <hostname>]
\end{verbatim}
This command removes the job from the local or specified host.  Any processes running on the host 
under the specified job name will be terminated by this command.

So \texttt{-add\_job} and \texttt{-remove\_job} can be used by a scheduler to create a window when a user is allowed to
start jobs on a set of nodes.

When the window is open the user can run jobs using the job name.  First the user must run:
\begin{verbatim}
mpiexec.exe -associate_job <name> [-host <hostname>]
\end{verbatim}
This will associate the user's token with the job object on the local or
specified host.  This must be done for all of the hosts allocated to the user.  Then the user can issue
mpiexec commands.  The mpiexec commands are of the usual format except they must contain one extra option -
``\texttt{-job <name>}''.  This job name must match the job allocated by the \texttt{-add\_job} command.  So a typical command
would look like this:
\begin{verbatim}
mpiexec.exe -job foo -machinefile hosts.txt -n 4 myapp.exe
\end{verbatim}
Multiple mpiexec commands can be issued until the \texttt{-remove\_job} command is issued.
This allows the users to issue multiple 
mpiexec commands and multiple MPI\_Comm\_spawn commands all using the same job name until the job is removed
from the nodes.

The rationale for the design where an adminstrator can create and destroy jobs but the user must first associate
the job with his own token before running jobs is so that the administrator does not need to know the user's
password.  In order for an administrator to do both the job allocation and association he would have to call
LogonUser with the user name and password for each user that submits a job request.

\subsection{Firewalls}
Windows comes with a default firewall that is usually turned on by default.
Firewalls block all TCP ports by default which renders MPICH2 applications inoperable
because the default communication mechanism used by MPICH2 are sockets on arbitrary
ports assigned by the operating system.  This can be solved in several ways:

\begin{itemize}
\item Turn off the firewall completely.
\item  MPICH2 applications can be limited to a range of TCP ports using the 
MPICH\_PORT\_RANGE environment variable.  If you set your firewall to allow the
same port range then MPICH2 applications will run.
\item Leave the Windows firewall on and allow exceptions for your MPICH2 applications.
This can be done through the Security Center module of the Windows Control 
Panel.  Click the Windows Firewall option in the Security Center to bring up
the properties page and select the Exceptions tab.  Here you can add each 
MPICH2 application to exempt.  Note that this exception includes the path to
the executable so if you move the executable you will have to exempt the new
location.  This solution obviously only will work for a small number of 
applications since managing a large list would be difficult.  Make sure you add
mpiexec.exe and the smpd.exe process manager to this exception list.
\end{itemize}

\subsection{MPIEXEC options}
This section describes all the options to mpiexec.exe

\begin{itemize}
\item \texttt{-n x} or \texttt{-np x}
Specify the number of processes to launch.
\item \texttt{-localonly x} or \texttt{-localonly}
Specify that the processes should only be launched on the local host.  This
option can replace the \texttt{-n x} option or be used in conjunction with it
when it is only a flag.
\item \texttt{-machinefile filename}
Use the specified file to get host names to launch processes on.  Hosts are 
selected from this file in a round robin fashion.  One host is specified per
line.  Extra options can be specified.  The number of desired processes to
launch on a specific host can be specified with a colon followed by a number
after the host name: \texttt{hostname:n}.  This is usefull for multi-CPU hosts.
If you want to specify the interface that should be used for MPI communication
to the host you can add the \texttt{-ifhn} flag.  Here is an example:
\texttt{hostname:4 -ifhn 192.168.1.100}
\item \texttt{-configfile filename}
Use the specified job configuration file to launch the job.  Each line in the
file represents a set of options just like you would enter them on the \texttt{mpiexec}
command line.  The one difference is that there are no colons in the file.  The
colons are replaced by new-lines.
\item \texttt{-file filename}
Use the specified implementation specific job configuration file.  For Windows
this option is used to specify the old MPICH 1.2.5 configuration file format.
This is useful for users who have existing configuration files and want to upgrade
to MPICH2.
\item \texttt{-host hostname}
Specify that the processes should be launched on a specific host.
\item \texttt{-hosts n host1 host2 host3 ...}
Specify that the processes should be launched on a list of hosts.  This option
replaces the \texttt{-n x} option.
\item \texttt{-hosts n host1 m1 host2 m2 host3 m3 ...}
Specify that the processes should be launched on a list of hosts and how many
processes should be launched on each host.  The total number of processes 
launched is m1 + m2 + m3 + ... mn.
\item \texttt{-map drive:$\backslash$$\backslash$host$\backslash$share}
Specify a network mapped drive to create on the hosts before launching the 
processes.  The mapping will be removed when the processes exit.  This option
can be specified multiple times.
\item \texttt{-dir drive:$\backslash$my$\backslash$working$\backslash$directory}
Specify the working directory for the processes.
\item \texttt{-wdir drive:$\backslash$my$\backslash$working$\backslash$directory}
\texttt{-wdir} and \texttt{-dir} are synonyms.
\item \texttt{-env variable value}
Specify an environment variable and its value to set in the processes' environments.
This option can be specified multiple times.
\item \texttt{-genvlist a,b,c,d...}
Specify a list of environment variables to taken from the environment local to mpiexec and propagated to the launched processes.
\item \texttt{-logon}
Prompt for user credentials to launch the job under.
\item \texttt{-pwdfile filename}
Specify a file to read the user name and password from.  The user name should be
on the first line and the password on the second line.
\item \texttt{-exitcodes}
Specify that the exit code of each process should be printed to stdout as each
processes exits.
\item \texttt{-noprompt}
Prevent mpiexec for prompting for information.  If user credentials are needed
to launch the processes mpiexec usually prompts for this information but this
flag causes an error to be printed out instead.
\item \texttt{-priority class[:level]}
Specify the priority class and optionally the thread priority of the processes
to be launched.  The class can be 0,1,2,3, or 4 corresponding to idle, below,
normal, above, and high.  The level can be 0,1,2,3,4, or 5 corresponding to
idle, lowest, below, normal, above, highest.  The default is 2:3.
\item \texttt{-localroot}
Specify that the root process should be launched on the local machine directly
from mpiexec bypassing the smpd process manager.  This is useful for applications
that want to create windows from the root process that are visible to the interactive
user.  The smpd process manager creates processes in a hidden service desktop
where you cannot interact with any GUI.
\item \texttt{-channel channel\_name}
This option is only available under Windows and allows the user to select which
channel implementation of MPICH2 to select at runtime.  The current channels
supported are sock, mt, ssm, shm, and sshm.  These represent the sockets, 
multi-threaded sockets, sockets plus shared memory, shared memory, and scalable
shared memory channels.  The shared memory channels only work on one node.  
The sockets, multi-threaded sockets, and sockets plus shared memory channels 
work on multiple nodes.  There are also profiled versions of the channels that
produce RLOG files for each process when selected.  They are named p, mtp, ssmp,
shmp, and sshmp.  See the section on channel selection for additional information.
\item \texttt{-log}
This option is a short cut to selecting the MPE wrapper library to log the MPI
application.  When the job finishes there will be a .clog2 file created that
can be viewed in Jumpshot.
\item \texttt{-l}
This flag causes mpiexec to prefix output to stdout and stderr with the rank of
the process that produced the output.  (This option is the lower-case L not the
number one)
\item \texttt{-port port}
Specify the port where the smpd process manager is listening.
\item \texttt{-p port}
Short version of the \texttt{-port} option.
\item \texttt{-smpdfile filename}
Specify the location of the smpd configuration file.  The default is \texttt{\~/.smpd}.
This is a Unix only option.  Under Windows the settings are stored in the Windows Registry.
\item \texttt{-phrase passphrase}
Specify the passphrase used to authenticate with the smpd process managers.
\item \texttt{-path search\_path}
Specify the search path used to locate executables.  Separate multiple paths with semicolons.
The path can be mixed when using both Windows and Linux machines.  For example:
\texttt{-path c:$\backslash$temp;/home/user} is a valid search path.
\item \texttt{-register [-user n]}
Encrypt a user name and password into the Windows registry so that it can be 
automatically retrieved by mpiexec to launch processes with.  If you specify
a user index then you can save more than one set of credentials.  The index
should be a positive non-zero number and does not need to be consecutive.
\item \texttt{-remove [-user n]}
Remove the encrypted credential data from the Registry.  If multiple entries are
saved then use the \texttt{-user} option to specify which entry to remove.
\texttt{-user all} can be specified to delete all entries.
\item \texttt{-validate [-user n] [-host hostname]}
Validate that the saved credentials can be used to launch a process on the local
or specified host.  If more that one credentials has been saved then the 
\texttt{-user} option can be used to select which user credentials to use.
\item \texttt{-user n}
Specify which encrypted credentials to retrieve from the Registry.  The corresponding
entry must have been previously saved using the \texttt{-register -user n} option.
\item \texttt{-timeout seconds}
Specify the maximum number of seconds the job is allowed to run.  At the end of
the timeout period, if the job has not already exited then all processes will
be killed.
\item \texttt{-plaintext}
Specify that user credentials should go over the wire un-encrypted.  This is
required if both Linux and Windows machines are used in the same job because
the Linux machines cannot encrypt and decrypt the data created by the Windows
machines.
\item \texttt{-delegate}
Specify that you want to use passwordless SSPI delegation to launch processes.
The machines must be configured to use SSPI as described in the section on 
security.
\item \texttt{-impersonate}
Specify that you want to use passwordless SSPI impersonation to launch processes.
This will create processes on the remote machines with limited access tokens.
They wil not be able to open files on remote machines or access mapped network
drives.
\item \texttt{-add\_job job\_name domain$\backslash$user [-host hostname]}
Create a job object on the local or specified host for the specified user.
Administrator privileges are required to execute this command.
\item \texttt{-remove\_job job\_name [-host hostname]}
Remove a job object on the local or specified host.  Any processes running under
this job will be terminated.  Administrator privileges are required to execute
this command.
\item \texttt{-associate\_job job\_name [-host hostname]}
Associate the current user token with the specified job on the local or specified
host.  The current user must match the user specifed by the \texttt{-add\_job job\_name username}
command.
\item \texttt{-job job\_name}
Specify that the processes should be launched under the specifed job object.
This can only be used after successful calls to \texttt{-add\_job} and \texttt{-associate\_job}.
\item \texttt{-whoami}
Print out the current user name in the format that mpiexec and smpd expect it to be.
This is useful for users who use a screen name that is different from their user
name.
\item \texttt{-hide\_console}
Detach from the console so that no command prompt window will appear and consequently
not output will be seen.
\item \texttt{-quiet\_abort}
Use this flag to prevent extensive abort messages to appear.  Instead the job 
will simply exit with minimal error output.
\item \texttt{-rsh or -ssh}
Use the remote shell command to execute the processes in the job instead of
using the smpd process manager.  The default command is ``\texttt{ssh -x}'' no
matter whether \texttt{-rsh} or \texttt{-ssh} is used.  If this is the only
flag specified then an smpd process manager must be running on the local host
where \texttt{mpiexec} is executed.  \texttt{mpiexec} contacts the local smpd process to start
a PMI service required by the MPI job and then starts the processes using the 
remote shell command.  On the target machines the application ``\texttt{env}''
must be available since it is used to set the appropriate environment variables
and then start the application.  The remote shell command can be changed using
the environment variable MPIEXEC\_RSH.  Any command can be used that takes a 
host name and then everything after that as the user command to be launched.
\item \texttt{-nopm}
This flag is used in conjunction with the \texttt{-rsh} flag.  With this flag
specified there need not be any smpd process manager running on any of the nodes
used in the job.  \texttt{mpiexec} provides the PMI interface and the remote
shell command is used to start the processes.  Using these flags allows jobs to
be started without any process managers running but the MPI-2 dynamic process
functions like MPI\_Comm\_spawn are consequently not available.
\item \texttt{-verbose}
Output trace data for mpiexec.  Only useful for debugging.
\item \texttt{-pmi\_server num\_processes or -pmiserver num\_processes}
This option specified by itself connects to the local smpd process manager and
starts a PMI service.  This service is used by MPICH2 processes to communicate
connection information to each other.  This option is only good for a single
MPICH2 job.  The input parameter is the number of processes in the job.
\texttt{mpiexec} immediately outputs three lines of data.  The first line is the
host name.  The second line is the port it is listening on and the third line
is the name of the PMI KVS.  A process manager that can set environment variables
and launch processes but does not implement the PMI service can use this option
to start jobs.  Along with the other PMI environment variables the process 
manager must set PMI\_HOST to the host name provided, PMI\_PORT to the port
provided and PMI\_KVS and PMI\_DOMAIN to the KVS name provided.  It is the 
responsibility of the process manager to set the other environment variables
correctly like PMI\_RANK and PMI\_SIZE.  See the document on the smpd PMI implementation
for a complete list of the environment variables.  When the job is finished the
PMI server will exit.  This option can be executed in separate command simultaneously
so that multiple jobs can be executed at the same time.
\end{itemize}

\subsection{SMPD process manager options}
This section describes some of the options for the smpd process manager.

smpd.exe runs as a service under Windows.  This is required so that it can start
processes under multiple user credentials.  Only services have the privileges
necessary to log on users and start processes for them.  Since this is a privileged
operation administrator rights are required to install the smpd service.  This is
what the default installer package does.

But smpd can be run in other ways for debugging or single user use.

If you have smpd.exe installed first execute \texttt{smpd.exe -stop} to stop the
service.

Then you can run it by hand for single user mode or for debugging.  The flag for
debugging single user mode is \texttt{-d debug\_output\_level}.

If you run it like this you will get full trace output:
\begin{verbatim}
smpd.exe -d
\end{verbatim}

If you run it like this you will get no output except for errors:
\begin{verbatim}
smpd.exe -d 0
\end{verbatim}

Here are all the options to smpd.exe:
\begin{itemize}
\item \texttt{-install or -regserver}
Install the smpd service.  Requires administrator privileges.
\item \texttt{-remove or -uninstall or -unregserver}
Uninstall the smpd service.  Requires administrator privileges.
\item \texttt{-start}
Start the smpd service.  Requires administrator privileges.
\item \texttt{-stop}
Stop the smpd service.  Requires administrator privileges.
\item \texttt{-restart}
Stop and restart the smpd service.  Requires administrator privileges.
\item \texttt{-register\_spn}
Register the Service Prinicipal Name for the smpd service of the local machine
on the domain controller.  Requires DOMAIN administrator privileges.  This is
used in conjunction with passwordless SSPI authentication described in the
section on security.
\item \texttt{-remove\_spn}
Remove the Service Prinicipal Name from the domain controller for the smpd service
of the local machine.  Requires DOMAIN administrator privileges.
\item \texttt{-traceon filename [hostA hostB ...]}
Turn on the trace logging of the smpd service on the local or specified hosts
and set the output to the specified file.  The file location must be available
on the local drive of each of the hosts.  It cannot be located on a remote 
machine.
\item \texttt{-traceoff [hostA hostB ...]}
Turn off the trace logging of the smpd service on the local or specified hosts.
\item \texttt{-port n}
Listen on the specified port number.  If this option is not specified then smpd
listens on the default port (8676).
\item \texttt{-anyport}
Listen on any port assigned by the OS.  smpd immediately prints out the port that
it has been assigned.
\item \texttt{-phrase passphrase}
Use the specified passphrase to authenticate connections to the smpd either by
mpiexec or another smpd process.
\item \texttt{-getphrase}
Prompt the user to input the passphrase.  This is useful if you don't want to 
specify the phrase on the command line.
\item \texttt{-noprompt}
Don't prompt the user for input.  If there is missing information, print an error
and exit.
\item \texttt{-set option value}
Set the smpd option to the specified value.  For example, \texttt{smpd -set logfile c:$\backslash$temp$\backslash$smpd.log} will set the log file to the 
specified file name.  \texttt{smpd -set log yes} will turn trace logging on and
\texttt{smpd -set log no} will turn it off.
\item \texttt{-get option}
Print out the value of the specified smpd option.
\item \texttt{-hosts}
Print the hosts that mpiexec and this smpd will use to launch processes on.  
If the list is empty then processes will be launched on the local host only.
\item \texttt{-sethosts hostA hostB ...}
Set the hosts option to a list of hosts that mpiexec and smpd will use to launch
processes on.
\item \texttt{-d [level] or -debug [level]}
Start the smpd in debug or single user mode with the optionally specified amount
of output.  For example, \texttt{smpd -d} will start the smpd will lots of trace
output and \texttt{smpd -d 0} will start the smpd with no output except for errors.
\item \texttt{-s}
Only available on Unix systems.  This option starts the smpd in single user daemon
mode for the current user.
\item \texttt{-smpdfile filename}
On Unix systems the smpd options are stored in a file that is readable only by
the current user (chmod 600).  This file stores the same information that would
be stored in the Windows registry like the port and passphrase.  The default
file is named \texttt{\~/.smpd} if this option is not specified.
\item \texttt{-shutdown}
Shutdown a running smpd that was started by \texttt{smpd -s} or \texttt{smpd -d}.
\item \texttt{-printprocs}
On a Windows machine you can run \texttt{smpd -printprocs} and it will print out
the processes started and stopped by smpd on the current host.  The format of the
output is \texttt{+/-pid cmd}.  Plus means a process was started and minus means
the process has exited.  The process id is specified next and then the rest of the line
is the command that was launched.
\item \texttt{-enum or -enumerate}
Print the smpd options set on the local host.
\item \texttt{-version}
Print the smpd version and exit.
\item \texttt{-status [-host hostname]}
Print the status of the smpd on the local or specified host.
\item \texttt{-help}
Print a brief summary of the options to smpd.
\end{itemize}

\subsection{Debugging jobs by starting them manually}
This section describes how to start a job by hand without the use of a process
manager so the job can be stepped through with a debugger.

You can launch an MPICH2 job by hand if you set the minimum required environment
variables for each process and then start the processes yourself (or in a debugger).

Here is a script that sets the environment variables so that a job can be started
 on the local machine:
The file is called \texttt{setmpi2.bat}
\begin{verbatim}
if ``%1'' == ``'' goto HELP
if ``%2'' == ``'' goto HELP
set PMI_ROOT_HOST=%COMPUTERNAME%
set PMI_ROOT_PORT=9222
set PMI_ROOT_LOCAL=1
set PMI_RANK=%1
set PMI_SIZE=%2
set PMI_KVS=mpich2
goto DONE
:HELP
REM usage: setmpi2 rank size
:DONE
\end{verbatim}

For example, to debug a two process job bring up two separate command prompts.
In the first prompt execute \texttt{setmpi2.bat 0 2} and in the second prompt
execute \texttt{setmpi2.bat 1 2}.  Then run your application always starting
the root process first.  The root process must call MPI\_Init before any of the
other processes because it is the process that listens on the port specifed by
the environment variable PMI\_ROOT\_PORT.  Simply execute \texttt{myapp.exe} from
each command prompt to run your job.  Or better yet run each process in a debugger.
If you have the Microsoft developer studio installed you can run the following
from each command prompt: \texttt{devenv.exe myapp.exe}.  This will bring up a
debugger for each process.  Then you can step through each process and debug it.
Remember that the first process must call MPI\_Init before any of the rest of the
processes do.  You can restart the processes at any time as long as you restart
all of them.

The script can be modified to launch on multiple hosts by changing the line:
\begin{verbatim}
set PMI_ROOT_HOST=%COMPUTERNAME%
\end{verbatim}
to set the variable to the hostname where the root process will be started instead
of the local host name.

The limitation of this method of starting processes is that MPI-2 spawning operations
are not supported.  If your application calls MPI\_Comm\_spawn it will produce
an error.

\subsection{Environment variables}
This section describes the environment variables used by MPICH2 and smpd.

\begin{itemize}
\item \texttt{MPICH\_ABORT\_ON\_ERROR}
Call abort() when an error happens instead of returning an error and calling MPID\_Abort. useful for unix where calling abort() creates a core file.
\item \texttt{MPICH\_PRINT\_ERROR\_STACK}
Print the entire error stack when an error occurs (currently this is the default)
\item \texttt{MPICH\_CHOP\_ERROR\_STACK}
Split the error stack output at the character position specified.  A value of 79
would cause carriage returns to be inserted after the 79th character.
\item \texttt{MPICH\_WARNINGS}
Print runtime warnings (unmatched messages at MPI\_Finalize, unreleased resources, etc)
\item \texttt{MPICH\_SOCKET\_BUFFER\_SIZE}
socket buffer size
\item \texttt{MPICH\_SOCKET\_RBUFFER\_SIZE}
socket receive buffer size
\item \texttt{MPICH\_SOCKET\_SBUFFER\_SIZE}
socket send buffer size
\item \texttt{MPICH\_SOCKET\_NUM\_PREPOSTED\_ACCEPTS}
number of accepts posted for MPIDU\_Sock\_listen
\item \texttt{MPICH\_PORT\_RANGE}
Range of ports to use for sockets: min..max or min,max
\item \texttt{MPICH\_INTERFACE\_HOSTNAME}
hostname to use to connect sockets
\item \texttt{MPICH\_NETMASK}
bitmask to select an ip subnet: ip/numbits, ie 192.0.0.0/8
\item \texttt{MPIEXEC\_TIMEOUT}
job timeout in seconds
\item \texttt{MPIEXEC\_LOCALONLY}
launch job processes on the local machine only
\item \texttt{MPIEXEC\_NOPROMPT}
Don't prompt for user input for missing information, print an error instead.
\item \texttt{MPIEXEC\_SMPD\_PORT}
Connect to smpd on the specified port.

The following two only affect mpiexec for smpd if -rsh is on the command line:
\item \texttt{MPIEXEC\_RSH}
rsh command to use, default is ``ssh -x''
\item \texttt{MPIEXEC\_RSH\_NO\_ESCAPE}
create an rsh command compatible with cygwin's ssh
\item \texttt{MPICH\_SPN}
Service Principal Name used for passwordless authentication
\item \texttt{SMPD\_DBG\_OUTPUT}
Print debugging output
\item \texttt{SMPD\_DBG\_LOG\_FILENAME}
name of logfile to send output to
\item \texttt{SMPD\_MAX\_LOG\_FILE\_SIZE}
maximum number of bytes the logfile can grow to before it is truncated
\item \texttt{MPICH\_DBG\_OUTPUT}
stdout, memlog or file.  determines where debugging output goes
\item \texttt{MPI\_DLL\_NAME}
name of the dll that contains the MPI and PMPI interfaces
\item \texttt{MPICH2\_CHANNEL}
short name of the channel used to create the full name of the MPI dll (ie. ib becomes mpich2ib.dll)
\item \texttt{MPI\_WRAP\_DLL\_NAME}
name of the dll that contains only the MPI interface, not the PMPI interface
\item \texttt{MPICH\_TRMEM\_INITZERO}
used by the memory tracing package
\item \texttt{MPICH\_TRMEM\_VALIDATE}
used by the memory tracing package
\item \texttt{MPITEST\_DEBUG}
used by the test suite
\item \texttt{MPITEST\_VERBOSE}
used by the test suite
\item \texttt{PATH}
used by smpd to search for executables under Unix.
\end{itemize}

SMPD options specified on the command line can also be specified in the environment
by prefixing \texttt{SMPD\_OPTION\_} to the option name and saving it as an
environment variable.
\begin{itemize}
\item \texttt{SMPD\_OPTION\_app\_path}
\item \texttt{SMPD\_OPTION\_logfile}
\item \texttt{SMPD\_OPTION\_nocache}
\item \texttt{SMPD\_OPTION\_phrase}
\item \texttt{SMPD\_OPTION\_sspi\_protect}
\item \texttt{SMPD\_OPTION\_max\_logfile\_size}
\item \texttt{SMPD\_OPTION\_plaintext}
\item \texttt{SMPD\_OPTION\_port}
\item \texttt{SMPD\_OPTION\_timeout}
\item \texttt{SMPD\_OPTION\_exitcodes}
\item \texttt{SMPD\_OPTION\_priority}
\item \texttt{SMPD\_OPTION\_localonly}
\item \texttt{SMPD\_OPTION\_noprompt}
\item \texttt{SMPD\_OPTION\_channel}
\item \texttt{SMPD\_OPTION\_hosts}
\item \texttt{SMPD\_OPTION\_delegate}
\item \texttt{SMPD\_OPTION\_internode\_channel}
\item \texttt{SMPD\_OPTION\_log}
\item \texttt{SMPD\_OPTION\_no\_dynamic\_hosts}
\end{itemize}

Variables to control debugging output when enabled:
\begin{itemize}
\item \texttt{MPICH\_DBG}
\item \texttt{MPICH\_DBG\_CLASS}
\item \texttt{MPICH\_DBG\_FILENAME}
\item \texttt{MPICH\_DBG\_LEVEL}
\item \texttt{MPICH\_DBG\_OUTPUT}
\item \texttt{MPICH\_DBG\_RANK}
\item \texttt{MPICH\_DEBUG\_ITEM}
\end{itemize}

The following variables affect the MPE logging library:
\begin{itemize}
\item \texttt{MPE\_LOGFILE\_PREFIX}
name of the clog file to create without the extension
\item \texttt{MPE\_DELETE\_LOCALFILE}
true,false - delete or not the local clog file
\item \texttt{MPE\_LOG\_OVERHEAD}
I think this one adds an event to the clog files representing the time it takes to write a clog buffer to disk
\item \texttt{CLOG\_BLOCK\_SIZE}
number of bytes in a clog block
\item \texttt{CLOG\_BUFFERED\_BLOCKS}
number of blocks
\item \texttt{MPE\_CLOCKS\_SYNC}
yes/no - synchronize clocks

directories to store temporary files:
\item \texttt{MPE\_TMPDIR}
\item \texttt{TMPDIR}
\item \texttt{TMP}
\item \texttt{TEMP}
\end{itemize}

PMI environment variables created by smpd are described in the smpd documentation:
\begin{itemize}
\item \texttt{PMI\_DLL\_NAME}
name of the PMI dll to load (replaces the default smpd functions)
\item \texttt{PMI\_NAMEPUB\_KVS}
name of the key-val-space where MPI service names are stored for MPI\_Lookup\_name()
\item \texttt{PMI\_ROOT\_HOST}
\item \texttt{PMI\_ROOT\_PORT}
\item \texttt{PMI\_ROOT\_LOCAL}
\item \texttt{PMI\_SPAWN}
\item \texttt{PMI\_KVS}
\item \texttt{PMI\_DOMAIN}
\item \texttt{PMI\_RANK}
\item \texttt{PMI\_SIZE}
\item \texttt{PMI\_CLIQUE}
\item \texttt{PMI\_APPNUM}
\item \texttt{PMI\_SMPD\_ID}
\item \texttt{PMI\_SMPD\_KEY}
\item \texttt{PMI\_SMPD\_FD}
\item \texttt{PMI\_HOST}
\item \texttt{PMI\_PORT}
\item \texttt{PMI\_APPNUM}
\end{itemize}

Used by the process managers other than smpd:
\begin{itemize}
\item \texttt{MPIEXEC\_DEBUG}
\item \texttt{MPIEXEC\_MACHINES\_PATH}
\item \texttt{MPIEXEC\_PORTRANGE}
\item \texttt{MPIEXEC\_PREFIX\_STDERR}
\item \texttt{MPIEXEC\_PREFIX\_STDOUT}
\item \texttt{MPIEXEC\_REMSHELL}
\item \texttt{MPIEXEC\_USE\_PORT}
\end{itemize}

\subsection{Compiling}
This section describes how to set up a project to compile an MPICH2 application
using MS Visual Studio 6.0 or MS Developer Studio .NET 2003.

\subsubsection{Visual Studio 6.0}
Visual C++ 6.0 cannot handle multiple functions with the same type signature 
that only differ in their return type.  So you must define \texttt{HAVE\_NO\_VARIABLE\_RETURN\_TYPE\_SUPPORT}
in your project.

\begin{enumerate}
\item Create a project and add your source files.

\item Bring up the settings for the project by hitting Alt F7.  Select the Preprocessor
Category from the C/C++ tab.  Enter \texttt{HAVE\_NO\_VARIABLE\_RETURN\_TYPE\_SUPPORT} into
the Preprocessor box.  Enter \texttt{C:$\backslash$Program Files$\backslash$MPICH2$\backslash$include}
into the ``Additional include directories'' box.

\item Select the Input Category from the Link tab.  Add \texttt{cxx.lib} and \texttt{mpi.lib} to
the Object/library modules box.  Add \texttt{C:$\backslash$Program Files$\backslash$MPICH2$\backslash$lib}
to the ``Additional library path'' box.

\item Compile your application.
\end{enumerate}

\subsubsection{Developer Studio .NET 2003}
For Developer Studio .NET 2003 or newer you can use the example projects provided
with the release as a guide to creating your own projects.

\begin{enumerate}
\item Create a project and add your source files.

\item Bring up the properties dialog for your project by right clicking the project
name and selecting Properties.

\item Navigate to Configuration Properties::C/C++::General
\item Add \texttt{C:$\backslash$Program Files$\backslash$MPICH2$\backslash$include}
to the ``Additional Include Directories'' box.

\item Navigate to Configuration Properties::Linker::General
\item Add \texttt{C:$\backslash$Program Files$\backslash$MPICH2$\backslash$lib}
to the ``Aditional Library Directories'' box.

\item Navigate to Configuration Properties::Linker::Input
\item Add \texttt{cxx.lib} and \texttt{mpi.lib} and \texttt{fmpich2.lib} to the ``Additional Dependencies'' box.  If your
application is a C application then it only needs \texttt{mpi.lib}.  If it is a C++ application then it
needs both \texttt{cxx.lib} and \texttt{mpi.lib}.  If it is a Fortran application then it only needs one of the \texttt{fmpich2[s,g].lib} libraries.
The fortran library comes in three flavors \texttt{fmpich2.lib}, \texttt{fmpich2s.lib} and \texttt{fmpich2s.lib}.  \texttt{fmpich2.lib}
contains all uppercase symbols and uses the C calling convention like this: \texttt{MPI\_INIT}.  \texttt{fmpich2s.lib}
contains all uppercase symbols and uses the stdcall calling convention like this: \texttt{MPI\_INIT@4}.
\texttt{fmpich2g.lib} contains all lowercase symbols with double underscores and the C calling convention
like this: \texttt{mpi\_init\_\_}.  Add the library that matches your Fortran compiler.

\item Compile your application.
\end{enumerate}

\end{document}
