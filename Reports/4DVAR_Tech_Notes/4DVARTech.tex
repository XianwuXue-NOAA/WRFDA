%%%%%%%%%%%%%%%%%%%%%%%%%%%%%%%%%%%%%%%%%%%%%%%%%%%%%%%%%%%%%%%%     
%%%%%%%%%%%%%%%%%%%%%%%%%%%%%%%%%%%%%%%%%%%%%%%%%%%%%%%%%%%%%%%%
%																															 %
%				USE THIS TEMPLATE, AMETSOC.STY, AND AMETSOC.BST				 %
%			        OR YOUR TEX FILES WILL NOT BE USED				  		 %
%																															 % 
%%%%%%%%%%%%%%%%%%%%%%%%%%%%%%%%%%%%%%%%%%%%%%%%%%%%%%%%%%%%%%%%
%%%%%%%%%%%%%%%%%%%%%%%%%%%%%%%%%%%%%%%%%%%%%%%%%%%%%%%%%%%%%%%%

%%%%%%%%%%%%%%%%%%%%%%%%%%%%%%%%%%%%%%%%%%%%%%%%%%%%%%%%%%%%%%%%%%%%%
% PREAMBLE
%%%%%%%%%%%%%%%%%%%%%%%%%%%%%%%%%%%%%%%%%%%%%%%%%%%%%%%%%%%%%%%%%%%%%
%
% The following two commands will generate a PDF that follows all the requirements for submission
% and peer review.  Uncomment these commands to generate this output (and comment out the two lines below.)
%
% DOUBLE SPACE VERSION FOR SUBMISSION TO THE AMS
\documentclass[12pt]{article}
\usepackage{ametsoc}
%
% The following two commands will generate a single space, double column paper that closely
% matches an AMS journal page.  Uncomment these commands to generate this output (and comment
% out the two lines above. FOR AUTHOR USE ONLY. PAPERS SUBMITTED IN THIS FORMAT WILL BE RETURNED
% TO THE AUTHOR for submission with the correct formatting.
%
% TWO COLUMN JOURNAL PAGE LAYOUT FOR AUTHOR USE ONLY
%\documentclass[10pt]{article}
%\usepackage{ametsoc2col}
\usepackage{subfigure}
%
%%%%%%%%%%%%%%%%%%%%%%%%%%%%%%%%%%%%%%%%%%%%%%%%%%%%%%%%%%%%%%%%%%%%%
% ABSTRACT
%
% Enter your Abstract here
%%%%%%%%%%%%%%%%%%%%%%%%%%%%%%%%%%%%%%%%%%%%%%%%%%%%%%%%%%%%%%%%%%%%%
\newcommand{\myabstract}{The Limited Area Model (LAM) forecasting problem is a lateral boundary 
condition problem in addition to the initial condition problem. For 
numerical weather prediction, the lateral boundary conditions are generally 
provided by a host model, for example a global numerical weather prediction 
model. One may argue that the available boundary conditions from the host 
model have to be accepted (without change) for usage in the LAM. However, 
one may use the observations close to the lateral boundaries and let these 
influence the initial conditions. In case the initial conditions are used as the 
lateral boundary conditions at the initial time, the subsequent forecast will 
also be influenced by these observations. Depending on the frequency of the 
updating of the lateral boundary conditions, the host model will gradually 
take over the definition of the lateral boundary conditions. With application of
 4D-Var for the LAM forecasting, one could do a bit more, since one 
may control the lateral boundary conditions during the period of the data 
assimilation window. This may be particularly important for assimilation of 
phenomena that are observed well inside the LAM domain during the later 
part of the data assimilation window, while being propagated through the 
lateral boundaries during the early part of the data assimilation window. 
If we do not control the lateral boundary conditions, observed information 
related to these phenomena may be lost.}
%
\begin{document}
%
%%%%%%%%%%%%%%%%%%%%%%%%%%%%%%%%%%%%%%%%%%%%%%%%%%%%%%%%%%%%%%%%%%%%%
% TITLE
%
% Enter your TITLE here
%%%%%%%%%%%%%%%%%%%%%%%%%%%%%%%%%%%%%%%%%%%%%%%%%%%%%%%%%%%%%%%%%%%%%
\title{\textbf{\large{A Description of the WRF 3/4D-Var}}}
%
% Author names, with corresponding author information. 
% [Update and move the \thanks{...} block as appropriate.]
%
\author{\textsc{Xin Zhang}
				\thanks{\textit{Corresponding author address:} 
				Dr. Xin Zhang, NCAR/MMM, P.O. Box 3000, 
				 Boulder, CO 80307. 
				\newline{E-mail: xinzhang@ucar.edu}}
             \textsc{and Xiang-Yu Huang}\\
\textit{\footnotesize{MMM, National Center for Atmospheric Research, Boulder, CO 80305}}
%\and 
%\centerline{\textsc{Nils Gustafsson}}\\% Add additional authors, different insitution
%\centerline{\textit{\footnotesize{Swedish Meteorological and Hydrological Institute, SE-60176 Norrk�oping, Sweden}}}
}
%
% The following block of code will handle the formatting of the title page depnding on whether
% we are formatting a double column (dc) author draft or a single column paper for submission.
% AUTHORS SHOULD SKIP OVER THIS... There is nothing to do in this section of code.
\ifthenelse{\boolean{dc}}
{
\twocolumn[
\begin{@twocolumnfalse}
\amstitle

% Start Abstract (Enter your Abstract above.  Do not enter any text here)
\begin{center}
\begin{minipage}{13.0cm}
\begin{abstract}
	\myabstract
	\newline
	\begin{center}
		\rule{38mm}{0.2mm}
	\end{center}
\end{abstract}
\end{minipage}
\end{center}
\end{@twocolumnfalse}
]
}
{
\amstitle
\begin{abstract}
\myabstract
\end{abstract}
\newpage
}
%%%%%%%%%%%%%%%%%%%%%%%%%%%%%%%%%%%%%%%%%%%%%%%%%%%%%%%%%%%%%%%%%%%%%
% MAIN BODY OF PAPER
%%%%%%%%%%%%%%%%%%%%%%%%%%%%%%%%%%%%%%%%%%%%%%%%%%%%%%%%%%%%%%%%%%%%%
%

\section{Basic formulations}

The WRF 4D-Var algorithm  takes the incremental 4D-Var formulation that is commonly used in operational systems. The incremental approach is designed to find the analysis increment that minimizes a cost function defined as a function of the analysis increment instead of the analysis itself. In the incremental 4D-Var, the tangent linear and adjoint models usually derived from a simplified forward model are used in the inner-loop minimization, while the evolution of the background is predicted with the full for ward model. 

Mathematically WRF 4D-Var minimizes a cost function $J$:
\begin{align}
J=J_b+J_o+J_c
\end{align}
which includes quadratic measure of distance to the background, observation, and balanced solution. The background cost function term $J_b$ is\\
Define:\\
$\mathbf{x}_k$ : Model state vector at $k^{th}$ time step;\\
$\mathbf{y}_k$ : Observation vector at $k^{th}$ time step;\\
$\bm{\mathsf{B}}$ : Background error covariance matrix;\\
$\bm{\mathsf{O}}$: Observational error covariance matrix;\\
$M_k$ : Nonlinear model at $k^{th}$ time step, $\mathbf{x}_k=M_k(\mathbf{x}_0)$;\\
$\bm{\mathsf{M}}_k$ : Tangent linear model at $k^{th}$ time step;\\
$\bm{\mathsf{M}}^T_k$ : Adjoint model at $k^{th}$ time step;\\
$H_k$: Observational operator at $k^{th}$ time step;\\
$\bm{\mathsf{H}}_k$: Tangent linear observational operator;\\
$\bm{\mathsf{H}}^T_k$: Adjoint observational operator.\\
\begin{align}
\begin{split}
J_b &=-\frac{1}{2}(\mathbf{x}^n-\mathbf{x}^b)^T\bm{\mathsf{B}}^{-1}(\mathbf{x}^n-\mathbf{x}^b) \\
&=-\frac{1}{2}[(\mathbf{x}^n-\mathbf{x}^{n-1})+(\mathbf{x}^{n-1}-\mathbf{x}^b)]^T\bm{\mathsf{B}}^{-1}[(\mathbf{x}^n-\mathbf{x}^{n-1})+(\mathbf{x}^{n-1}-\mathbf{x}^b)] \\
&=-\frac{1}{2}[(\mathbf{x}^n-\mathbf{x}^{n-1})+\sum_{i=1}^{n-1}{(\mathbf{x}^{i}-\mathbf{x}^{i-1})}]^T\bm{\mathsf{B}}^{-1}[(\mathbf{x}^n-\mathbf{x}^{n-1})+\sum_{i=1}^{n-1}(\mathbf{x}^{i}-\mathbf{x}^{i-1})]
\end{split}
\end{align}
where superscripts $-1$ and $T$ denote inverse and adjoint of a matrix or a linear operator. Here $\bm{\mathsf{B}}$ is the background error covariance matrix, which is typically climatological estimates, but it may also be derived from prior or ensemble-based flow-dependent estimates or, with slightly different formulation, combination of both climatological and ensemble estimates.  In a slight abuse of notation,
\begin{align}
\begin{split}
J_o &=-\frac{1}{2}\sum_{k=1}^N[(H_k\mathbf{x}_k-\mathbf{y}_k)^T\bm{\mathsf{O}}^{-1}(H_k\mathbf{x}_k-\mathbf{y}_k)]  \\
& =-\frac{1}{2}\sum_{k=1}^N[(H_kM_k(\mathbf{x}_0)-\mathbf{y}_k)^T\bm{\mathsf{O}}^{-1}(H_kM_k(\mathbf{x}_0)-\mathbf{y}_k)]
\end{split}
\end{align}
\\*
Introduce: \\
$\mathbf{v}=\delta\mathbf{x}_0=\mathbf{x}_0-\mathbf{x}_b$;\\
$\mathbf{U}=\mathbf{B}^{1/2}$;\\
$\mathbf{v}=\mathbf{U}^{-1}\delta\mathbf{x}_0$;\\
$S_{v-w}$: Var space to WRF space transform; \\
$\mathbf{S}_{v-w}$: tangent linear; \\
$\mathbf{S}^T_{v-w}$: adjoint.\\
$S_{w-v}$: WRF space to VAR space transform; \\
$\mathbf{S}_{w-v}$: tangent linear; \\
$\mathbf{S}^T_{w-v}$: adjoint
\\*
\begin{align}
J_b(\mathbf{v})&=-\frac{1}{2}[\mathbf{v}^T\mathbf{v}]
\end{align}
\begin{align}
\begin{split}
J_o(\mathbf{v})& =-\frac{1}{2}[(H_kS_{w-v}M_k(\mathbf{x}_0)-\mathbf{y}_k)^T\mathbf{O}^{-1}(H_kS_{w-v}M_k(\mathbf{x}_0)-\mathbf{y}_k)]  \\
& =-\frac{1}{2}[(\mathbf{H}_k\mathbf{S}_{w-v}\mathbf{M}_k(\mathbf{x}_b)+H_kS_{w-v}M_kS_{v-w}\mathbf{Uv}-\mathbf{y}_k)^T\mathbf{O}^{-1}  \\
&  \qquad (\mathbf{H}_k\mathbf{S}_{w-v}\mathbf{M}_k(\mathbf{x}_b)+H_kS_{w-v}M_kS_{v-w}\mathbf{Uv}-\mathbf{y}_k)]  \\
& =-\frac{1}{2}[(\mathbf{H}_k\mathbf{S}_{w-v}\mathbf{M}_k(\mathbf{x}_b)-\mathbf{y}_k+H_kS_{w-v}M_kS_{v-w}\mathbf{Uv})^T\mathbf{O}^{-1}  \\
&  \qquad (\mathbf{H}_k\mathbf{S}_{w-v}\mathbf{M}_k(\mathbf{x}_b)-\mathbf{y}_k+H_kS_{w-v}M_kS_{v-w}\mathbf{Uv})]
\end{split}
\end{align}
\\*
Gradient:
\begin{align}
J^{'}_b(\mathbf{v}) & =-\mathbf{v}
\end{align}
\begin{align}
\begin{split}
J^{'}_o(\mathbf{v}) & =-\sum_{k=K}^1[\mathbf{U}^T\mathbf{S}^T_{v-w}\mathbf{M}^T_k\mathbf{S}^T_{w-v}\mathbf{H}^T_k\mathbf{O}^{-1}(\mathbf{H}_k\mathbf{S}_{w-v}\mathbf{M}_k\mathbf{S}_{v-w}\mathbf{Uv}\\
& \qquad +H_kS_{w-v}M_k(\mathbf{x}_b)-\mathbf{y}_k)]\\
& =-\mathbf{U}^T\mathbf{S}^T_{v-w}\sum_{k=K}^1[\mathbf{M}^T_k\mathbf{S}^T_{w-v}\mathbf{H}^T_k\mathbf{O}^{-1}(\mathbf{H}_k\mathbf{S}_{w-v}\mathbf{M}_k\mathbf{S}_{v-w}\mathbf{Uv}\\
& \qquad +H_kS_{w-v}M_k(\mathbf{x}_b)-\mathbf{y}_k)]\\
\end{split}
\end{align}
\begin{align}
J^{'}_c(\mathbf{v})&=
\end{align}

\section{Variable index}
\begin{tabular}{|l|c|c|}
\hline
\textbf{Name} & \textbf{Type} & \textbf{Description}\\
\hline
\hline
$be$ & $be\_type$ & Background Error $\mathbf{O}$\\
\hline
$cvt, cv$ & vector & Outer loop Control variables\\
\hline
$ghat, fhat$  & vector  & Gradient $J^{'}_o(\mathbf{v})$\\
\hline
$iv$ &  $iv\_type$ & Innovation $H_kS_{w-v}M_k(\mathbf{x}_b)-\mathbf{y}_k$\\
\hline
$j$ & $j\_type$ & Cost function\\
\hline
$jo$ & $jo\_type$ & Cost function $J_o(\mathbf{v})$\\
\hline
$jo\_grad\_x, grid\%xa$ & $x\_type$ & $\mathbf{H}_k^T\mathbf{O}^{-1}(\mathbf{H}_k\mathbf{S}_{w-v}\mathbf{M}_k\mathbf{S}_{v-w}\mathbf{Uv}+H_kS_{w-v}M_k(\mathbf{x}_b)-\mathbf{y}_k)$\\
\hline
$jo\_grad\_y$ & $y\_type$ &$\mathbf{O}^{-1}(\mathbf{H}_k\mathbf{S}_{w-v}\mathbf{M}_k\mathbf{S}_{v-w}\mathbf{Uv}+H_kS_{w-v}M_k(\mathbf{x}_b)-\mathbf{y}_k)$\\
\hline
$phat$ & vector & Reverse gradient: $-J^{'}_o(\mathbf{v})$\\
\hline
$re$ & $y\_type$ & Residual $\mathbf{H}_k\mathbf{S}_{w-v}\mathbf{M}_k\mathbf{S}_{v-w}\mathbf{Uv}+H_kS_{w-v}M_k(\mathbf{x}_b)-\mathbf{y}_k$\\
\hline
$vv$ & & \\
\hline
$vp$ & & \\
\hline
$xhat$ & & Inner loop control variables: $\mathbf{v}$\\
\hline
$y$ &  $y\_type$ &$\mathbf{H}_k\mathbf{S}_{w-v}\mathbf{M}_k\mathbf{S}_{v-w}\mathbf{Uv}$\\
\hline
\end{tabular}

\section{Subroutine index}

\subsection*{da\_allocate\_y:}

\subsection*{da\_calculate\_gradj:}
\[
U^TS^T_{w-v}H_k^TO^{-1}(H_kS_{w-v}M_kS_{v-w}Uv+H_kS_{w-v}M_k(x_b)-y_k)
\] 

\begin{verbatim}
da_calculate_grady :
\end{verbatim} 
\[
\mathbf{O}^{-1}(\mathbf{H}_k\mathbf{S}_{w-v}\mathbf{M}_k\mathbf{S}_{v-w}\mathbf{Uv}+H_kS_{w-v}M_k(x_b)-y_k)
\] 

\begin{verbatim}
da_calculate_residual :
 \end{verbatim} 
\[
H_kS_{w-v}M_kS_{v-w}Uv+H_kS_{w-v}M_k(x_b)-y_k
\] 
 
 
\begin{verbatim}
da_calculate_j :
 \end{verbatim}
 
\begin{verbatim}
da_jo_and_grady :
 \end{verbatim} 
\[
O^{-1}(H_kS_{w-v}M_kS_{v-w}Uv+H_kS_{w-v}M_k(x_b)-y_k)
\] 
\[
\frac{1}{2}(H_kS_{w-v}M_kS_{v-w}Uv+H_kS_{w-v}M_k(x_b)-y_k)^TO^{-1}(H_kS_{w-v}M_kS_{v-w}Uv+H_kS_{w-v}M_k(x_b)-y_k)
\] 

\begin{verbatim}
da_transform_xtoxa:
\end{verbatim} 

\begin{verbatim}
da_transform_xtoxa_adj:
\end{verbatim} 
input: $grid$\\
output: $grid\%xa$\\

\subsection*{da\_transform\_vtox:}
input: $\mathbf{cv} (\mathbf{xhat})$\\
output: $grid\%xa)$\\
\[
\mathbf{U}
\]

\begin{verbatim}
da_transform_vtox_adj:
\end{verbatim} 
\[
\mathbf{U}_T
\]
 
\begin{verbatim}
da_transform_xtoy:
\end{verbatim} 
input: $grid$\\
output: $grid\%xa$\\
\[
\mathbf{H}_k
\]

\begin{verbatim}
da_transform_xtoy_adj:
 \end{verbatim}
input: $iv, jo\_grad\_y$\\
output: $jo\_grad\_x (grid\%xa)$\\
\[
\mathbf{H}_k^T\mathbf{O}^{-1}(\mathbf{H}_k\mathbf{S}_{w-v}\mathbf{M}_k\mathbf{S}_{v-w}\mathbf{Uv}+H_kS_{w-v}M_k(\mathbf{x}_b)-\mathbf{y}_k)
\]

\begin{verbatim}
da_transform_vtoy:
\end{verbatim} 
input: $xhat$\\
output: $y$
\[
output=\mathbf{H}_k\mathbf{S}_{w-v}\mathbf{M}_k\mathbf{S}_{v-w}\mathbf{U}input
\]

\begin{verbatim}
da_transform_vtoy_adj:
\end{verbatim} 
input: $y (jo\_grad\_y)$\\
output: $cv ,cv\_jl (grad\_jo, grad\_jl)$
\begin{align*}
output=-\mathbf{U}^T\mathbf{S}^T_{v-w}\sum_{k=K}^1[\mathbf{M}^T_k\mathbf{S}^T_{w-v}\mathbf{H}^T_kinput]
\end{align*}

\begin{verbatim}
da_transfer_wrftltoxa:
\end{verbatim} 
\[
\mathbf{S}_{w-v}
\] 

\begin{verbatim}
da_transfer_wrftltoxa_adj:
\end{verbatim} 
\[
\mathbf{S}^T_{w-v}
\] 

\begin{verbatim}
da_transfer_xatowrftl:
\end{verbatim} 
\[
\mathbf{S}_{v-w}
\] 

\begin{verbatim}
da_transfer_xatowrftl_adj:
\end{verbatim} 
\[
\mathbf{S}^T_{v-w}
\] 

\begin{verbatim}
da_transform_vtox:
\end{verbatim} 
\[
Uv
\] 

\begin{verbatim}
da_transform_vtox_adj:
\end{verbatim} 
\[
U^T
\] 

 
\end{document}