\documentclass[a4paper,12pt,titlepage]{article}
\usepackage{authblk}
\usepackage{graphicx}

\author{Xin Zhang}
\author{Xiang-Yu Huang}
\affil{ MMM, National Center for Atmospheric Research,\\Boulder,CO USA 80305}
\author{Nils Gustafsson}
\affil{Swedish Meteorological and Hydrological Institute,\\SE-60176 Norrk�oping, Sweden}

\title{Control of lateral boundary conditions in WRF 4D-Var}

\date{\today}

%\pagestyle{headings}

\begin{document}

\maketitle

\begin{abstract}
The Limited Area Model (LAM) forecasting problem is a lateral boundary 
condition problem in addition to the initial condition problem. For 
numerical weather prediction, the lateral boundary conditions are generally 
provided by a host model, for example a global numerical weather prediction 
model. One may argue that the available boundary conditions from the host 
model have to be accepted (without change) for usage in the LAM. However, 
one may use the observations close to the lateral boundaries and let these 
influence the initial conditions. In case the initial conditions are used as the 
lateral boundary conditions at the initial time, the subsequent forecast will 
also be influenced by these observations. Depending on the frequency of the 
updating of the lateral boundary conditions, the host model will gradually 
take over the definition of the lateral boundary conditions. With application of
 4D-Var for the LAM forecasting, one could do a bit more, since one 
may control the lateral boundary conditions during the period of the data 
assimilation window. This may be particularly important for assimilation of 
phenomena that are observed well inside the LAM domain during the later 
part of the data assimilation window, while being propagated through the 
lateral boundaries during the early part of the data assimilation window. 
If we do not control the lateral boundary conditions, observed information 
related to these phenomena may be lost. 
\end{abstract}

\tableofcontents

\newpage

\section{Background}
For the control of lateral boundary conditions in HIRLAM 4D-Var, we follow \cite{kawa} and we introduce the lateral boundary condition perturbations as a control variable $\delta{x_{lbc}}(t_K)$  at the end of the data assimilation window (time $t_K$).  For the lateral boundary increment $\delta{x_{lbc}}(t_0)$ at the start of the assimilation window (time $t_0$) we will use the initial condition increment $\delta{x}(t_0)$. For intermediate time-steps during the integration of the tangent linear model, we will obtain the lateral boundary conditions by the same linear time interpolation scheme that is used in the non-linear model. Once the lateral boundary conditions are defined, the same lateral boundary relaxation scheme that is used in the non-linear model \cite{Davis} can also be used in the tangent-linear model, and the adjoint of the lateral 
boundary relaxation is also well defined, see \cite{Gusta}. Note however, that while the lateral boundary conditions are input data to the tangent linear model the lateral boundary conditions are output data from 
the adjoint model. 

Again following \cite{kawa} we will introduce a cost function term $J_{lbc}$ that will measure the distance to the un-perturbed lateral boundary conditions used in the non-linear model 
\begin{equation}
J_{lbc}=(\delta{x_{lbc}}(t_K))^T~B_{lbc}^{-1}~\delta{x_{lbc}}(t_K)
\end{equation}
where $B_{lbc}$ represents the covariance of the lateral boundary condition 
errors. Lateral boundary conditions only need to be defined in the narrow 
lateral boundary relaxation zone, but obviously it would be difficult to specify the proper spatial scales and balances representing the lateral boundary condition errors over such a narrow domain. We have therefore made the 
choice to specify the lateral boundary control variable over the same total
 model domain as the initial condition control variable. Since both of 
these control variables represent forecast errors, albeit for different forecast 
lengths and possibly for different models, it is clear that their respective 
error covariance can be represented in a similar way. In a first trial to test 
the sensitivity of the assimilation to the control of the lateral boundary 
conditions, we will simply apply $B_{lbc} = B$. 

One potential problem is that the introduction of the lateral boundary 
condition constraint in the form described above, would worsen the conditioning of the 4D-Var minimization problem, since the lateral boundary condition errors at the end of the assimilation window would be strongly 
correlated with the initial condition errors at the start of the assimilation 
window, at least for the large scale and slowly varying phenomena that are 
important for the lateral boundary conditions. One simple pre-conditioning 
would be to subtract the lateral boundary conditions at the start of the 
assimilation window from the lateral boundary conditions at the end of 
the assimilation window, thus to treat the tendency of the lateral boundary 
conditions $(\frac{\partial{\delta{x_{lbc}}}}{\partial{t}})=\frac{\delta{x_{lbc}}(t_K)-\delta{x_{lbc}}(t_0)} {t_{K}-t_0}$ as the control variable. This approach would also require the covariance of forecast tendency errors, that possibly 
could be estimated by the NMC method from differences between forecast 
tendencies valid at the same time. 


\section{Formulation}
A similar control of lateral boundary condition as developed for HIRLAM 
4D-Var will be developed also for the WRF 4D-Var. As a starting point, the 
lateral boundary conditions at the end of the data assimilation window will
be used as the control variable. The main difference between the implemen- 
tation in WRF 4D-Var and HIRLAM 4D-Var is due to the differences between 
the lateral boundary relaxation schemes, where the WRF uses a boundary 
relaxation applied in a �nudging� of model tendencies, following \cite{Davis1977}
\begin{equation}
\frac{\partial{x}}{\partial{t}}=F_1(x_{lbc}-x)-F_2\Delta^2(x_{lbc}-x)
\end{equation}
where $\Delta^2$ is a 5-point smoothing operator,$ F_1$ and $F_2$ are (nudging) 
weighting coefficients depending on the distance to the lateral boundary, $x$ 
is any model state variable and $x_{lbc}$ is the corresponding boundary value 
provided by the host model. $x_{lbc}$ is speci�ed in the following form:
\begin{equation}
x_{lbc}(time=t)=x_{lbc}(time=t_0)+(t-t_0)\frac{\partial{x_{lbc}}}{\partial{t}}
\end{equation}
Considering a data assimilation window from time $t_0$ until time $t_K$ and 
having $\delta{x}(t_0)$ and $\delta{x_{lbc}}(t_K)$ as the assimilation control variables, the quantities needed for the LBC of the tangent linear WRF model are given by
\begin{equation}
\delta{x}_{lbc}(t_0)=\delta{x}(t_0)
\end{equation}
\begin{equation}
\frac {\partial{\delta{x}_{lbc}}} {\partial{t}}=\frac{\delta{x}_{lbc}(t_K)-\delta{x}(t_0)} {t_K-t_0}
\end{equation}
The lateral boundary conditions for the adjoint model, $x_{lbc}^{AD}(t_0)$ and $(\frac {\partial{x_{lbc}}} {\partial{t}})^{AD}$
, will be initialized with zeroes at the end of the data assimilation 
window (time $t_K$ ). After the backwards integration of the adjoint model to 
time $t_0$ the adjoint control variables (or the error gradients) can be obtained 
from: 
\begin{equation}
x^{AD}(t_0)=x_{inner}^{AD}(t_0)+x_{lbc}^{AD}(t_0)-\frac{1.}{t_K-t_0}(\frac{\partial{x_{lbc}}}{\partial{t}})^{AD}
\end{equation}
\begin{equation}
x_{lbc}^{AD}(t_K)=\frac{1.}{t_K-t_0}(\frac{\partial{x_{lbc}}}{\partial{t}})^{AD}
\end{equation}
where $x_{inner}^{AD}(t_0)$ denotes the �inner domain� adjoint model model variable as provided at the initial time $t_0$.

Note that $x_{lbc}^{AD}(t_0)$ and $(\frac{\partial{x_{lbc}}}{\partial{t}})^{AD}$ will be defined in the boundary relaxtion 
zone only. Consider, however, that they are full domain �elds that were 
initialized with zeroes at the the end of the data assimilation window and that the boundary relaxation will only �ll in values into this full domain 
�eld inside the boundary relaxtion zone.

Also note that the calculation of $J_{lbc}$ and $\frac{\partial{J_{lbc}}} {\partial{v_{lbc}}}$, where $v_{lbc}$ is the lateral boundary condition control variable in control vector space, will follow exactly the same calculations as for the background error constraint. 

\section{Code modification}

To introduce the new control variable of lateral boundary, we made following code modification in the latest WRFDA repository :

Add some variables definition for LBC 4DVAR.
\begin{verbatim}
M      var/da/da_control/da_control.f90
M      var/da/da_define_structures/da_define_structures.f90
\end{verbatim}
Update the output gradient function print out and the subroutine interface.
\begin{verbatim}
M      var/da/da_minimisation/da_calculate_gradj.inc
\end{verbatim}
Update the output cost function print out.
\begin{verbatim}
M      var/da/da_minimisation/da_calculate_j.inc
\end{verbatim}
Add calling \textsl{da\_transfer\_xatowrftl\_adj\_lbc} and \textsl{da\_transform\_vtox\_adj}
\begin{verbatim}
M      var/da/da_minimisation/da_transform_vtoy_adj.inc
\end{verbatim}
Update the subroutines' interface due to changes.
\begin{verbatim}
M      var/da/da_minimisation/da_minimise_lz.inc
M      var/da/da_minimisation/da_minimise_cg.inc
\end{verbatim}
Update the USE statements to accomodate the changes.
\begin{verbatim}
M      var/da/da_minimisation/da_minimisation.f90
\end{verbatim}
Add statements to call \textsl{da\_transform\_vtox} for 6-hour control variables.\\
Add statements to call \textsl{da\_transfer\_xatowrftl\_lbc}.
\begin{verbatim}
M      var/da/da_minimisation/da_transform_vtoy.inc
\end{verbatim}
Change cv\_size initialization due to $J_{lbc}$.
\begin{verbatim}
M      var/da/da_setup_structures/da_setup_cv.inc
\end{verbatim}
Minor changes to include more namelist variables in USE statement.
\begin{verbatim}
M      var/da/da_setup_structures/da_setup_structures.f90
\end{verbatim}
Include the new subroutines.
\begin{verbatim}
M      var/da/da_transfer_model/da_transfer_model.f90
\end{verbatim}
Add calling statement to output \textsl{wrfbdy\_af07}.
\begin{verbatim}
M      var/da/da_transfer_model/da_transfer_wrftltoxa_adj.inc
\end{verbatim}
Add the adjoint of converting WRFTL variables to analysis increments at boundary.
\begin{verbatim}
M      var/da/da_transfer_model/da_transfer_xatowrftl_adj.inc
\end{verbatim}
Adjoint of converting 6-hour analysis increments into WRFTL increments.
\begin{verbatim}
A      var/da/da_transfer_model/da_transfer_xatowrftl_adj_lbc.inc
\end{verbatim}
Convert 6-hour analysis increments into WRFTL increments.
\begin{verbatim}
A      var/da/da_transfer_model/da_transfer_xatowrftl_lbc.inc
\end{verbatim}
Clean-up.
\begin{verbatim}
M      var/da/da_transfer_model/da_transfer_wrftoxb.inc
\end{verbatim}
Change the subroutines' interface due to changes happen elsewhere.
\begin{verbatim}
M      var/da/da_test/da_check_vtoy_adjoint.inc
M      var/da/da_minimisation/da_adjoint_sensitivity.inc
M      var/da/da_minimisation/da_sensitivity.inc
\end{verbatim}
Input \textsl{wrfbdy\_ad01}, which is the output of the adoint model.
\begin{verbatim}
A      var/da/da_main/da_med_initialdata_input_lbc.inc
\end{verbatim}
Output \textsl{wrfbdy\_af07}, which is the LBC forcing for the adoint model.
\begin{verbatim}
A      var/da/da_main/da_med_initialdata_output_lbc.inc
\end{verbatim}
Read in \textsl{wrfbdy\_d01} when 4DVAR.
\begin{verbatim}
M      var/da/da_main/da_wrfvar_init2.inc
\end{verbatim}
Include the new subroutines \textsl{da\_med\_initialdata\_input\_lbc.inc} and \textsl{da\_med\_initialdata\_output\_lbc.inc}
\begin{verbatim}
M      var/da/da_main/da_wrfvar_io.f90
\end{verbatim}
Add statements to read \textsl{wrfbdy\_ad01} output from adjoint model.
\begin{verbatim}
M      var/da/da_main/da_med_initialdata_input.inc
\end{verbatim}
Initialize 6-hour $vv$ and $vp$.
\begin{verbatim}
M      var/da/da_main/da_solve.inc
\end{verbatim}
Add namelist variables to switch on/off LBC control of 4DVAR.\\
Add control variables space: $xa$, $vv$ and $vp$ for 6-hour.
\begin{verbatim}
M      Registry/Registry.wrfvar
\end{verbatim}

\section{Verification}
\begin{figure}
%\centering
\includegraphics[scale=0.9]{lbc_2}
\caption{{\bf With LBC Control }}
\label{default}
\end{figure}


\begin{thebibliography}{99}

\bibitem{Davis} Davies, H. C., 1983: Limitations of some common lateral boundary schemes 
used in NWP models.  \sl{Mon. Wea. Rev.}, \bf111\rm, 1002�1012.

\bibitem{Davis1977}Davies, H. C., and R. E. Turner, 1977: Updating prediction model by 
dynamical relaxation: An examination of the technique.  \sl{Quart. J. Roy. Meteor. Soc.}, \bf103\rm, 225-245.

\bibitem{Gusta} Gustafsson, N., E. Kallen and S. Thorsteinsson, 1998: Sensitivity of 
forecast errors to initial and lateral boundary conditions.  \sl{Tellus}, \bf50A\rm, 167- 185. 

\bibitem{kawa} Kawabata, T., Seko, H., Saito, K., Kuroda, T., Tamiya, K., Tsuyuki, 
T., Honda, Y. and Wakazuki, Y., 2007: An assimilation and forecasting 
Experiment of the Nerima Heavy rainfall with a cloud-resolving nonhydro- 
static 4-dimensional variational data assimilation system. \sl{J. Meteor. Soc. 
of Japan}, \bf85\rm, 255-276.

\end{thebibliography}

\end{document}